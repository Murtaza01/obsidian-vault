\documentclass[12pt, a4paper]{article}
\usepackage{polyglossia}
\usepackage{geometry}

\geometry{top=0.6in}
\title{Antigone}
\author{Sophocles}
\date{}

\newfontfamily\englishfont{Recia}
\setmainfont{Recia}
\newfontfamily\fontpar{Synonym}

\begin{document}
\maketitle

\section*{Antigone: the Female Stoic}

{\fontpar
From the right start we see Antigone (the protagonist) as a brave
and a virtuous person, a person who is willing to defend what is right
to his last breath, this bravery and absolute morality is similar to the
character of Socrates, reveling itself through her dialogues and reactions. 

}

\section*{Right Against Right}

{\fontpar
Antigone and Creon argument show a great and complex tension,
Creon thinks that because her
brother is an enemy he should not have the honour of burial, Antigone
thinks that they should bury people whoever they are and that her duty
complies her to to bury her brother, this is why i like this character
she is in touch with her values, both Creon and Antigone are right, both
are compiled by strong reasons.
}

 \section*{Creon Ruling}

{\fontpar
Creon is Challenged by Antigone actions, Creon is a just ruler,
he doesn't discriminate between his people, he realize that if he 
make an exception his ruling would be minimized, even when he knows that
the people agree on what Antigone did, he say that he is the ruler not 
the people, his son encounter this with "then you will be a good ruler in 
an island".
I think that Creon is right, because its  essential that the ruler 
isn't moved around by his people, yes he should take advice from them which Creon actually does at the end.
} 

\section*{Tragedy at it Best}

{\fontpar
After Creon order to kill Antigone and banish Ismene, his son who is 
in love with Ismene goes to the cave where she was banished, his father
goes after him after having been persuaded by an oracle to bury the brother
and not kill Antigone, when he arrive at the cave he sees Ismene hanged
and his son crying, he tries to talk to him but his son tries to his him
with his sword, he then kills himself in front of his father, dying next
to his beloved. Creon goes back to the city, the wife gets the news of her
son, and she kill herself as well, cursing Creon before her death, this
all was said through a messenger, even though Creon changes his mind after
been convinced by the oracle, it is too late.

}

\section*{The Better Tragedy}

{\fontpar
Unlike Oedipus, Antigone is more mature and complex, the rivalry between
Creon and Antigone has more wight than Oedipus and Creon, even though 
they have the same theme (pride), Creon had more reasons to keep going
with his rivalry. Antigone is a more interesting character then Oedipus,
and the complexity of the story as it unravels is more engaging. The play
is very quotable and has fantastic dialogues, the only reason why Oedipus is more popular is because Aristotle really liked
it and mentioned it in his book Poetics as one of the best tragedies.

}

\section*{Pride}

{\fontpar
Sophocles loves to talk about pride, in both plays pride is the main
theme, Oedipus is too prideful to listen to the oracle and accept his
fate, Creon is too prideful to listen to Antigone and the gods commands,
both face consequences as a result, what makes both plays a tragedy is
that both characters couldn't really accept their fate, who would accept
the fact that he is going to kill his father and marry his mother?

}
\end{document}
