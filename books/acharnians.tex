\documentclass[12pt, a4paper]{article}
\usepackage{polyglossia}

\title{The Acharnians}
\author{Aristophanes}
\date{}

\newfontfamily\englishfont{Recia}
\setmainfont{Recia}
\newfontfamily\fontpar{Synonym}

\begin{document}
\maketitle

\section*{The Chorus}
{\fontpar
Like most of The chorus in Aristophanes play, they are alright
bit mediocre really, sometimes Aristophanes uses rhyming of (ABCB)
Which is lyrical at reading at first, but gets boring and 
unpoetic after a few lines, although there is some good chorus here
and there, but most of the time it is normal.
}

\section*{Politics}

{\fontpar
Yes the play is about war and of course it will include a lot of 
politics, since it is really just an appeal for peace, Aristophanes 
was really just trying to convince the people responsible for the 
war to end it. However for us readers we really need an efficient
background on the Athenian and Spartan politics in order to
understand some characters and some jokes, which makes the 
experience a bit frustrating.
}

\section*{Euripides}

{\fontpar
Euripides appears as a character in the play, although he appears
as himself, i really liked the scene which he appears in, he uses 
archaic English and poetic language, Aristophanes uses him to
mock both the poet and his works
}

\section*{Imitation of Telephus}

{\fontpar
Aristophanes quotes Euripides' Telephus throughout the play,
not only to mock but to also show his inspiration and knowledge 
about his contemporaries, its funny because some of the 
best verses are the ones taking from Telephus.
}

\section*{The Best Scene}

{\fontpar
There is this scene in the play, where a father of two daughters
approach Dikaiopolis (the protagonist) to buy food as the latter had a stall
to sell some stuff, the father comes from one of the effected 
countries by the war, and he doesn't have any money to buy food 
for his two daughters. In a comical manner he tell his daughter 
that its either they starve to death or he sell as pigs to 
Dikaiopolis in exchange of some food,the scene is both sad and
hilarious, which shows how Aristophanes can make these humane scene 
in comical way. 
}

\end{document}
