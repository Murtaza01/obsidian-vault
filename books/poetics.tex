\documentclass[12pt, a4paper]{article}
\usepackage{geometry}
\usepackage{lua-ul}
\usepackage{graphicx}
\usepackage{enumitem}
\usepackage{color,soul}
\usepackage{enumitem}
\usepackage{defaultpreamble}


\setmainfont{Bembo Book MT Pro}
\setlength\parindent{0pt}

\begin{document}

\head{Introduction}

Poetics is the first literary criticism. Plato did critique literature in his Republic but in a fragmented 
fashion.

\subhead{Three ethical treatises}

Aristotle wrote three treatises on ethics: Eudemian Ethics, Magna Moralia and Nicomachean Ethics.
Eudemian Ethics is Aristotle's first book on ethics, yet it is less popular because
it was not available to most of the medieval philosophers and it is less complete than 
NE.\medbreak

\textbf{Pindar:} Was the most famous Greek lyric poet (518–446). His most popular work
--The Odes-- was a collection of victory songs
to celebrate winners in athletic games.\medbreak

\textbf{Aeschylus} (525– 456) who had fought in the Persian wars
was the first great writer of tragedy (before Euripides and Sophocles).\medbreak

\subhead{Difference Between Poetry and Prose}

To Aristotle, it is content not form that distinguishes poetry from prose. For example Nietzsche's 
work uses metaphors and Empedocles\footnote{\, Greek pre-Socratic philosopher.}
wrote in hexameter but their works are not considered poetry
because of the content and purpose. Poetry involves imaginative writings; it makes us feel not just understand.

\subhead{Art: Twice Removed from Reality}

Plato wanted not just poetry\footnote{\, actually back in the days, poetry meant art.}
but art altogether to be banned from his city because it was twice removed from reality.
To Plato our world is an imitation of the perfect world (The Forms). Thus art is an imitation
of the imitation. He also thought that art appealed to the emotions instead of reason. 
To Aritotle, imitation is a natural human tendency that we use from our childhood, for example
when we mimic our parents. Also it brings us delight, for example we enjoy and admire paintings of 
objects that in themselves would annoy or disgust us.

\subhead{Plato x Aristotle on Poetry}
Plato ranked philosophy first, then history and after it poetry.
Aristotle, however, offers a different ranking: philosophy, poetry, and history. He
does so on the basis that poetry is more philosophical than history, since it
deals with universals rather than particulars. This is similar to Henry Fielding's
theory of novel. He explains that he depicts universal characters 
not men. This actually make sense, since Fielding does quote Aristotle a lot.

Aristotle saw that tragedy reached its perfect form in the works of Sophocles. His tragedies is the answer
to the question ''What is tragedy for ?'' Most critics disagree with Aristotle's idea that Art has
a final goal or distention.

\subhead{Plot, The Soul of Tragedy}
Aristotle thought that the plot is the soul of tragedy, and that characters must serve the plot.
Breaking bad follows Aristotelian concept, every action Walter does serve the plot; there is
no sub-plots. The transformation of Walter is the plot. 
This is different from The Sopranos, where Tony's psyche is the subject.
There is really no specific plot in The Sopranos, we are only discovering the psychology 
of characters. And there is no cause and effects chain of action, which Breaking Bad does:
every action has its consequences.

Aristotle says that emotions are: feelings that alter people’s judgements, and they are
accompanied by pain and pleasure.

Aristotle thought that the poet has an ethical responsibility, that art should be morally good (that
art should serve morality).
Nietzsche disagrees with this: he sees art to be its own thing, and even superior to morality,
because art expresses the will to power. 

Aristotle preferred complex tragedy to other forms of art, because he thought that the ultimate 
purpose of art is to purify fear and pity. Fear to not do what the protagonist, and pity for his
situation. This is seen best in Oedipus King. He thought that the Odyssey does not depicts this.
The odyssey follows a double issue pattern, where the good gets rewarded and the bad gets punished,
and there is a happy ending. To Aristotle this was a second class plot because it does not inspire
fear or pity in the audience.

Averroes (Ibn Rushed) thought that a Muslim philosopher (Imam) should rule the city. He
was influenced by Plato's Republic even though he could not read the actual text because
it was in Greek and there was no English translation. English wasn't even widespread, and the
first translation of The Republic was in 1701 by Thomas Taylor\footnote{\, the first to translate into English 
the complete works of Aristotle and of Plato.}. So Ibn Rushed ended to do his work on Plato's 
Republic using fragments of Arabic translation.

Percy Bysshe Shelley was English Romantic poet who wrote The Necessity of Atheism while he was 
in Oxford which got him to be expelled. His second wife Mary Shelley wrote Frankenstein. 

Aristotle says that Homer's serious epics (Iliad and Odyssey) were the foundational models to tragedy.
And his lost work (The Margites) was the foundation for comedy. Aristotle explains that Margites was 
not just a lampoon like the other comic poems, but rather a dramatization of the ridicules.

\end{document}
