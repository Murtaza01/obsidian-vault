\documentclass[12pt, a4paper]{article}
\usepackage{polyglossia}
\usepackage{geometry}
\usepackage{lua-ul}
\usepackage{color,soul}
\usepackage{xcolor}
\usepackage{hyperref}
\usepackage{setspace}
\usepackage[fonthead=Fustat]{defaultpreamble}

\doublespacing
%\onehalfspacing
\setlength\parindent{0pt}

%fonts
\setotherlanguage{arabic}
\newfontfamily\arabicfont[Script=Arabic, Scale=1.1]{ِAmiri}
\newfontfamily\arabichead[Script=Arabic, Scale=1]{Reem Kufi}

\begin{document}


\begin{otherlanguage}{arabic}

\begin{center}
{\arabichead\Large
  
  \textbf{ القرعة 8491}\bigbreak

  بقلم شيرلي جاكسون 
}\bigbreak
\end{center}

صباح السابع والعشرين من يونيو كان صافياً ومشمساً بدفء عليل كأواسط الصيف.  كانت الزهور مشرقة لحد الاسراف
والعشب غني بلخضرة. بدأ أهالي القرية في التجمع في الساحة، بين مكتب البريد والمصرف، حوالي الساعة العاشرة;
ففي بعض البلدات كان عدد الناس هائل لدرجة أن اليانصيب كان يستغرق يومين مما اضطرهم  الى البدء به في السابع والعشرين 
من يونيو. أما في هذه القرية، حيث كان عدد السكان نحو ثلاثمئة نسمة فقط، فقد استغرقت اليانصيب بأكملها أقل من 
ساعتين، لذا كانت تبدأ عند العاشرة صباحًا وتنتهي في الوقت المناسب ليتمكن القرويون  من العودة لوجبة منتصف النهار. 

تجمع الاطفال اولاً, بطبيعة الحال. فقد انتهت المدارس مؤخراً للصيف, وكان شعور الحرية يثقل اغلبهم; اذ مالو الى
التجمع بهدوء لبعض من الوقت قبل ان يندفعوا في اللعب الصاخب, وبقي حديثهم عن الفصل والمعلم, الكتب والتوبخيات.
كان بوبي مارتن قد عبئ جيوبه بلحجارة حتى امتلات, وسرعان مااخذ الاطفال يقلدوه, منتقين اكثر الحجر استدارة واملسها.
ثم شكّل بوبي وهاري جونز وديكي ديلاكروا — وكان أهل القرية ينطقون اسمه "ديلاكروي" — كومةً كبيرةً من الحجارة في 
زاويةٍ من الساحة، وحموها من غارات الصبية الآخرين. 

وسرعان ما بدء الرجال بلتجمع, يفحصون اطفالهم, ويتحثون عن الزراعة والمطر, الجرارات والضرائب. وقفوا معاً, بعيدين
من كومة الحجارة في الزاوية, وكانت نكاتهم خافتة يبتسمون من دون ضحك. وبمدة ليست بطويلة لحقت النساء برجالهن, 
يرتدين اثواباً منزلية باهتة ومعاطف صوف. تبادلو التحايا والنميمة بينما كانوا ينضمون الى ازواجهم. وسرعان ما بدأت
النساء, الواقفات بجانب ازواجهن, بنداء اطفالهن, فاقبل الاطفال كرهاً, بعد نداءهم للمرة الرابعة او الخامسة.
تجنب مارتن بوبي يد أمه المحاولة الإمساك به وركض, ضاحكاً, نحو كومة الحجارة. فصاح به والده بلهجة صارمة، 
فأسرع بوبي إلى مكانه بين والده وشقيقه الأكبر. 

\newpage
اُجريت اليانصيب — كما كانت الرقصات الرباعية, نادي المراهقين, وبرنامج الهالوين — تحت اشراف السيد سامرز
الذي كان لديه الوقت والهمة لتكريسها للانشطة المدنية. كان مستديرَ الوجه، بشوش,  يدير تجارة 
الفحم, وكان الناس يشعرون بالأسف عليه لأنه لم يُرزق بأطفال وزوجته سليطة. عند وصوله الى الساحة, حاملاً صندوق
خشبي اسود, كان هناك دمدمة بين القروين, فأومأ بيده منادياً: ”تأخرنا قليلاً اليوم, يا قوم.“ ساعي البريد, 
السيد غريفرز, تبعه حاملاً  كرسياً ثلاثي القوائم الذي وضِع  وسط الساحة وثبت  السيد سامرز الصندوق الأسود عليه.
حافظ القروين على مسافتهم, تاريكين مسافة بينهم وبين الكرسي, وعندما قال السيد سامرز ”أيستطيع احد مساعدتي؟“
كان هناك تردد بين رجلين. السيد مارتن وابنه الاكبر, باكستر, تقدما ليمسكا بلصندوق ويثبتاه على الكرسي بينما
قلب السيد سامرز الاوراق داخله. فُقدت الادوات الاصلية لليانصيب منذ زمن بعيد, والصندوق الاسود الموجود الان
على الكرسي قد بدء استخدمه حتى قبل ولادة العجوز وارنر, اكبر رجل في البلدة. كان السيد سامرز يُكلِّم القرويين 
كثيرًا عن صنع صندوق جديد, لكن لم يرد احد ان يتلاعب حتى بقدر ضئيل بتقاليد الصندوق الاسود. كان هناك قصة تقول
ان الصندوق الحالي قد صنع من بعض قطع الصندوق السابق, ذلك الذي صنع عندما قرر الاجداد الاوائل الاستقرار هنا.
كل عام, بعد انتهاء اليانصيب, كان السيد سامرز يعيد الكلام عن الصندوق الجديد, ولكن الموضوع كان يترك ليختفي
دون ان يتخذ اي اجراء. ازدادت رثاثة الصندوق الاسود كل عام فلم يعد اسود بالكامل, بل تشقق على جانب واحد حتى
بدا يظهر لون خشبه الاصلي, وفي جوانب اخرى بدا باهتاً او ملطخاً.

السيد مارتن وابنه الاكبر, باكستر, امسكوا الصندوق الاسود على المقعد باحكام حتى قلّب السيد سامرز الأوراق بدقة بيده.
وبسبب التخلي ونسيان الكثير من الطقوس, فقد نجح السيد سامرز في استبدال قصاصات الورق بقطع الخشب التي استخدمت عبر الأجيال.
جادل السيد سامرز ان قطع الخشب كانت جيدة بما يكفي في زمن كانت فيه القرية صغيرة,  لكن مع تجاوز عدد السكان الثلاثمائة 
نسمه ومن المرجح ان يستمر في النمو,  اصبح من الضروري استخدام شيء يتسع بسهولة اكبر في الصندوق الاسود. في الليلة السابقة
للقرعة, أعد السيد سامرز والسيد غريفز قصاصات الورق ووضعوها في الصندوق, ثم نقل الصندوق إلى خزنة شركة السيد سامرز للفحم 
وأقفل عليها الى حين ان استعداد  السيد سامرز لأخذه إلى الساحة صباح اليوم التالي. أما بقية العام، فكان الصندوق يوضع 
جانبا، تارة في مكان وتارة في آخر; فقد قضى عاماً  في مخزن السيد غريفز وعاماً  آخر تحت الأقدام في مكتب البريد وأحيانا 
كان يوضع على رف في بقالة مارتن ويترك هناك. كان هناك الكثير من الضجة التي تحدث قبل ان يعلن السيد سامرز عن بدء القرعة.
تطلبت التحضيرات وضع قوائم تشمل: ارباب العائلات، ورب كل أسرة في كل عائلة، وأفراد كل أسرة في كل عائلة.
كان هناك قسم رسمي يؤديه السيد سامرز من قبل مدير البريد، بصفته المسؤول الرسمي عن القرعة. في وقت ما، كما يتذكر بعض 
الناس، كان هناك نوع من الترتيل, يؤديه مسؤول, ترتيل روتيني, خالي من اللحن يلقى باتقان كواجب سنوي.
اعتقد بعض الناس ان مسؤول القرعة كان يقف في مكانه عند الاداء, فيما ظن اخرون انه كان يفترض به السير بين الحضور,
غير ان هذا الجزء من المراسيم قد اندثر منذ عقود. كان هناك, ايضاً, تحية رسمية, واجبه على مسؤول القرعة عند مخاطبة كل شخص
يأتي لسحب ورقة من الصندوق, لكن هذه العادة تغيرت هي الاخرى مع الزمن، حتى اصبح المسؤول يكتفي بأن يخاطب كل شخص يقترب منه.
كان السيد سامرز بارعا في كل هذا؛ بقميصه الأبيض النظيف وجينزه الأزرق, ويدٍ تستند بلا اكتراث على الصندوق الأسود، 
ليبدوا في منتهى الرسمية والاهمية وهو يتحدث بلا انقطاع للسيد غريفز وعائلة مارتن. في اللحظة التي توقف فيها السيد 
سامرز عن الكلام والتفت الى القرويين المجتمعين، أتت السيدة هاتشينسون مسرعة على الدرب نحو الساحة، ، سترتها ملقاة على 
كتفيها, ثم انسلت خلسة الى مكانها في الصفوف الخلفية. ”نسيت تماماً اي يومٌ هوة اليوم“ قالت للسيدة ديلاكروا, الواقفة بجانبها,
فضحكتا معا بهدوء. ”ظننت ان زوجي في الخلف يكدس الحطب“ تابعت السيدة هاتشينسون, ” ثم نظرت من النافذة فلم أجد الاولاد، 
فتذكرت انه السابع والعشرون فجئت مسرعة.“  نشفت يديها بمئزرها، فقالت السيدة ديلاكروا: ”مع ذلك لقد وصلتي في الوقت المحدد.
فهم لا يزالون يتحدثون هناك في الاعلى.“ مدت السيدة هاتشينسون رقبتها لترى بين الحشود فوجدت زوجها واطفالها بلقرب من المقدمة.
ودعت السيدة ديلاكروا بلمسة على ذراعها وبدأت تشق طريقها بين الجمهور. تنحى الواقفون بروح الدعابة ليفسحوا لها المجال.
وقال شخصان او ثلاثة بأصوات عالية بما يكفي لسماعها عبر الحشد, ”ها قد أتت سيدتك يا هاتشينسون“ و 
”بيل، ها هي قد وصلت بعد كل شيء.“ وصلت السيدة هاتشينسون الى زوجها, والسيد سامرز, الذي كان ينتظر, قال بمرح 
”لقد خشينا ان نضطر للاستمرار بدونك, يا تسلي“ قالت السيدة هاتشينسون وهي تبتسم: ”ما كنتم لتسمحوا لي بترك صحوني في 
المغسلة,  اليس كذلك, يا جو؟“  وانتشرت ضحكات خفيفة بين الحشد بينما عاد الناس يتحركون الى مواقعهم بعد وصول السيدة هاتشينسون.
”حسناً الان“ قال السيد سامرز بجدية ”اظن انه الافضل ان نبدأ, وننتهي من هذا الامر, حتى نستطيع العودة الى العمل. هل من غائب؟“









\end{otherlanguage}
  
\end{document}
