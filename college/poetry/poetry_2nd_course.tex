\documentclass[12pt, a4paper]{article}
\usepackage{polyglossia}
\usepackage{geometry}
\usepackage{lua-ul}
\usepackage{color,soul}
\usepackage{verse}
\usepackage{xcolor}
\usepackage{hyperref}

\geometry{bottom=1in, top=1in}
\setlength\parindent{0pt}
\newcommand{\attrib}[1]{\nopagebreak{\raggedleft\footnotesize #1\par}}

%fonts
\setmainfont{Alegreya}
\newfontfamily\fontverse{Alegreya-Bold.otf} 
\newfontfamily\fonthead{Cinzel Decorative}


%sections
\newcommand{\head}[1]{
  \phantomsection
  \section*{\centering{\fonthead{#1}}}
  \addcontentsline{toc}{section}{#1}
}


\newcommand{\subhead}[1]{
  \phantomsection
  \subsection*{#1}
  \addcontentsline{toc}{subsection}{#1}
}

\newcommand{\subsubhead}[1]{
  \phantomsection
  \subsubsection*{#1}
  \addcontentsline{toc}{subsubsection}{#1}
}

\newcommand{\poemhead}[1]{
  \phantomsection
  \subsubsection*{\centering{\large{#1}}}
  \addcontentsline{toc}{subsubsection}{#1}
}

\begin{document}
\newgeometry{top=0.7in,bottom=1in}
%{\fonthead
%\begin{center}
%\huge{\textbf{Poetry Second Course}}\\[0.2cm]
%\Large{By: Birdman \& Fried Potato \& Mohamed}\\[0.3cm]
%\end{center}
%}

\input{poetry_first.tex}

\subhead{Edmund Waller}
Edmund was part of the parliament and was against the king, he
wrote a poem in praise of Oliver Cromwell. However he changed sides
during the civil war and became part of the royalist (Cavalier poets).

\poemhead{Song: Go, Lovely Rose}
\settowidth{\versewidth}{When I resemble her to thee}
\begin{verse}[\versewidth]
{\fontverse
Go, lovely rose!\\
Tell her that wastes her time and me,\\
That now she knows,\\
When I resemble her to thee,\\
How sweet and fair she seems to be.
} 
\end{verse}

The poet sends the rose as a messenger to tell his beloved; to stop wasting his and her time
and that she is as beautiful as the rose.
\textbf{Apostrophe} is used to address the rose in \textit{"Go, lovely rose"}.
\textbf{Metaphor} in \textit{"resemble her to thee"} comparing a rose to a woman.

\begin{verse}[\versewidth]
{\fontverse
Tell her that’s young,\\
And shuns to have her graces spied,\\
That hadst thou sprung\\
In deserts, where no men abide,\\
Thou must have uncommended died.
} 
\end{verse}

The poet asks the rose to tell his beloved; that if she hides her beauty
like a rose in a dessert she will die unnoticed and without anyone praising her beauty.

\begin{verse}[\versewidth]
{\fontverse
Small is the worth\\
Of beauty from the light retired;\\
Bid her come forth,\\
Suffer herself to be desired,\\
And not blush so to be admired.
} 
\end{verse}

If you have beauty and its not seen, then it has no worth. The poet
tells the rose to invite his beloved to go out and show her beauty, she 
should not be shy to be desired and admired.\footnote{because this poem was 
written at the restoration time, it is a vulgar request for girls to show
their beauty.}

\begin{verse}[\versewidth]
{\fontverse
Then die! that she\\
The common fate of all things rare\\
May read in thee;\\
How small a part of time they share\\
That are so wondrous sweet and fair!
} 
\end{verse}

The poet tell the rose to die so that his beloved may learn that 
beauty only last for a brief time and is distend to fade.

\subsubhead{\textit{Go, Lovely Rose} Summary}

Edmund uses \hl{Apostrophe} to address the rose as 
his messenger. He tell the rose to encourage his beloved and invite her to show her beauty and to not be
shy to be admired, because beauty that is not seen has little worth. He tell the rose to die so 
that his beloved may learn that all this is the destiny of all beautiful things. This poem 
is \hl{carpe diem}.

\subhead{John Suckling}

One of the leaders of the Cavalier poets and part of the school
of Ben Jonson.

\poemhead{Song: Why so Pale?}
\settowidth{\versewidth}{Why so pale and wan fond lover?}
\begin{verse}[\versewidth]
{\fontverse
Why so pale and wan fond lover?\\
\vin Prithee why so pale?\\
Will, when looking well can't move her,\\
\vin Looking ill prevail?\\
\vin Prithee why so pale?
} 
\end{verse}

Why you are so pale foolish lover? If looking good 
did not move her, you think looking ill will? Here
the poet asks a \textbf{rhetorical} question that 
serve as a challenge to the lover action.

\begin{verse}[\versewidth]
{\fontverse
Why so dull and mute young sinner?\\
\vin Prithee why so mute?\\
Will, when speaking well can’t win her,\\
\vin Saying nothing do't?\\
\vin Prithee why so mute?
}
\end{verse}

Why so dull and mute? When you spoke well you did not
move her saying nothing does? \textit{"young sinner"} refers
to the idea of original sin\footnote{In Christian we are born 
sinners because of the first sin (Adam and Eve eating from the tree of knowledge).}

\begin{verse}[\versewidth]
{\fontverse
Quit, quit for shame, this will not move,\\
\vin This cannot take her;\\
If of herself she will not love,\\
\vin Nothing can make her;\\
\vin The devil take her.
} 
\end{verse}

Stop and have some dignity this will not work, this cannot move her.
If she did not love on her own then nothing will force her. Forget about
her "let her go to hell".

\subsubhead{\textit{Why so Pale} Summary}

In this poem the poet asks a \hl{rhetorical question} to challenge
the lover actions and make him think and reflect on them. He 
tells him that if she does not love you on her own then you are wasting
your time, just quit. This poem is taken from a play and is about 
\hl{unrequited love}

\subhead{Richard Lovelace}

One of the Cavalier poets and close friend to John Suckling and part
of the school of Ben Jonson. He was accustomed to court life and
spent some time in prison.

\poemhead{To Althea, from Prison}
\settowidth{\versewidth}{When Love with unconfinèd wings}
\begin{verse}[\versewidth]
{\fontverse
When Love with unconfinèd wings\\
Hovers within my Gates,\\
And my divine Althea brings\\
To whisper at the Grates;\\
When I lie tangled in her hair,\\
And fettered to her eye,\\
The Gods that wanton in the Air,\\
Know no such Liberty.
} 
\end{verse}

Even though I'm trapped in my cell, love hovers around me
and Althea comes to visit me. Not even the gods know such
freedom. Here the poet tries to say that even though
he is in prison cell, his imagination is not, his soul
is free. \textbf{Personification} in \textit{"Love with unconfinèd wings"}.

\begin{verse}[\versewidth]
{\fontverse
When flowing Cups run swiftly round\\
With no allaying Thames,\\
Our careless heads with Roses bound,\\
Our hearts with Loyal Flames;\\
When thirsty grief in Wine we steep,\\
When Healths and draughts go free,\\
Fishes that tipple in the Deep\\
Know no such Liberty.
} 
\end{verse}

When our cups of wine is passed around and we are in a careless state, when 
our heart is filed with love to the king, when our grief is drown in wine and
we make a toast for our health \textit{then} even the fishes that drink from the depth
of the ocean does not know such freedom.


\begin{verse}[\versewidth]
{\fontverse
Stone Walls do not a Prison make,\\
Nor Iron bars a Cage;\\
Minds innocent and quiet take\\
That for an Hermitage.\\
If I have freedom in my Love,\\
And in my soul am free,\\
Angels alone that soar above,\\
Enjoy such Liberty.
} 
\end{verse}

A cell in jail do not make a prison, nor the iron in the cell.
A mind that is quite and calm is like a Hermitage (A place where
religious people go in isolation to pray). If I'm free to love
and my soul is free, then only Angels have the same freedom that i have.

\subsubhead{\textit{To Althea, from Prison} Summary}

In this poem the poet tries to distinguish between physical 
and spiritual freedom. He says that even in jail he is still
free; his imagination/spirit can go anywhere. In the first stanza 
he talks about the sky and his love for Althea. In the second stanza
he talks about the sea and his love for the king. In the final
stanza he talks about his cell being like a Hermitage. He says that
not \textit{The gods in the sky} nor \textit{The fishes in the deep sea} 
are as free as he, but only the Angels that fly above.

\input{poetry_second.tex}

\newpage
\head{Past Exam Sheets \& Answers}\bigbreak

\subhead{First Exam, A}\bigbreak

\subsubhead{Question 1: Fill in the blanks (6 Marks)}

\begin{enumerate}

  \item The Civil War (1642-1649) was between \underLine{Charles I} and
    \underLine{The Parliament}.

  \item The most important schools of poetry in the seventeenth century are
    \underLine{School of Ben Jonson} and \underLine{School of Jon Donne}.

  \item Among the most important features of Ben Jonson's poetry are
    \underLine{logic}, \underLine{wit} and \underLine{clarity}.
  
\end{enumerate}

\subsubhead{Question 2: Select the correct answer (4 Marks)}

\begin{enumerate}

  \item \textbf{The "School of Ben Jonson" is known for emphasizing which of 
    the following in their poetry? } 

  A) Pastoral themes. \\ 
  B) Metaphysical conceits. \\
  \hl{C) Classical form and decorum.} \\
  D) Romantic imagery.
  
 \item \textbf{In Jonson's poem "Come, My Celia, Let Us Prove" What is the 
   main theme of the poem?} 

 \hl{A) The brevity of life and the importance of sizing the moment.}\\ 
  B) The pain of unrequited love. \\
  C) The exploration of nature's beauty. \\
  D) The pursuit of knowledge.
  
\item \textbf{Robert Herrick's "Delight in Disorder" primarily explores the 
  theme of?} 

  A) The virtue of order and control. \\ 
\hl{B) The appeal of disorder and imperfection.} \\
  C)  The pain of lost love.\\
  D) The pursuit of spiritual purity.

  \item \textbf{In Robert Herrick's "To daffodils" the daffodils symbolize?} 

  A) Immortality and eternal life. \\ 
\hl{B)  The fleeting nature of youth and beauty.}\\
  C)  The coming of spring and renewal.\\
  D) The permanence of nature.
  
  
\end{enumerate}

\subsubhead{Question 3: Answer Two of the following (10 Marks)}\bigbreak

\textbf{A) Comment on the most important features of the poetry of the 
  school of Ben Jonson.}\medbreak

The school of Ben Jonson were highly influenced by the classics (Roman and Greek). 
Their poetry were simple and plain, using easy and clear language. They also 
used Logic and wit, didacticism and instruction, realism and 
the use of controlled feelings in their poetry.\bigbreak

\textbf{B) Comment on Ben Jonson's "On my First Son" as an elegy. }\medbreak

In the poem "On my First Son" Ben mourns the death of his son at the age 
of seven. His love for his son were so great that he says he will never 
love anything like him again. The poem is a great example of an elegy because 
it shows the pain and suffering of the poet through the sad tone of the poem.\bigbreak

\textbf{C) The theme of carpe diem is very common in English 17th century poetry.
Discus with reference to one poem of your choice.}\medbreak

Carpe diem is a Latin term which means seize the moment/day. It was a reaction to the Puritans 
rule in the Restoration period. A great example of carpe diem philosophy is Ben Jonson's 
\textit{Song to Celia}. In this poem Ben urges his beloved to enjoy love while they can.
He tries to convince her of the shortness of time and youth
and how they should seize it in private while they can.


\subhead{First Exam, B}\bigbreak

\subsubhead{Question 1: Fill in the blanks (6 Marks)}

\begin{enumerate}

\item \textbf{Carpe diem means:} Seize the day/moment.
  \item \textbf{The cavalier poets are:} Robert Herrick, John Suckling,
     Richard Lovelace and Edmund Waller.
   \item \textbf{A very clear example of elegy can be found in:} Song:
     To my First Son.
  
\end{enumerate}


\subsubhead{Question 2: Select the correct answer (4 Marks)}

\begin{enumerate}

 \item \textbf{Which of the following best describes the style of 
   Robert Herrick's poetry?} 

  A) Metaphysical and intellectual. \\ 
  B) Classical and formal. \\
  \hl{C) Lyric, light and celebratory.} \\
  D) Dark, somber and introspective.
  

 \item \textbf{In "To the Virgins, to Make Much of Time" the poet advises 
   the virgins to:} 

  A) Save their beauty for the right man.  \\ 
  \hl{ B) Marry young to avoid regret.}  \\
  C)  Live a life of purity and chastity. \\
  D) Worship nature and its beauty.

 \item \textbf{Ben Jonson's poetry is known for its:} 

  A) Use of elaborate metaphysical conceit. \\ 
  B)  Pursuit of simple, natural language.\\
\hl{C) Classical and reference and adherence to form.} \\
  D) Emphasis on personal, emotional expression.

 \item \textbf{The phrase "Come, My Celia, Let us prove" in Ben Jonson's 
   poem is an example of which literary technique?} 

   \hl{A) Apostrophe.} \\ 
  B) Hyperbole. \\
  C) Metaphysical conceit. \\
  D) Allusion.
  
\end{enumerate}

\subsubhead{Question 3: Answer Two of the following (10 Marks)}\bigbreak

\textbf{A) Explain the following lines "Seven years thou were lent to me
and i pay thee/ Exacted by thy fate on the just day"}\medbreak

Seven years you (my son) were lent to me by fate, and then it took you on the exact day you were
born (your birthday). In these lines the poet mourns the death of his son. He says that 
the things we own (children, wealth, etc.) is lent to us and someday will be taken 
away from us.\bigbreak


\textbf{B) Describe the poetry of Ben Jonson and his followers.}\medbreak

Ben Jonson's poetry was influenced by the classics (Roman and Greek).
The features of his poetry were:

\begin{itemize}
  \item Clarity, brevity, simplicity and order.
  \item Realism and the use of controlled feelings.
  \item Logic and wit.
  \item Didacticism and instruction.
  \item Refinement of the classics. 
\end{itemize}

\textbf{C) Who are the Roundheads and who are the Royalists?}\medbreak

The Roundheads were the supporters of Parliament. They were primarily common people 
such as merchants and tradesmen. The Royalists were the supporters of the king. They 
were people who lived a courtly life near the king. 


\end{document}
