\documentclass[12pt, a4paper]{article}
\usepackage{polyglossia}
\usepackage{geometry}
\usepackage{lua-ul}
\usepackage{color,soul}
\usepackage{verse}
\usepackage{xcolor}
\usepackage{hyperref}

\geometry{bottom=1in, top=1in}
\setlength\parindent{0pt}
\newcommand{\attrib}[1]{\nopagebreak{\raggedleft\footnotesize #1\par}}

%fonts
\setmainfont{Alegreya}
\newfontfamily\fontverse{Alegreya-Bold.otf} 
\newfontfamily\fonthead{Cinzel Decorative}


%sections
\newcommand{\head}[1]{
  \phantomsection
  \section*{\centering{\fonthead{#1}}}
  \addcontentsline{toc}{section}{#1}
}


\newcommand{\subhead}[1]{
  \phantomsection
  \subsection*{#1}
  \addcontentsline{toc}{subsection}{#1}
}

\newcommand{\subsubhead}[1]{
  \phantomsection
  \subsubsection*{#1}
  \addcontentsline{toc}{subsubsection}{#1}
}

\newcommand{\poemhead}[1]{
  \phantomsection
  \subsubsection*{\centering{\large{#1}}}
  \addcontentsline{toc}{subsubsection}{#1}
}

\begin{document}
\newgeometry{top=0.7in,bottom=1in}
%{\fonthead
%\begin{center}
%\huge{\textbf{Poetry Second Course}}\\[0.2cm]
%\Large{By: Birdman \& Fried Potato \& Mohamed}\\[0.3cm]
%\end{center}
%}

\head{Historical Background of the Seventeen Century}

\subhead{King x Parliament}

Both \textit{James I} and his son \textit{Charles I} thought that kings ruled 
by \underLine{Divine Right}. They abused the power of ruling, made illegal taxes
on working people. They were in constant disputes with the Parliament and tried 
many times to rule without them which led to the civil war.

\subhead{Civil War (1642-1649)}

The civil war happened between \textit{Charles I} and his supporters 
(\underLine{the Royalist}) against the Parliament and their supporters which were
merchants and tradesmen, they were of the common people and were called (\underLine{the Roundheads}). 
With the help of Scotland the Parliament defeated the king and executed him in 1649.

\subhead{The Puritans}

In the seventeen century there were two major religious groups: the
\textit{Puritans} who were very strict and thought that all entertainment is
distasteful to God. And the \textit{Catholics} who wanted the Pope in Rome to 
be the head of the church.\medbreak

In the period (1649-1660) \underLine{Oliver Cromwell} (who was a Puritan) ruled England. He closed the 
theaters and other entertainments because he thought they were distraction
from the Bible. In the \underLine{Restoration period} after Oliver rule was over, 
people started to be more vulgar and indulgent as a reaction to the Puritan's strict rules.

\head{School of Ben Jonson}

The school of Ben or the \textit{Tribe of Ben} were group of poets who imitated
Ben Jonson style. They were Royalist and were called \underLine{Cavalier Poets} because they
supported king \textit{Charles I} against the Parliament. This group included: 
\underLine{Thomas Carew}, \underLine{Richard Lovelace}, 
\underLine{John Suckling}, \underLine{Robert Herrick}. \hl{Ben and his followers often wrote 
carpe diem\footnote{\, seize the day/moment. Its a philosophy that recognizes the brevity of life
and the need to live in the moment.} poems}.\medbreak

\subhead{Ben Jonson}

Playwright, critic and poet. Was part of the Royalist who supported 
king Charles I. He lived in the city, 
so most of \underLine{his work were about 
the city side} (urban side). He was influenced by classical 
literature (Greek and Roman) and imitated their style. He used
\textbf{classicism}\footnote{\, the following of Greek and Roman style and 
principles in literature.} 
in his form. \underLine{\textbf{John Donne} was his rival-artist}, his 
form was complex, scientific and included metaphysical elements, unlike
Ben's form which was simple and plain.


\subsubhead{Characteristics of Ben's School}

\begin{itemize}
  \item Clarity, brevity, simplicity and order.
  \item Realism and the use of controlled feelings.\footnote{\, describing things in realistic  manner and staying away 
    from exaggeration.}
  \item Logic and wit.
  \item Didacticism\footnote{\, is a type of literature that aims to teach}
    and instruction.
  \item Refinement of the classics. 
\end{itemize}


\subsubhead{Ben Jonson Poems} 
Ben favored the \hl{shorter forms} in his work, such as: \underLine{the epigram}
, \underLine{the epitaph}, \underLine{the elegy} and 
\underLine{the epistle}.\medbreak


\textbf{Epigram:} A short poem consisting of 2 line verse that teaches a moral lesson.\medbreak

\textbf{Elegy:} A poem mourning the death of a loved one.\medbreak

\textbf{Epitaph:} A short poem often written on tombstone
to honor the dead.\medbreak

\textbf{Epistle:} A short poem in the form of a letter.

\poemhead{On my First Son}
\settowidth{\versewidth}{Farewell, thou child of my right hand, and joy}
\begin{verse}[\versewidth]
{\fontverse
Farewell, thou child of my right hand, and joy;\\
My sin was too much hope of thee, lov'd boy. 
}
\end{verse}

Goodbye my child, you were my best thing and the thing that brought me joy.
My sin was that i had high expectation for you. Ben says "right 
hand" to mean best or favorite.

\begin{verse}[\versewidth]
{\fontverse
Seven years tho' wert lent to me, and I thee pay, \\
Exacted by thy fate, on the just day. 
}
\end{verse}

Seven years you were lent to me by fate, and then it took you on the 
exact day you were born (your birthday).

\begin{verse}[\versewidth]
{\fontverse
O, could I lose all father now! For why \\
Will man lament the state he should envy? 
}
\end{verse}

No one will ever call me "father" for you were my only child.
And why should i cry when i really should envy you.

\begin{verse}[\versewidth]
{\fontverse
To have so soon 'scap'd world's and flesh's rage, \\
And if no other misery, yet age?
}
\end{verse}

You have escaped the world too soon, escaped from your
body's trouble, and if you have lived you would suffer from old age.\bigbreak

\begin{verse}[\versewidth]
{\fontverse
Rest in soft peace, and, ask'd, say, "Here doth lie \\
Ben Jonson his best piece of poetry." 
}
\end{verse}

Here Ben says to his child "if they ask you (the angels), say that 
\textit{I (Ben)} lie here (in the grave)". His great pain makes him
feel as if he were dead.

\begin{verse}[\versewidth]
{\fontverse
For whose sake henceforth all his vows be such, \\
As what he loves may never like too much. 
}
\end{verse}


That i promise you my child i will never love anything like you 
again.\medbreak


\subsubhead{\textit{On my First Son} Summary}

In his epigram Ben mourns the death of his favourite and only child, the best thing
that ever happened to him. He says that fate \textit{lent him} his child
only for seven years and then it took it from him. Because of his great
pain he envies the dead. He concludes by promising that he will never love
anything like his child. \hl{This poem is an elegy to Ben's
son}. It is about \hl{honoring the father} and the moral lesson
is that \hl{nothing in life is ours}. Everything is given 
(lent) to us by God, whether it is money, children or health.

\poemhead{Song to Celia}
\settowidth{\versewidth}{Come, my Celia, let us prove}
\begin{verse}[\versewidth]
{\fontverse
Come, my Celia, let us prove\\
While we may, the sports of love;
}
\end{verse}

Come Celia let us experience the joys of love while we can.

\begin{verse}[\versewidth]
{\fontverse
Time will not be ours forever;\\
He at length our good will sever.
}
\end{verse}

Time will not be ours forever and will separate us from 
our health.  \textbf{personification} in \textit{"He at length"}
given time human attribute.

\begin{verse}[\versewidth]
{\fontverse
Spend not then his gifts in vain.\\
Suns that set may rise again;
}
\end{verse}

Don't waste the gift of time. The day that is over 
may be followed by a new day.

\begin{verse}[\versewidth]
{\fontverse
But if once we lose this light,\\
'Tis with us perpetual night.
}
\end{verse}

But if we lose the light of the sun (life), then 
we will face the night (death). The poet tries to warn 
his beloved about the importance of time.

\begin{verse}[\versewidth]
{\fontverse
Why should we defer our joys?\\
Fame and rumor are but toys.
}
\end{verse}

Why should we delay joys? (this is a \textbf{rhetorical question}
\footnote{\, a question that doesn't need an answer and asked to create dramatic effect.}).
Reputation is trivial. 

\begin{verse}[\versewidth]
{\fontverse
Cannot we delude the eyes\\
Of a few poor household spies,\\
Or his easier ears beguile,\\
So removed by our wile?
}
\end{verse}

Cannot we deceive the people around us? (a \textbf{rhetorical question}).
The poet tries to encourage his beloved to meet in secret.

\begin{verse}[\versewidth]
{\fontverse
'Tis no sin love's fruit to steal;\\
But the sweet theft to reveal.\\
To be taken, to be seen,\\
These have crimes accounted been.
}
\end{verse}


Its no sin to fall in love and act on it if it was in 
private. It is only a crime if we are seen.


\subsubhead{\textit{Song to Celia} Summary}

In this poem Ben invite his beloved to enjoy love while 
they have the time, it is a \hl{carpe diem} poem (seize the day).
He tries to convince her of the shortness of time and youth and how
they should seize it in private while they can. This poem is from the play Volpone.


\subhead{Robert Herrick}
A Royalist and Cavalier poet, part of the school of Ben Jonson,
he is the \underLine{best representative of Ben's school}.

\poemhead{Delight in Disorder}
\settowidth{\versewidth}{A sweet disorder in the dress. }
%i have some space here just to fix the width of the verse
\begin{verse}[\versewidth]
{\fontverse
A sweet disorder in the dress\\
Kindles in clothes a wantonness;\\
A lawn about the shoulders thrown\\
Into a fine distraction;\\
An erring lace, which here and there\\
Enthrals the crimson stomacher;\\
A cuff neglectful, and thereby\\
Ribands to flow confusedly;\\
A winning wave, (deserving note),\\
In the tempestuous petticoat;\\
A careless shoe-string, in whose tie\\
I see a wild civility:\\
Do more bewitch me, than when art\\
Is too precise in every part.
}
\end{verse}

\subsubhead{\textit{Delight in Disorder} Summary}

In this poem Robert tells his beloved that there is
some joy in disorder. He tells her that she doesn't 
need to look perfect, that imperfection is charming and natural.
He expresses this in \underLine{subject/predicate} manner where
the first 12 lines is subject and the last two lines is the 
predicate that describe the subject.

\poemhead{To the Virgins, to Make Much of Time}
\settowidth{\versewidth}{Gather ye rose-buds while ye may}
\begin{verse}[\versewidth]
{\fontverse
Gather ye rose-buds while ye may,\\
Old Time is still a-flying;\\
And this same flower that smiles today\\
Tomorrow will be dying.
} 
\end{verse}
seize your youth days while you can, time passes quickly and the
beautiful flower you see today will die tomorrow. \textbf{Metaphor} in \textit{"rose-buds"}
to mean youth. \textbf{Personification} in \textit{"old time is still a-flying"}.

\begin{verse}[\versewidth]
{\fontverse
The glorious lamp of heaven, the sun,\\
The higher he’s a-getting,\\
The sooner will his race be run,\\
And nearer he’s to setting.
} 
\end{verse}

The higher the sun gets, the nearer our death would be. Here the poet 
uses metaphor to compare our life to one day; the sunrise 
is when we are born, the noon is our youth, the afternoon is middle age and
the sunset is death. 

\begin{verse}[\versewidth]
{\fontverse
That age is best which is the first,\\
When youth and blood are warmer;\\
But being spent, the worse, and worst\\
Times still succeed the former. 
} 
\end{verse}

The best age to live is youth time, you are passionate and energetic
but when it is spent it will become worse and worse (when you get older)
and eventually time will surpass your youth. \textbf{Personification} in \textit{"time still succeed"}.

\begin{verse}[\versewidth]
{\fontverse
Then be not coy, but use your time,\\
And while ye may, go marry;\\
For having lost but once your prime,\\
You may forever tarry.
} 
\end{verse}

Then do not be shy, marry while you can because if
your prime years past, you may end up waiting forever.


\subsubhead{\textit{To the Virgins, to Make Much of Time} Summary}

In this poem Robert Herrick tries to encourage women to marry while they can
because their prime (youth) will not last forever.
Time is quick and when our youth passes we become worse because of old age. 
Robert uses a metaphor to compare between our life to one day;
it start by sunrise (being born) and end by sunset (death).
\hl{The poem is about carpe diem philosophy}.

\newpage
\poemhead{To Daffodils}
\settowidth{\versewidth}{Fair Daffodils, we weep to see}
\begin{verse}[\versewidth]
{\fontverse
Fair Daffodils, we weep to see\\
You haste away so soon;\\
As yet the early-rising sun\\
Has not attain'd his noon.
} 
\end{verse}

Beautiful daffodil\footnote{\, a type of flower that blooms in the spring time.} we 
are sad to see you die so soon, even before the arrival of noon. \textbf{Apostrophe}
is used to addresses \textit{"daffodil"}.

\begin{verse}[\versewidth]
{\fontverse
Stay, stay,\\
Until the hasting day\\
Has run\\
But to the even-song;\\
And, having pray'd together, we\\
Will go with you along.
} 
\end{verse}

Stay daffodil until the day that fades so quickly reaches noon. Wait until the evening 
prayer so we can pray together and then we can go (die). 

\begin{verse}[\versewidth]
{\fontverse
We have short time to stay, as you,\\
We have as short a spring;\\
As quick a growth to meet decay,\\
As you, or anything.\\
We die\\
As your hours do, and dry\\
Away,\\
Like to the summer's rain;\\
Or as the pearls of morning's dew,\\
Ne'er to be found again.
} 
\end{verse}

Here the poet uses simile\footnote{\, a figure of speech that compares between 
two things directly using "as" and "like".} to compare human life to:

\begin{itemize}
  \item \textbf{Daffodil}, that our time is as short as daffodil's.
  \item \textbf{Spring}, that our time is similar to spring; fleeing.
  \item \textbf{Summer rain}, that our time is as quick as the rain in summer.
  \item \textbf{Pearls of morning dew}, that our time is valuable.
\end{itemize}

The poet says that we have short time in this life and that we will eventually die. 
He compares it to daffodil, spring, summer rain and pearls morning's dew. 

\subsubhead{\textit{To Daffodils} Summary}
In this poem Robert addresses daffodil using \hl{Apostrophe}. He compares
our life to the daffodil's; short and fleeting, precious and beautiful.
He says that we have short time in this life, and we are heading for our quick
death that feels like hours. He uses simile repeatedly to emphasize his point.
The poem is a \underLine{contemplation of life} using \hl{carpe diem} philosophy, its message is to
\underLine{cherish the time we got}.



\subhead{Edmund Waller}
Edmund was part of the parliament and was against the king, he
wrote a poem in praise of Oliver Cromwell. However he changed sides
during the civil war and became part of the royalist (Cavalier poets).

\poemhead{Song: Go, Lovely Rose}
\settowidth{\versewidth}{When I resemble her to thee}
\begin{verse}[\versewidth]
{\fontverse
Go, lovely rose!\\
Tell her that wastes her time and me,\\
That now she knows,\\
When I resemble her to thee,\\
How sweet and fair she seems to be.
} 
\end{verse}

The poet sends the rose as a messenger to tell his beloved; to stop wasting his and her time
and that she is as beautiful as the rose.
\textbf{Apostrophe} is used to address the rose in \textit{"Go, lovely rose"}.
\textbf{Metaphor} in \textit{"resemble her to thee"} comparing a rose to a woman.

\begin{verse}[\versewidth]
{\fontverse
Tell her that’s young,\\
And shuns to have her graces spied,\\
That hadst thou sprung\\
In deserts, where no men abide,\\
Thou must have uncommended died.
} 
\end{verse}

The poet asks the rose to tell his beloved; that if she hides her beauty
like a rose in a dessert she will die unnoticed and without anyone praising her beauty.

\begin{verse}[\versewidth]
{\fontverse
Small is the worth\\
Of beauty from the light retired;\\
Bid her come forth,\\
Suffer herself to be desired,\\
And not blush so to be admired.
} 
\end{verse}

If you have beauty and its not seen, then it has no worth. The poet
tells the rose to invite his beloved to go out and show her beauty, she 
should not be shy to be desired and admired.\footnote{because this poem was 
written at the restoration time, it is a vulgar request for girls to show
their beauty.}

\begin{verse}[\versewidth]
{\fontverse
Then die! that she\\
The common fate of all things rare\\
May read in thee;\\
How small a part of time they share\\
That are so wondrous sweet and fair!
} 
\end{verse}

The poet tell the rose to die so that his beloved may learn that 
beauty only last for a brief time and is distend to fade.

\subsubhead{\textit{Go, Lovely Rose} Summary}

Edmund uses \hl{Apostrophe} to address the rose as 
his messenger. He tell the rose to encourage his beloved and invite her to show her beauty and to not be
shy to be admired, because beauty that is not seen has little worth. He tell the rose to die so 
that his beloved may learn that all this is the destiny of all beautiful things. This poem 
is \hl{carpe diem}.

\subhead{John Suckling}

One of the leaders of the Cavalier poets and part of the school
of Ben Jonson.

\poemhead{Song: Why so Pale?}
\settowidth{\versewidth}{Why so pale and wan fond lover?}
\begin{verse}[\versewidth]
{\fontverse
Why so pale and wan fond lover?\\
\vin Prithee why so pale?\\
Will, when looking well can't move her,\\
\vin Looking ill prevail?\\
\vin Prithee why so pale?
} 
\end{verse}

Why you are so pale foolish lover? If looking good 
did not move her, you think looking ill will? Here
the poet asks a \textbf{rhetorical} question that 
serve as a challenge to the lover action.

\begin{verse}[\versewidth]
{\fontverse
Why so dull and mute young sinner?\\
\vin Prithee why so mute?\\
Will, when speaking well can’t win her,\\
\vin Saying nothing do't?\\
\vin Prithee why so mute?
}
\end{verse}

Why so dull and mute? When you spoke well you did not
move her saying nothing does? \textit{"young sinner"} refers
to the idea of original sin\footnote{In Christian we are born 
sinners because of the first sin (Adam and Eve eating from the tree of knowledge).}

\begin{verse}[\versewidth]
{\fontverse
Quit, quit for shame, this will not move,\\
\vin This cannot take her;\\
If of herself she will not love,\\
\vin Nothing can make her;\\
\vin The devil take her.
} 
\end{verse}

Stop and have some dignity this will not work, this cannot move her.
If she did not love on her own then nothing will force her. Forget about
her "let her go to hell".

\subsubhead{\textit{Why so Pale} Summary}

In this poem the poet asks a \hl{rhetorical question} to challenge
the lover actions and make him think and reflect on them. He 
tells him that if she does not love you on her own then you are wasting
your time, just quit. This poem is taken from a play and is about 
\hl{unrequited love}

\subhead{Richard Lovelace}

One of the Cavalier poets and close friend to John Suckling and part
of the school of Ben Jonson. He was accustomed to court life and
spent some time in prison.

\poemhead{To Althea, from Prison}
\settowidth{\versewidth}{When Love with unconfinèd wings}
\begin{verse}[\versewidth]
{\fontverse
When Love with unconfinèd wings\\
Hovers within my Gates,\\
And my divine Althea brings\\
To whisper at the Grates;\\
When I lie tangled in her hair,\\
And fettered to her eye,\\
The Gods that wanton in the Air,\\
Know no such Liberty.
} 
\end{verse}

Even though I'm trapped in my cell, love hovers around me
and Althea comes to visit me. Not even the gods know such
freedom. Here the poet tries to say that even though
he is in prison cell, his imagination is not, his soul
is free. \textbf{Personification} in \textit{"Love with unconfinèd wings"}.

\begin{verse}[\versewidth]
{\fontverse
When flowing Cups run swiftly round\\
With no allaying Thames,\\
Our careless heads with Roses bound,\\
Our hearts with Loyal Flames;\\
When thirsty grief in Wine we steep,\\
When Healths and draughts go free,\\
Fishes that tipple in the Deep\\
Know no such Liberty.
} 
\end{verse}

When our cups of wine is passed around and we are in a careless state, when 
our heart is filed with love to the king, when our grief is drown in wine and
we make a toast for our health \textit{then} even the fishes that drink from the depth
of the ocean does not know such freedom.


\begin{verse}[\versewidth]
{\fontverse
Stone Walls do not a Prison make,\\
Nor Iron bars a Cage;\\
Minds innocent and quiet take\\
That for an Hermitage.\\
If I have freedom in my Love,\\
And in my soul am free,\\
Angels alone that soar above,\\
Enjoy such Liberty.
} 
\end{verse}

A cell in jail do not make a prison, nor the iron in the cell.
A mind that is quite and calm is like a Hermitage (A place where
religious people go in isolation to pray). If I'm free to love
and my soul is free, then only Angels have the same freedom that i have.

\subsubhead{\textit{To Althea, from Prison} Summary}

In this poem the poet tries to distinguish between physical 
and spiritual freedom. He says that even in jail he is still
free; his imagination/spirit can go anywhere. In the first stanza 
he talks about the sky and his love for Althea. In the second stanza
he talks about the sea and his love for the king. In the final
stanza he talks about his cell being like a Hermitage. He says that
not \textit{The gods in the sky} nor \textit{The fishes in the deep sea} 
are as free as he, but only the Angels that fly above.

\head{The School of John Donne}

Was not a real school but rather a group of poets whose poetry
had certain features in common. \underLine{Samuel Johnson}\footnote{\, an eighteenth century critic}
called them \textbf{Metaphysical poets} to indicate that these poets used \underLine{scientific 
images} in their poetry. They lived in a period of scientific, intellectual, political, and religious 
changes. Their poetry can be divided into parts: \underLine{the amatory}\footnote{\, relating to lovers or lovemaking.} 
and \underLine{the religious}, though these two aspects are sometimes found together.

\subhead{Characteristics of Metaphysical Poetry}

\begin{itemize}
  \item Metaphysical conceit.
  \item Scientific imagery.
  \item Political and religious themes.
  \item Fusion of mind and heart.
  \item Intellect and Controlled Sentiment\footnote{\, avoiding exaggeration when expressing emotions.}.
  \item Wit\footnote{\, to say something clever and funny.} and humor.
  \item Epigrammatic conciseness\footnote{\, short, clever and memorable statement.}.
\end{itemize}

\subsubhead{Metaphysical conceit}

Or simply \textbf{conceit}, is an extended metaphor that compares between two
seemingly unrelated things to create a surprising resemblance, often to explore 
abstract ideas (e.g., love, faith, death) through concrete imagery.
This unconventional comparison is used to startle and intellectually challenge the reader.
The \hl{most famous example is John Donne's comparison of
two lovers to the legs of a mathematical compass} in \textit{"A Valediction: Forbidding Mourning"}.

\subhead{John Donne}

Born in London in 1572, he was the most outstanding metaphysical poet. His poetry can be divided into: 
the \underLine{secular}\footnote{\, not related to religious or spiritual subjects.}
poems and the \underLine{religious} poems. The majority of his poems deal with 
death, disloyalty or unrequited love. One of the features of his poetry is the \underLine{consistent 
use of the first person}. Most of his poems often center upon himself.


\poemhead{A Valediction: Forbidding Mourning}
\settowidth{\versewidth}{As virtuous men pass mildly away, \vin \vin}
\begin{verse}[\versewidth]
{\fontverse
As virtuous men pass mildly away,\\
\vin And whisper to their souls to go,\\
Whilst some of their sad friends do say\\
\vin The breath goes now, and some say, No:
} 
\end{verse}

When virtuous people die, they pass away gently, their souls departing without struggle.
Their friends, grieving but confused, can’t even agree on the exact moment of death.
The poet compares virtuous people to the departure of lovers, saying that they should 
do it quietly and peacefully. 

\begin{verse}[\versewidth]
{\fontverse
So let us melt, and make no noise,\\
\vin No tear-floods, nor sigh-tempests move;\\
'Twere profanation of our joys\\
\vin To tell the laity our love.
} 
\end{verse}

So do not cry loudly and make a scene, it is profanation to make 
grief public and let laity\footnote{\, ordinary people (outsiders) that do not know us.}
(people) know. In these lines the poet tell his beloved that its 
a profanation to make our separation public and known to outsiders.

\begin{verse}[\versewidth]
{\fontverse
Moving of th' earth brings harms and fears,\\
\vin Men reckon what it did, and meant;\\
But trepidation of the spheres,\\
\vin Though greater far, is innocent.
} 
\end{verse}

Earthquacks brings fears and damage, people know how much 
damage it caused and the meaning of it (e.g, its a punishment from God).
But the plants and stars when crashed causes more damage yet no one notice them.
\textbf{Metaphysical conceit} is used to compare earthquack and trepidation of the spheres
to lovers separation. In physical love separation is like an earthquack, it causes harm and fear.
While in spiritual love separation is like planets and stars crashing, even though their damage is 
great, it goes unnoticed by us.

\begin{verse}[\versewidth]
{\fontverse
Dull sublunary lovers' love\\
\vin (Whose soul is sense) cannot admit\\
Absence, because it doth remove\\
\vin Those things which elemented it.
} 
\end{verse}

Earthly love rooted entirely in physical senses (touch, sight) 
cannot endure separation because absence removes these sensory elements.

\begin{verse}[\versewidth]
{\fontverse
But we by a love so much refined,\\
\vin That our selves know not what it is,\\
Inter-assured of the mind,\\
\vin Care less, eyes, lips, and hands to miss. 
} 
\end{verse}

But we have a love so refined and transcendent that we don't even know how to explain it.
It is mutual and rooted in the mind (spirit), that the absence of 
physical sense is insignificant. \textbf{Paradox}\footnote{
\, a literary device that contradicts itself but contains some truth.} 
is used in \textit{"So much refined..know not what it is"}
The poet says that the lovers love is so sublime yet they do not know what it is.

\begin{verse}[\versewidth]
{\fontverse
Our two souls therefore, which are one,\\
\vin Though I must go, endure not yet\\
A breach, but an expansion,\\
\vin Like gold to airy thinness beat. 
} 
\end{verse}

Our souls are bound as one. Though I must leave, our physical separation won’t break us.
Instead it will stretch and refine our connection, just like the gold that grows thinner and more 
expansive when hammered, yet remains unbroken. \textbf{Metaphysical conceit} is used to compare 
gold to the lovers bond in \textit{"like gold to airy thinness"}. \textbf{Paradox}
in \textit{"Our two souls therefore, which are one"} describes two souls as one.

\begin{verse}[\versewidth]
{\fontverse
If they be two, they are two so\\
\vin As stiff twin compasses are two;\\
Thy soul, the fixed foot, makes no show\\
\vin To move, but doth, if the other do. 
} 
\end{verse}

Though we are two individuals, we are like the legs of a compass;
your soul is the fixed foot and the center of my life while
I am the foot that moves. \textbf{Metaphysical conceit} is used 
to compare the legs of a compass to two lovers.

\begin{verse}[\versewidth]
{\fontverse
And though it in the center sit,\\
\vin Yet when the other far doth roam,\\
It leans and hearkens after it,\\
\vin And grows erect, as that comes home. 
} 
\end{verse}

Though one leg is in the center, when the other moves away the 
centered leg lean with it and when it comes close it 
becomes straight and adjust to its movements. 

\begin{verse}[\versewidth]
{\fontverse
Such wilt thou be to me, who must,\\
\vin Like th' other foot, obliquely run;\\
Thy firmness makes my circle just,\\
\vin And makes me end where I begun.
} 
\end{verse}

You, my beloved, are the fixed foot of the compass anchored at the center of my world.
While i keep moving to form a perfect circle by the help of your firmness, then i return
to you (this symbolizes the eternity of love).

\subsubhead{\textit{A Valediction: Forbidding Mourning} Summary}

In this poem John uses \underLine{metaphysical conceit} to compare (using scientific images)
two types of love: spiritual love and physical love. He tells his beloved that physical 
love is temporary and is entirely depended on sense. That eventually it will end 
(by separation of the bodies). However his love to
his beloved is transcendent and not effected by physical separation, it 
is eternal. In each stanza he describes spiritual lovers and their separation:

\begin{itemize}
  \item \textbf{First stanza}: Their separation is quite and peaceful. Comparing it to virtuous men leaving the world.
  \item \textbf {Second stanza}: Their separation is private and without much noise.
  \item \textbf {Third stanza}: Their separation is innocent and harmless. Comparing it to the trepidation of the spheres using
    \textbf{metaphysical conceit}.
  \item \textbf {Fourth stanza}: Their love is transcendent and not depended on sense (touch, sight). 
  \item \textbf {Fifth stanza}: Their love is refined. Using a \textbf{paradox} in 
    \textit{"So much refined..know not what it is"}.
  \item \textbf {Sixth stanza}: Separation strengthen their love and it cannot be broken. Comparing their love 
    to gold using \textbf{metaphysical conceit} (\textit{"Like gold to airy thinness beat"}). 
    When gold is hammered it expand and remains unbroken.
  \item \textbf {Seventh, Eighth, Ninth stanza}: They are connected and depended on each other. 
    She is the center of his life. And their love is eternal.
    Using \textbf{metaphysical conceit} to compare themselves to two legs of a compass.
\end{itemize}

\poemhead{Death, be not proud}
\settowidth{\versewidth}{Death, be not proud, though some have called thee}
\begin{verse}[\versewidth]
{\fontverse
Death, be not proud, though some have called thee\\
Mighty and dreadful, for thou art not so;\\
For those whom thou think'st thou dost overthrow\\
Die not, poor Death, nor yet canst thou kill me.
} 
\end{verse}

The poet uses \hl{\textbf{apostrophe}\footnote{\, a figure of speech that addresses something or someone that cannot respond.}
to addresses death}, saying it is not scary nor powerful. He tells death that when you kill someone, his body 
dies, but his soul lives forever. The poet uses a \textbf{Paradox} in \textit{"canst thou kill me"} to say that 
death cannot kill him and to emphasize the soul' immorality.

\begin{verse}[\versewidth]
{\fontverse
From rest and sleep, which but thy pictures be,\\
Much pleasure; then from thee much more must flow,\\
And soonest our best men with thee do go,\\
Rest of their bones, and soul's delivery.
} 
\end{verse}

The poet uses \textbf{metaphor} to compare death to rest and sleep. He 
says if rest and sleep bring pleasure, death must bring more
pleasure. That righteous and good people die early, and death is nothing but the 
release of soul from body.

\begin{verse}[\versewidth]
{\fontverse
Thou art slave to fate, chance, kings, and desperate men,\\
And dost with poison, war, and sickness dwell,\\
And poppy or charms can make us sleep as well\\
And better than thy stroke; why swell'st thou then?
} 
\end{verse}

The poet addresses death using \textbf{apostrophe} to say that
it is controlled by fate, chance, kings (who order wars) and desperate men (who commit suicide).
That it is associated with poison, wars and sickness. That even drugs and magic spells make us
sleep and does it better than death. So why you (death) are proud and arrogant?

\begin{verse}[\versewidth]
{\fontverse
One short sleep past, we wake eternally\\
And death shall be no more; Death, thou shalt die.
} 
\end{verse}

The poet uses a \textbf{metaphor} to compare death to a short sleep.
He says that death is no more than a short sleep between our life 
and the after life. \hl{\textbf{Paradox} is used in \textit{"death, thou shalt die"}.
That in the after life there is no death, we will live eternally}. 

\subsubhead{\textit{Death, be not proud} Summary}

In this poem John addresses death using \hl{apostrophe} to tell it that
it is not great nor powerful. Death depend on external factors like chance, 
fate, diseases, and war. John compares death to sleep and rest saying it 
is more pleasurable. He also compares it to a short sleep between 
our life and the after life. And that it will no longer exist in the after life.
This is a \underLine{religious poem} (also called \textit{"Sonnet X"}), it is part of Donne's 
\underLine{Holy Sonnets}\footnote{\, also called Divine Meditations, 
are a series of 19 poems by John Donne.}.


\subhead{George Herbert}

Was a metaphysical poet born in 1593 into an aristocratic family. Educated at
\underLine{Trinity College, Cambridge} he become 
\underLine{public orator} there. In 1624 he become a \underLine{member of the parliament},
but soon lost the favor of the king. He got interested in religion and became a \underLine{priest}.
His volume of collected poems \textbf{The Temple} indicates how important religion was to him.

\poemhead{The Pulley}
\settowidth{\versewidth}{Having a glass of blessings standing by "let"}
\begin{verse}[\versewidth]
{\fontverse
When God at first made man,\\
Having a glass of blessings standing by,\\
“Let us,” said he, “pour on him all we can.\\
Let the world’s riches, which dispersèd lie,\\
Contract into a span.”
} 
\end{verse}

When God first made humans, He was holding a glass of blessings that
He poured to create humans with. All the riches of the world (strength, beauty, pleasure, etc) were gathered in 
one entity; human life. \textbf{Metaphor} is used in \textit{"glass of blessings"} to represent God grace.

\begin{verse}[\versewidth]
{\fontverse
So strength first made a way;\\
Then beauty flowed, then wisdom, honour, pleasure.\\
When almost all was out, God made a stay,\\
Perceiving that, alone of all his treasure,\\
Rest in the bottom lay.
} 
\end{verse}

God first gave us strength, then beauty, wisdom, honour and pleasure.
When the glass was almost poured fully, God made a pause, withholding 
fulfilment in the bottom of the glass. \textbf{Personification}\footnote{\,
describing a non-human entity with human attribute.}
used in \textit{"strength first made a way"} giving strength human attributes
(making a way). \textbf{Metaphor} used in \textit{"treasure"} to compare
a treasure to God's blessings.

\begin{verse}[\versewidth]
{\fontverse
“For if I should,” said he,\\
“Bestow this jewel also on my creature,\\
He would adore my gifts instead of me,\\
And rest in Nature, not the God of Nature;\\
So both should losers be.
} 
\end{verse}

God says that if He gave his creation fulfilment then they would 
forget all about Him. Adoring the gifts and forgetting 
about the one who gave it to them. Then both God and humans would
lose; God by not having people to worship and glorify Him. Humans by
losing eternal fulfilment and knowing their creator.  \textbf{Metaphor}
used in \textit{"Bestow this jewel"} to refer to fulfilment.

\begin{verse}[\versewidth]
{\fontverse
“Yet let him keep the rest,\\
But keep them with repining restlessness;\\
Let him be rich and weary, that at least,\\
If goodness lead him not, yet weariness\\
May toss him to my breast.”
} 
\end{verse}

Let humans keep earthly blessings (strength, beauty, wisdom) to
themselves, but keep them with constant dissatisfaction. Let them 
be rich but weary. So that if moral goodness wouldn't lead them
to me, weariness would. \textbf{Metaphor} used in \textit{"weary"}
comparing physical tiredness to spiritual exhaustion. \textbf{Personification}
used in \textit{"weariness May toss him"} giving weariness human attributes \textit{"toss"}.

\subsubhead{\textit{The Pulley} Summary}

In the poem \textit{The Pulley} God blesses people with all worldly riches
(e.g, strength, wisdom, beauty, etc) expect \underLine{spiritual fulfilment}. Withholding
it so that people would return to him. The poem shows that earthly blessings will not
bring spiritual fulfilment, and that this need was made in us by God so that we would not forget
about him. George uses \textbf{metaphysical conceit} in the poem title to compare a pulley\footnote{
\, a machine that used to lift heavy objects.
\href{https://i.pinimg.com/736x/08/3b/ad/083badbb5a73ae0b9ae40093063f61e1.jpg}{\textcolor{blue}{See image}} }
to God's plan to lift humans to Him. Human restlessness (dissatisfaction) 
act as a weight which God uses to draw people to Him.
The poem is part of George volume \underLine{The Temple}.

\subhead{Henry Vaughan}

Was a metaphysical poet and a \underLine{medical physician} born in 1622.
He was influenced by Donne but his \underLine{major influence 
came from George Herbert}. His best poetry appeared in (1650-1655) in his
religious volume (\textit{Silex Scintillans}).


\poemhead{The Retreat}
\settowidth{\versewidth}{Happy those early days! when I}
\begin{verse}[\versewidth]
{\fontverse
Happy those early days! when I\\
Shined in my angel infancy.
} 
\end{verse}

Happy were the days of my childhood when i was pure like an angel.
\textbf{Metaphor} used in \textit{"angel infancy"} to compare divine purity to childhood.
The poet expresses a nostalgia\footnote{\, a longing for a period in the past.}
for his childhood.

\begin{verse}[\versewidth]
{\fontverse
Before I understood this place\\
Appointed for my second race, 
} 
\end{verse}

Before i understand that this earth is a temporary place which 
i came to from another life. 

\begin{verse}[\versewidth]
{\fontverse
Or taught my soul to fancy aught\\
But a white, celestial thought;
} 
\end{verse}

Or before my soul desired anything expect pure and divine thoughts.
The poet says that in childhood his desires were innocent and pure.
\textbf{Metaphor} used in \textit{"celestial thought"} to mean
transcendent thinking. 

\begin{verse}[\versewidth]
{\fontverse
When yet I had not walked above\\
A mile or two from my first love, 
} 
\end{verse}

When i had not walked away from God, when i was pure and innocent.
\textbf{Metaphor} in \textit{"walked above a mile"} to compare
between physical walking to separation from God. Also 
in \textit{"first love"} to refer to God. \textbf{Paradox} used in \textit{"walked above"}
the upward movement is contrasted in the falling away from divine grace (God).


\begin{verse}[\versewidth]
{\fontverse
And looking back, at that short space,\\
Could see a glimpse of His bright face;
} 
\end{verse}

And when i look back at my childhood (the days of innocence), i could
briefly remember being close to God. \textbf{Metaphor} used in \textit{"bright face"} 
to refer to God's presence. 

\begin{verse}[\versewidth]
{\fontverse
When on some gilded cloud or flower\\
My gazing soul would dwell an hour, 
} 
\end{verse}

When i was a child i would look at nature for hours and admire it.
\textbf{Personification} \textit{"soul would dwell"} giving the soul
human attributes. 

\newpage
\begin{verse}[\versewidth]
{\fontverse
and in those weaker glories spy\\
some shadows of eternity;
} 
\end{verse}

Earthly beauties are mere reflection of heaven, they are 
but a tiny thing compared to heaven. \textbf{Metaphor} in 
\textit{"weaker glories"} to refer to nature and earthly beauty. 

\begin{verse}[\versewidth]
{\fontverse
before i taught my tongue to wound\\
my conscience with a sinful sound, 
} 
\end{verse}

Before i learned to use my tongue in a sinful way that hurts
my conscience (cause guilt). The poet remembers a time where he was free of guilt 
from speaking sinfully and hurtfully. \textbf{Personification} in \textit{"tongue to wound"}
giving the tongue a human attribute, also in \textit{"wound my conscience"} making 
his conscience as something that can be wounded. 

\begin{verse}[\versewidth]
{\fontverse
Or had the black art to dispense\\
A \underLine{s}everal \underLine{s}in to every \underLine{s}ense, 
} 
\end{verse}

If i had black magic i would assign a sin to each sense (sight, hearing, touch, etc.).
The poet is exaggerating his capacity for evil and assigning every part of the body to a sin.
\textbf{Alliteration} by repeating the 's' sound.

\begin{verse}[\versewidth]
{\fontverse
But felt through all this fleshly dress\\
Bright shoots of everlastingness.
} 
\end{verse}

Despite being in this physical body, i still feel bursts of spiritual divinity.
 \textbf{Metaphor} in \textit{"fleshly dress"} to refer to physical body.
 \textbf{Paradox} in \textit{"shoots of everlastingness"} that eternal divinity is
felt through a mortal body. \textbf{Allusion} to Platonism which asserts
that the physical body is a temporary container of the soul. \textbf{Metaphysical conceit} 
in \textit{"fleshly dress..shoots of everlastingness"} to compare 
Physical body to eternal existence.


\begin{verse}[\versewidth]
{\fontverse
O, how i long to \underLine{t}ravel back,\\
and \underLine{t}read again that ancient \underLine{t}rack!
} 
\end{verse}

How i long to get back to that state (of being pure, innocent and close to God).
To the original path of God. \textbf{Alliteration}\footnote{\, 
the repetition of the same letter in the beginning of words} by repeating the 't' sound.
\textbf{Metaphor} in \textit{"ancient track"} to refer to the path of God.

\begin{verse}[\versewidth]
{\fontverse
That I might once more reach that plain\\
Where first I left my glorious train, 
} 
\end{verse}

That i might once again get back to a simpler and purer time
where i left my true path to God. \textbf{Metaphor} in \textit{"glorious train"} 
to refer to divine grace, also in \textit{"that plain"} to refer to spiritual peacefulness.

\newpage
\begin{verse}[\versewidth]
{\fontverse
From whence th’ enlightened \underLine{s}pirit \underLine{s}ees\\
That \underLine{s}hady city of palm trees.
} 
\end{verse}

From there the enlightened spirits can see a city which is peaceful and heavenly.
The poet suggest that when we enter spiritual clarity, we are no longer blind
and can see a heavenly place. \textbf{Metaphor} in \textit{"Shady city of palm trees"} to refer
to a heavenly place. \textbf{Alliteration} by repeating the 's' sound.

\begin{verse}[\versewidth]
{\fontverse
But, ah! my \underLine{s}oul with too much \underLine{s}tay\\
Is drunk, and \underLine{s}taggers in the way. 
} 
\end{verse}

But sadly my soul is in too much sin. My soul has become confused 
and lost its way. The poet expresses his despair about his moral corruption
and his inability to change it. \textbf{Personification} in \textit{"my soul...Is drunk"} 
giving his soul human attributes. \textbf{Alliteration} by repeating the 's' sound. 
\textbf{Metaphysical conceit} in \textit{"drunk, and staggers in the way"} 
comparing Spiritual decline to drunkenness.



\begin{verse}[\versewidth]
{\fontverse
Some men a forward motion love;\\
But I by backward steps would move,
} 
\end{verse}

Some men seek earthly progress. But I seek to go backward (to spiritual purity
and being close to God). \textbf{Metaphor} in \textit{"forward motion"} 
to refer to earthly progress, also in \textit{"backward steps"} to refer 
to spiritual purity. \textbf{Paradox} in 
\textit{"forward motion"} which is considered as regression, leading to moral corruption. Also
in \textit{"backward steps"} which is considered to be spiritual advancement.

\begin{verse}[\versewidth]
{\fontverse
And when this dust falls to the urn,\\
In that state I came, return.
} 
\end{verse}

And when this physical body dies and is buried. I will return to the state 
which i came in. The poet asserts that after death his soul will revert back
to its divine place and eternal from. \textbf{Metaphor} in \textit{"dust"} 
to refer to human body, also in \textit{"urn"} to refer to grave. 

\subsubhead{\textit{The Retreat} Summary}

In this \hl{nostalgic poem} Henry lament a time where he was once pure, innocent and close to God.
Those times were his childhood and the life before. Being older he begun to sin
and walk away from God's path. He longs for divine closeness and wants to relive the blessings he had.
In the last couplet he wishes to die so that  he could return to God and his
pure and innocent state. Throughout the poem Henry uses Platonic and Biblical allusions.

\subhead{Andrew Marvell}

A politician and a metaphysical poet who was born in 1621. He was 
friend to John Milton. He was influenced by both the 
metaphysical school and the school of Ben Jonson.

\newpage
\poemhead{A Dialogue between the Soul and the Body}
\settowidth{\versewidth}{O who shall, from this dungeon, raise }
\centerline{\textbf{\large{Soul}}}
\begin{verse}[\versewidth]
{\fontverse
O who shall, from this dungeon, raise\\
a soul enslav’d so many ways?\\
with bolts of bones, that fetter’d stands\\
in feet, and manacled in hands;\\
here blinded with an eye, and there\\
deaf with the drumming of an ear;
} 
\end{verse}

Oh who shall free me from this painful prison (body)?
Whose bones (ribs) feels like prison bars and hands
and feet feels like chains. Whose eyes blind me 
and ears makes me deaf (symbolizing that sight can 
makes us blind at times and our ears are easily distracted). 
\textbf{Metaphor} in \textit{"bolts of bones"} to refer to prison bars. 
\textbf{Paradox} in \textit{"blinded with an eye"}, that
having eyesight blinds our soul. Also in \textit{"deaf with
drumming of an ear"}, that having ears deafens and distract our
soul. \textbf{Personification} the soul is given a human attribute
(a prisoner). \hl{\textbf{Rhetorical question}\footnote{\, 
a question that doesn't need an answer. It is asked to create dramatic effect.}.
in \textit{"a soul enslav’d so many ways?"}}. 

\begin{verse}[\versewidth]
{\fontverse
A soul hung up, as ’twere, in chains\\
Of nerves, and arteries, and veins;\\
Tortur’d, besides each other part,\\
In a vain head, and double heart\footnote{\, hypocrisy}. 
} 
\end{verse}

A soul that is tortured by the body and its organs and nerves.
Inside a prideful and arrogant head, and a heart that experience
mixed emotions (love/hate). \textbf{Personification} in \textit{"tortur'd"} 
given the soul a human attribute (being tortured), also in \textit{"a soul.. in chains"}.\bigbreak

\centerline{\textbf{\large{Body}}}
\begin{verse}[\versewidth]
{\fontverse
O who shall me deliver whole\\
From bonds of this tyrannic soul?\\
Which, stretch’d upright, impales me so\\
That mine own precipice I go;\\
And warms and moves this needless frame,\\
(A fever could but do the same)
} 
\end{verse}

Who shall set me free from this cruel and oppressive soul? Which compel 
me to do things that i don't like (like being virtuous and moral). It 
heats and warms (a task that a fever can do). The body mocks the soul 
for doing the same job a fever do. \textbf{Rhetorical question} in
\textit{"who shall..tyrannic soul?"}. \textbf{Personification} the soul
is given human attributes (being a tyrant). 

\begin{verse}[\versewidth]
{\fontverse
And, wanting where its spite to try,\\
Has made me live to let me die.\\
A body that could never rest,\\
Since this ill spirit it possest. 
} 
\end{verse}

The soul forces me to live, only to torture me and eventually let me die.
It won't let me rest because it is evil and tyrant.\bigbreak

\centerline{\textbf{\large{Soul}}}
\begin{verse}[\versewidth]
{\fontverse
What magic could me thus confine\\
Within another’s grief to pine?\\
Where whatsoever it complain,\\
I feel, that cannot feel, the pain;\\
And all my care itself employs;\\
That to preserve which me destroys;
} 
\end{verse}

What magic traps me like this and make me suffer the body's saddens? That
whatever the body feels I feel it too even though I shouldn't. And all my 
care goes to keep the body alive, even though that destroys me (by neglecting 
my spiritual needs and paying all my attention to the body). \textbf{Rhetorical question} in
\textit{"What magic..to pine?"}. \textbf{Personification} in which the soul is given 
human attributes (the soul feels, pine and care). 


\begin{verse}[\versewidth]
{\fontverse
Constrain’d not only to endure\\
Diseases, but, what’s worse, the cure;\\
And ready oft the port to gain,\\
Am shipwreck’d into health again. 
} 
\end{verse}

Am forced not only to endure the diseases of the body but also the cure.
And when Am ready to die and leave this body, am pulled back by health.
The soul expresses its frustration with the body's sickness and painful cures.
And when there is hope of the soul leaving this body, physical health destroys it. 
\textbf{Paradox} in \textit{"what's worse, the cure"} asserting that the cure is worse
than the disease. Also in \textit{"shipwreck’d into health again"} asserting that health
is a bad thing. \textbf{Metaphor} in \textit{"port to gain"} referring to death.\bigbreak

\centerline{\textbf{\large{Body}}}
\begin{verse}[\versewidth]
{\fontverse
But physic yet could never reach\\
The maladies thou me dost teach;\\
Whom first the cramp of hope does tear,\\
And then the palsy shakes of fear;\\
The pestilence of love does heat,\\
Or hatred’s hidden ulcer eat; 
} 
\end{verse}

But medicine cannot cure the emotional and mental pain that the soul
causes. The cramping hope, the paralyzing fear, the overwhelming love
that feels like a plague and the hatred that eats
the body. \textbf{Paradox} in \textit{"cramp of hope"} hope which is a positive
thing is being described negatively. Also in \textit{"pestilence of love"} 
framing love negatively. \textbf{Personification} in
\textit{"hope does tear"}, \textit{"palsy shakes of fear"}, \textit{"love does heat"} and 
\textit{"hatred..eat"} given emotions a human attribute. 

\newpage
\begin{verse}[\versewidth]
{\fontverse
Joy’s cheerful madness does perplex,\\
Or sorrow’s other madness vex;\\
Which knowledge forces me to know,\\
And memory will not forego.\\
What but a soul could have the wit\\
To build me up for sin so fit? 
} 
\end{verse}

Joy is overwhelming and confusing, sorrow is annoying. I 
(the body is speaking here) cannot forget these emotions because 
of the souls knowledge of them. Who other then the soul that have 
the intellect to make me perfect for sinning. \textbf{Rhetorical question}
in \textit{"What...for sin so fit?"}. 

\begin{verse}[\versewidth]
{\fontverse
So architects do square and hew\\
Green trees that in the forest grew.
} 
\end{verse}

So the architects (the soul) cut and shape the wood from
the green trees that grow in the forest. The Body is saying
that he was once natural like the green trees but the soul
came and changed his natural shape. \textbf{Metaphor} 
in \textit{"architects"} to refer to soul
and \textit{"Green trees"} to refer to body.

\subsubhead{\textit{A Dialogue between the Soul and the Body} Summary}

In this poem the Soul and Body have a debate, each complaining about the 
other. The Soul describes the \underLine{Body as a prison}, filed with diseases and
pains. The Body describes the \underLine{Soul as a tyrant}, filling the Body
with overwhelming emotions . The poem contains allusion to Platonism.
Andrew offers no resolution in the debate, symbolizing the \hl{constant 
struggle between physical desires and spiritual needs}. 



\newpage
\head{Past Exam Sheets \& Answers}\bigbreak

\subhead{First Exam, A}\bigbreak

\subsubhead{Question 1: Fill in the blanks (6 Marks)}

\begin{enumerate}

  \item The Civil War (1642-1649) was between \underLine{Charles I} and
    \underLine{The Parliament}.

  \item The most important schools of poetry in the seventeenth century are
    \underLine{School of Ben Jonson} and \underLine{School of Jon Donne}.

  \item Among the most important features of Ben Jonson's poetry are
    \underLine{logic}, \underLine{wit} and \underLine{clarity}.
  
\end{enumerate}

\subsubhead{Question 2: Select the correct answer (4 Marks)}

\begin{enumerate}

  \item \textbf{The "School of Ben Jonson" is known for emphasizing which of 
    the following in their poetry? } 

  A) Pastoral themes. \\ 
  B) Metaphysical conceits. \\
  \hl{C) Classical form and decorum.} \\
  D) Romantic imagery.
  
 \item \textbf{In Jonson's poem "Come, My Celia, Let Us Prove" What is the 
   main theme of the poem?} 

 \hl{A) The brevity of life and the importance of sizing the moment.}\\ 
  B) The pain of unrequited love. \\
  C) The exploration of nature's beauty. \\
  D) The pursuit of knowledge.
  
\item \textbf{Robert Herrick's "Delight in Disorder" primarily explores the 
  theme of?} 

  A) The virtue of order and control. \\ 
\hl{B) The appeal of disorder and imperfection.} \\
  C)  The pain of lost love.\\
  D) The pursuit of spiritual purity.

  \item \textbf{In Robert Herrick's "To daffodils" the daffodils symbolize?} 

  A) Immortality and eternal life. \\ 
\hl{B)  The fleeting nature of youth and beauty.}\\
  C)  The coming of spring and renewal.\\
  D) The permanence of nature.
  
  
\end{enumerate}

\subsubhead{Question 3: Answer Two of the following (10 Marks)}\bigbreak

\textbf{A) Comment on the most important features of the poetry of the 
  school of Ben Jonson.}\medbreak

The school of Ben Jonson were highly influenced by the classics (Roman and Greek). 
Their poetry were simple and plain, using easy and clear language. They also 
used Logic and wit, didacticism and instruction, realism and 
the use of controlled feelings in their poetry.\bigbreak

\textbf{B) Comment on Ben Jonson's "On my First Son" as an elegy. }\medbreak

In the poem "On my First Son" Ben mourns the death of his son at the age 
of seven. His love for his son were so great that he says he will never 
love anything like him again. The poem is a great example of an elegy because 
it shows the pain and suffering of the poet through the sad tone of the poem.\bigbreak

\textbf{C) The theme of carpe diem is very common in English 17th century poetry.
Discus with reference to one poem of your choice.}\medbreak

Carpe diem is a Latin term which means seize the moment/day. It was a reaction to the Puritans 
rule in the Restoration period. A great example of carpe diem philosophy is Ben Jonson's 
\textit{Song to Celia}. In this poem Ben urges his beloved to enjoy love while they can.
He tries to convince her of the shortness of time and youth
and how they should seize it in private while they can.


\subhead{First Exam, B}\bigbreak

\subsubhead{Question 1: Fill in the blanks (6 Marks)}

\begin{enumerate}

\item \textbf{Carpe diem means:} Seize the day/moment.
  \item \textbf{The cavalier poets are:} Robert Herrick, John Suckling,
     Richard Lovelace and Edmund Waller.
   \item \textbf{A very clear example of elegy can be found in:} Song:
     To my First Son.
  
\end{enumerate}


\subsubhead{Question 2: Select the correct answer (4 Marks)}

\begin{enumerate}

 \item \textbf{Which of the following best describes the style of 
   Robert Herrick's poetry?} 

  A) Metaphysical and intellectual. \\ 
  B) Classical and formal. \\
  \hl{C) Lyric, light and celebratory.} \\
  D) Dark, somber and introspective.
  

 \item \textbf{In "To the Virgins, to Make Much of Time" the poet advises 
   the virgins to:} 

  A) Save their beauty for the right man.  \\ 
  \hl{ B) Marry young to avoid regret.}  \\
  C)  Live a life of purity and chastity. \\
  D) Worship nature and its beauty.

 \item \textbf{Ben Jonson's poetry is known for its:} 

  A) Use of elaborate metaphysical conceit. \\ 
  B)  Pursuit of simple, natural language.\\
\hl{C) Classical and reference and adherence to form.} \\
  D) Emphasis on personal, emotional expression.

 \item \textbf{The phrase "Come, My Celia, Let us prove" in Ben Jonson's 
   poem is an example of which literary technique?} 

   \hl{A) Apostrophe.} \\ 
  B) Hyperbole. \\
  C) Metaphysical conceit. \\
  D) Allusion.
  
\end{enumerate}

\subsubhead{Question 3: Answer Two of the following (10 Marks)}\bigbreak

\textbf{A) Explain the following lines "Seven years thou were lent to me
and i pay thee/ Exacted by thy fate on the just day"}\medbreak

Seven years you (my son) were lent to me by fate, and then it took you on the exact day you were
born (your birthday). In these lines the poet mourns the death of his son. He says that 
the things we own (children, wealth, etc.) is lent to us and someday will be taken 
away from us.\bigbreak


\textbf{B) Describe the poetry of Ben Jonson and his followers.}\medbreak

Ben Jonson's poetry was influenced by the classics (Roman and Greek).
The features of his poetry were:

\begin{itemize}
  \item Clarity, brevity, simplicity and order.
  \item Realism and the use of controlled feelings.
  \item Logic and wit.
  \item Didacticism and instruction.
  \item Refinement of the classics. 
\end{itemize}

\textbf{C) Who are the Roundheads and who are the Royalists?}\medbreak

The Roundheads were the supporters of Parliament. They were primarily common people 
such as merchants and tradesmen. The Royalists were the supporters of the king. They 
were people who lived a courtly life near the king. 


\end{document}
