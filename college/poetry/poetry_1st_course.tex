\documentclass[12pt, a4paper]{article}
\usepackage{polyglossia}
\usepackage{geometry}
\usepackage{lua-ul}
\usepackage{color,soul}
\usepackage{verse}
\usepackage{xcolor}
\usepackage{hyperref}

\geometry{bottom=1in, top=1in}
\setlength\parindent{0pt}
\newcommand{\attrib}[1]{\nopagebreak{\raggedleft\footnotesize #1\par}}

%fonts
\setmainfont{Erode}
\newfontfamily\fontverse{Erode-Bold.otf}
\newfontfamily\fonthead{Cinzel Decorative}
\newfontfamily\fontsub{Bely}

%sections
\newcommand{\head}[1]{
  \phantomsection
  \section*{\centering{\fonthead{#1}}}
  \addcontentsline{toc}{section}{#1}
}


\newcommand{\poemhead}[1]{
  \phantomsection
  \subsection*{\centering{\large{\fontsub{#1}}}}
  \addcontentsline{toc}{subsection}{#1}
}


\newcommand{\subhead}[1]{
  \phantomsection
  \subsection*{\fontsub{#1}}
  \addcontentsline{toc}{subsection}{#1}
}

\newcommand{\subsubhead}[1]{
  \phantomsection
  \subsubsection*{\fontsub{#1}}
  \addcontentsline{toc}{subsubsection}{#1}
}

\newcommand{\df}[1]{
  {\fontsub\textbf{#1}}
}

\begin{document}

\head{General Characteristics of the Sixteenth Century}

\subhead{Major Events}

\begin{enumerate}

  \item \textbf{The fall of Constantinople}

The fall of the city lead many Greek scholars and their manuscripts to settle down
in Italy, because Italy was a great place for learning. The scholars and their teaching 
had a huge impact on Italy and its revival of
classical learning.
  
  \item \textbf{Geographical Discoveries}

The fall of Constantinople led to the loss of the old routes for spices, silk and
precious stones. As a result there were attempts to find new routs, this led to the discovery
of both \underLine{Cape route to India} and \underLine{The New World (America)}.

  \item \textbf{The Invention of Printing}

In the late 15th books were written by hand and they were expensive, only rich
people was able to afford them, \hl{in 1445} this changed. The Invention of printing allowed 
the widespread of knowledge to most people, books were cheap and available without much effort.

  \item \textbf{The Copernican system}

Copernicus work changed the idea held; that the earth is the center of the universe
and that the sun and other planets revolve around it. Copernicus instead advocated 
that the earth and other planets revolve around the sun.

  \item \textbf{The Reformation}

By the end of the Middle Ages many thought the catholic church needed reformation
because of the growing wealth of the clergy and the moral shortcomings to some of them.
The Reformation of the church included: 


\begin{itemize}
\item Humanism and the Renaissance.
\item The invention of printing.
\item The reaction of princes against the authority of the pope
\end{itemize}

\end{enumerate}

\subhead{The Renaissance}

It is the most significant movement of the 16th century, unlike the middle ages, it
was conscious of itself, it knew that it was the Renaissance.
It was marked by the growing interest of the part of scholars in the language and
literature of the classical worlds of Greece and Roman. \hl{Italy was the center
of the Renaissance} because of:

\begin{itemize}

\item Its geographical position close to Greece Egypt and Arabic empire.
\item Its traditional Roman law and government.
\item Its material prosperity and peace.
  
\end{itemize}

\subsubhead{Characteristics of The Renaissance}

\begin{enumerate}

  \item \textbf{Humanism} 

The believe that Humans is the most important being. It focuses on the
human element rather than the supernatural or divine. It relies
on science rather than superstition.

  \item \textbf{Secularism} 

The principle of separation between the state and religion. This was a major 
thing in the sixteenth century, before that (in the Middle Ages) the church
was in control of everything, from state to art and literature.

  \item \textbf{Naturalism} 

The believe that everything happens in the world is natural and can be tested
with science. This helped shift the emphasis from the supernatural to the here and now,
it helped people to seek means to achieve their goals instead of relaying on god
and superstition.
    
  \item \textbf{The Revival of Platonism} 

The belief in man’s ability to increase through wisdom and virtue his power and
knowledge with which to control the non-human element (the spirit).

  \item \textbf{Classic Translations and Standards}

The renaissance offered many great translations of the classics (Greek and Latin),
it also promoted classical standards to literature.
  
\end{enumerate}


\head{Sixteenth Century Poetry}

In the 16th century lyrics in its all kind were popular especially songs, madrigals and
\hl{lyrical poems}. The major poets at that time were: \underLine{Sir Thomas Wyatt.},
\underLine{Henry Howard (Earl of Surrey)}, \underLine{Philip Sidney}, \underLine{William Shakespeare},
\underLine{Edmund Spenser} and \underLine{Christopher Marlowe}.\medbreak

\textbf{Lyrical Poetry:} Type of poetry that expresses personal feelings in a rhyming and short way.
It is \hl{personal, short and musical}.\medbreak

\textbf{Iambic  Pentameter:} A line of a poem composed of 10 syllables (\underLine{five iambic feet}).\medbreak
\textbf{Iambic Hexameter (\underLine{Alexandrine}):} A line of a poem composed of 12 syllables
(\underLine{six iambic feet}).\medbreak

\subhead{Geoffrey Chaucer}

One of the greatest poets in the Middle Ages. Was called the father of English poetry.
He influenced later poets especially English poets in the 16th century. His influence 
appeared in:

\begin{itemize}

\item His employment of rhyme and regular metre.
\item His huge contribution to middle English.
\item His impact on native literature.
  
\end{itemize}


\subhead{Sonnet}

\hl{The sonnet is 14 line poem written in iambic pentameter}. It was the
most important form of poetry in the 16th century. The sonnet originated in Italy in the 13th century by
\underLine{Petrarch} and \underLine{Dante}. The
English sonnet was credited to \underLine{Sir Thomas Wyatt} and \underLine{Henry Howard} in the early
16th century. Sonnets were mostly about love, Petrarch wrote sonnets to his mistress \underLine{Laura}.

\subsubhead{Petrarch Sonnet}

Or Italian sonnet. Were composed of 14 lines of \underLine{octave and sestet} rhyming
\underLine{abba,abba}. The climax was in the octave and sometimes the octave 
were divided into two quatrain. The structure would look like this:

\begin{itemize}
  
\item \textbf{Octave:} The first 8 lines of the sonnet and it is sometimes divided into 2
  \textbf{Quatrain} (4 lines).
\item \textbf{Sestet:} The last 6 lines of the sonnet, it is sometimes divided into 2 \textbf{tercets}
(3 lines).

\end{itemize}

\subsubhead{English Sonnet}

Henry Howard was one of the first to modify the Italian Scheme of the sonnet,
but Shakespeare was the best to apply it. This is why the  English Sonnet was called
\underLine{The Shakespearean Sonnet}. English sonnet consisted of \underLine{3 quatrains and couplet}
rhyming \underLine{abab,cdcd} and the climax or the solution were in the couplet.

\subhead{Themes of The Sixteenth Century}\bigbreak

\begin{enumerate}

  \item \textbf{Courtly Love}

Love was treated as a kind of god to be worshiped, poets idealized their lovers and
would indulge themselves in \underLine{unrequited love}\footnote{\, one-sided love.}.
These poets would love unattainable females which were of high class and sometimes married. 
Most of the poets in the sixteenth century wrote about courtly love. 

  \item \textbf{Time}

Time was an important theme, it manifested in the idea of youth and the saying
“carpe florem\footnote{\, seize the flower. }", it represent the poet desire to for youth
and the beauty that comes from it.

  \item \textbf{Death}

Closely connected to time, is the theme of immortality and the fear of death,
there were two reaction from poets about it: It drove some poets to live in the 
moment following the motto \underLine{“carpe
diem\footnote{\, seize the day/moment.}"}, other took reckless risk and attacked the wicked.

  \item \textbf{Dreams and Sleep}
  
\end{enumerate}


\subhead{Poetic Devices of The Sixteenth Century}\bigbreak


\begin{itemize}

\item \textbf{Metaphor:} Is a figure of speech that compare between two things that isn’t
literally true in order to make a resemblance.

\textbf{Ex:} She is flower.

\item \textbf{Metonymy: } Is a figure of speech where an object name is replaced by an-
other closely connected to it.

\textbf{Ex:} What is your favourite \textit{dish}?

Here dish means food.

\item \textbf{Synecdoche:} A figure of speech that uses a part of something to refer to the
whole.

\textbf{Ex:} We Need more \textit{hands} to finish the job.

Here hands refers to men.

\item  \textbf{Apostrophe:} A figure of speech that is used to address a non human entity
(object) or someone that cannot reply.

In Sonnet 1 Edmund uses Apostrophe to describe the pages and rhymes.

\item  \textbf{Oxymoron:} A figure of speech that combines contradictory words with
opposing meanings.

\textbf{Ex:} Bitter sweet.

\item  \textbf{Alliteration:} The repetition of the same letter in the beginning of words in
a single line.

\textbf{Ex:} “Yet \underLine{m}ay I, by no \underLine{m}eans, \underLine{m}y wearied \underLine{m}ind”.

\item  \textbf{Consonance: } The repetition of the same letter in the mid or end of words
in a single line.

\textbf{Ex:} “Who lis\underLine{t} her hun\underLine{t}, I put him ou\underLine{t} of doub\underLine{t}”.

\end{itemize}


\subhead{Images of The Sixteenth Century}\bigbreak

\begin{enumerate}

  \item \textbf{Images of the sea}

The sea was the most important image of the 16th century because of the discovery
that were made through the sea especially the discovery of the new world
(America). The most popular poems that used images of the sea were:

\begin{itemize}
  
  \item The Galley By Thomas Wyatt.
  \item Sonnet 34 By Edmund Spenser.
  
\end{itemize}

 

  \item \textbf{Stars}

Stars was an important image of the 16th century, people back then depended
on stars to guide them through their joineries through the wide open seas.

  \item \textbf{Life as a stage}

Stage was a very important image in the 16th century, poets would compare our
life to a play, happiness and sadness to comedy and tragedy, it was widely used
because of the popularity of plays and how similar life can be to a play.
The most popular poems that used the images of stage were:


\begin{itemize}
  
  \item Sonnet 54 By Edmund Spenser.
  \item What is our Life By Walter Raleigh.
 
\end{itemize}

  \item \textbf{War}.
  \item \textbf{Imprisonment}.
  \item \textbf{Diseases}.
  \item \textbf{Nature}.

\end{enumerate}

\head{Sir Thomas Wyatt}

Thomas was responsible of introducing the Italian sonnet to English poetry, his
influence was major especially in his use of two forms of poetry:

\begin{itemize}
  \item \textbf{ottava rima:} A rhyming stanza of eight lines.
  \item  \textbf{terza rima:} A rhyming stanza of three lines.
  
\end{itemize}

\textbf{Stanza:} A group of rhyming lines separated from others in a poem.


\poemhead{To His Lady}
\settowidth{\versewidth}{Madam, withouten many words}
\begin{verse}[\versewidth]
{\fontverse
Madam, withouten many words \\
\vin Once I am sure ye will or no ...\\
And if ye will, then leave your bourds\\
\vin And use your wit and show it so,
} 
\end{verse}

Madam, without many words and playing around, tell me if you are into 
me or not. If you are, show it to me. In this stanza, the poet is addressing his beloved
and asking her to just tell him if she is interested or not.

\begin{verse}[\versewidth]
{\fontverse
  And with a beck\footnote{\, node or gesture.} ye shall me call; \\
\vin   And if of one that burneth alway \\
Ye have any pity at all, \\
\vin  Answer him fair with yea or nay.
} 
\end{verse}

And with a node from you i will come. And if you have any pity for someone
who is burning for your love, then answer him straight; yes or no. 



\begin{verse}[\versewidth]
{\fontverse
  If it be yea, I shall be fain\footnote{\, happy or glad.}\\
\vin   If it be nay, friends as before;\\
Ye shall another man obtain,\\
\vin   And I mine own and yours no more.
} 
\end{verse}

If yes then I will be happy, if no then we stay friends
and you shall find someone else. Then I will no longer bother you.


\subsubhead{\textit{To His Lady} Summary}

In this poem, the poet asks his beloved a simple question: yes or no. He is asking her
if she wants him or not, expressing this in a very simple and direct way. The poem
theme is courtly love, its tone is practical and the style is direct and simple.

\poemhead{Farewell}
\settowidth{\versewidth}{What should I say,}
\begin{verse}[\versewidth]
{\fontverse
What should I say, \\
Since faith is dead,\\
And truth away\\
From you is fled?\\
Should I be led\\
With doubleness\footnote{\, hypocrisy. Being two-faced.}?\\
Nay\footnote{\, stronger form of no.}, nay, mistress! 
} 
\end{verse}

I have no words for you since you have abandoned honesty and trust.
Should i be led by your hypocrisy? No way mistress. The poet is expressing
his frustration with his beloved after she killed the trust between them 
and lost her honesty. He tells her that he is no longer fooled by her 
deception. 

\begin{verse}[\versewidth]
{\fontverse
I promised you,\\
And you promised me,\\
To be as true\\
As I would be.\\
But since I see\\
Your double heart\footnote{\, loving two people at the same time, being a hypocrite.},\\
Farewell my part! 
} 
\end{verse}

We promised each other to love one another. But now i see your 
deception, i will no longer stay. The poet says that he will leave 
because his beloved loves two people at the same time while lying to them, that
she is a hypocrite.

\newpage
\begin{verse}[\versewidth]
{\fontverse
Though for to take\\
It is not my mind,\\
But to forsake\\
One so unkind \\
And as I find,\\
So will I trust:\\
Farewell, unjust! 
} 
\end{verse}

I no longer want to be with someone so cruel and unkind. Goodbye unjust.


\begin{verse}[\versewidth]
{\fontverse
Can ye say nay?\\
But you said\\
That I alway\\
Should be obeyed?\\
And thus betrayed\\
Or that I wiste—\\
Farewell, unkissed. 
} 
\end{verse}

How could you betray me? You said you will always love me. Goodbye! You don't even deserve a kiss.
The poet asks a \textbf{rhetorical question}\footnote{\, a question that doesn't need an answer and is asked to create dramatic effect.}
to express his amazement of his beloved
betrayal.

\subsubhead{\textit{Farewell} Summary}

In this poem, the poet says goodbye to his beloved because she 
betrayed him, she loved someone else and lied to him. The poet 
express his pain in a frustrated and sad tone. The poem
theme is courtly love, its tone is sad, anger and betrayed, and its
style is simple and direct.


\poemhead{An Appeal}
\settowidth{\versewidth}{And wilt thou leave me thus?}
\begin{verse}[\versewidth]
{\fontverse
And wilt thou leave me thus?\\
Say nay, say nay, for shame,\\
To save thee from the blame\\
Of all my grief and grame;\\
And wilt thou leave me thus?\\
Say nay, say nay!
} 
\end{verse}

Will you really leave me like this? Say no, because its a shameful act and
you will be guilty of being the cause of my misery. In this stanza, the poet asks his 
beloved a \textbf{rhetorical question} in \textit{"And wilt thou leave me thus?"} to express his wonder and frustration.

\newpage
\begin{verse}[\versewidth]
{\fontverse
And wilt thou leave me thus,\\
That hath loved thee so long\\
In wealth and woe among?\\
And is thy heart so strong\\
As for to leave me thus?\\
Say nay, say nay!

} 
\end{verse}

Will you really leave me like this? I have loved you for so long, in good time and bad ones.
Is your heart really that strong to leave me like this? Say no.

\begin{verse}[\versewidth]
{\fontverse
And wilt thou leave me thus,\\
That hath given thee my heart\\
Never for to depart,\\
Nother for pain nor smart;\\
And wilt thou leave me thus?\\
Say nay, say nay!
} 
\end{verse}

Will you leave me like this? I have given you my heart and promised you to never leave, not
for pain nor for suffering. 

\begin{verse}[\versewidth]
{\fontverse
And wilt thou leave me thus\\
And have no more pity\\
Of him that loveth thee?\\
Alas, thy cruelty!\\
And wilt thou leave me thus?\\
Say nay, say nay!
} 
\end{verse}

Will you leave me like this? And have no pity for someone that love you so much?
Alas how cruel you are. The poet expresses his pain and heartbreak because of
his beloved abandonment. He decided to leave her by saying \textit{"alas,they cruelty"}.

\subsubhead{\textit{An Appeal} Summary}


In this poem, the poet asks a rhetorical question, saying: \textit{“after all, you leave me
like this?”} to express his wonder and surprise. He feels heartbroken and frustrated.
The poet uses reputation to emphasis his feelings and message and for music.
In the end of the poem the poet leaves his beloved. The poem theme is courtly love,
its tone is sad and frustrated, and its style is simple and direct.

\poemhead{The Galley}
\settowidth{\versewidth}{My galley, chargèd with forgetfulness,}
\begin{verse}[\versewidth]
{\fontverse
My galley, chargèd with forgetfulness,\\
Thorough sharp seas in winter nights doth pass\\
'Tween rock and rock; and eke mine en'my, alas,\\
That is my lord, steereth with cruelness;
} 
\end{verse}

My ship is weighed down with neglect, it struggles through cold nights
and dangerous seas. It sail through the deadly rocks and between my
enemy. This cruel enemy is my master, he controls the ship. In this stanza, the poet
uses the images of the sea as a metaphor to compare his life to a ship.
He says that he suffers through cold nights and lonely times. His 
cruel beloved controls his life (the ship).

\begin{verse}[\versewidth]
{\fontverse
And every owre a thought in readiness,\\
As though that death were light in such a case.\\
An endless wind doth tear the sail apace\\
Of forced sighs and trusty fearfulness.
} 
\end{verse}

And every oar is an urgent call, as if death would be easier than enduring
this journey. A storm never ending attack the ship and rip off the sail. In this stanza, 
 the poet says that he is never rested through this journey of love. He 
feels that his suffering is endless and death would be better.


\begin{verse}[\versewidth]
{\fontverse
A rain of tears, a cloud of dark disdain,\\
Hath done the weared cords great hinderance;\\
Wreathèd with error and eke with ignorance.\\
The stars be hid that led me to this pain;
} 
\end{verse}

A cloudy storm that rains over my ship damages the ropes 
and hide the star that guide me through my journey. In this stanza, the poet says
that problems between him and his beloved damages their relationship
and takes her away from him (she is his guiding star).


\begin{verse}[\versewidth]
{\fontverse
Drownèd is Reason that should me comfort,\\
And I remain despairing of the port.
} 
\end{verse}

My mind (reason) which should comfort me is drowned. And i
will remain desperate and without hope.

\subsubhead{\textit{The Galley} Summary}

In this poem, the poet uses the \hl{images of the sea as a metaphor} to compare
the journey of the sea to the journey of love. The sea is dangerous,
filed with deadly rocks and storms, even worse there is enemies in the 
sea. Similarly the journey of love is filed with emotional pain and 
obstacles and the beloved is sometimes the enemy. The poet says that
he is the ship that is lost in the sea and the star that guides him is hid by a cloud
(a metaphor for problems). He concludes the poem in despair and hopelessness.
The poem theme is courtly love, its tone is sad and helpless, and its style is
metaphoric.

\poemhead{The Hind}
\settowidth{\versewidth}{Whoso list to hunt, I know where is an hind,}
\begin{verse}[\versewidth]
{\fontverse
  Whoso list to hunt, I know where is an hind\footnote{\, female deer.},\\
But as for me, alas, I may no more.\\
The vain travail hath wearied me so sore,\\
I am of them that farthest cometh behind.
} 
\end{verse}

Who are in the mood to hunt? I know where is a female. As for me
I no longer want to, the pointless pursuit has exhausted me. In this stanza, 
the poet says that he no longer want to purist females because 
it is pointless. 

\newpage
\begin{verse}[\versewidth]
{\fontverse
Yet may I by no means my wearied mind\\
Draw from the deer, but as she fleeth afore\\
Fainting I follow. I leave off therefore,\\
Sithens in a net I seek to hold the wind.
} 
\end{verse}

Even though I try, I keep thinking about her. As I chase her I faint and
give up, because trying to catch her is like trying to catch the wind in a net.


\begin{verse}[\versewidth]
{\fontverse
Who list her hunt, I put him out of doubt,\\
As well as I may spend his time in vain.\\
And graven with diamonds in letters plain\\
There is written, her fair neck round about:
} 
\end{verse}

Who wants to purist her? I assure you it is a waste of time
because there is a writing around her neck engraved with diamond.


\begin{verse}[\versewidth]
{\fontverse
  Noli me tangere\footnote{\, Do not touch me}, for Caesar's I am,\\
And wild for to hold, though I seem tame. 
} 
\end{verse}

Do not touch me because i belong to Caesar. And even though 
i seem gentle, i am wild. 

\subsubhead{\textit{The Hind} Summary}

In this poem, the poet refer to his beloved as Hind. He says that he is tired of
chasing her and that it is a pointless pursuit, because she already belongs
to someone else (Caesar). He says that even though he stopped the chase he still
thinks about her. He used Hind as a metaphor for females 
because it is beautiful and graceful, and because they used to hunt
hinds in his time. The poem theme is courtly love, its tone is sad and helpless, and its style is
metaphoric.


\head{Henry Howard}

\poemhead{To His Lady}
\settowidth{\versewidth}{Set me whereas the sun doth parch the green}
\begin{verse}[\versewidth]
{\fontverse
Set me whereas the sun doth parch the green\\
Or where his beams do not dissolve the ice,\\
In temperate heat where he is felt and seen;\\
In presence prest of people, mad or wise; 
} 
\end{verse}

Place me where the sun is scorching that it kills the green or when it is too light 
that it doesn't dissolve the ice. Place me where the weather is temperate and the sun is
visible or place me in the presence of mad or wise people.

\newpage
\begin{verse}[\versewidth]
{\fontverse
Set me in high or yet in low degree,\\
In longest night or in the shortest day,\\
In clearest sky or where clouds thickest be,\\
In lusty youth or when my hairs are gray.
} 
\end{verse}

Place me in high rank or low one, in the longest night or the shortest day,
in the clearest sky or when the clouds fills the sky, in the prime of youth
or old age.


\begin{verse}[\versewidth]
{\fontverse
Set me in heaven, in earth, or else in hell;\\
In hill, or dale, or in the foaming flood;\\
Thrall\footnote{\, slave.} or at large, alive whereso I dwell,\\
Sick or in health, in evil fame or good:
} 
\end{verse}

Place me in heaven, earth or even in hell. In a  hill,
a valley or in the raging ocean. Enslaved or free, sick 
or in good health, with good reputation or bad one.

\begin{verse}[\versewidth]
{\fontverse
Hers will I be, and only with this thought\\
Content myself although my chance be nought.
} 
\end{verse}

I will be yours and only with this thought i will be satisfied 
even though i know i don't have a chance with you. 

\subsubhead{\textit{To His Lady} Summary}

In this poem, the poet tells his beloved that if she places him in good weather
or bad one, in high rank or low one, sick or in good health, in old
age or youth, with good reputation or bad, enslaved or free, in heaven or in
hell. He will always love her, despite having no chance. The poet makes the
\hl{first 3 quatrains as subject and the couplet as the predicate}, the
whole sonnet serve as one statement. The poem theme is courtly love,
its tone is romantic, and its style is direct.


\poemhead{Spring}
\settowidth{\versewidth}{The soote season, that bud and bloom forth brings,}
\begin{verse}[\versewidth]
{\fontverse
  The soote\footnote{\, sweet.} season, that bud and bloom forth brings,\\
With green hath clad the hill and eke the vale;\\
The nightingale\footnote{\, a type of bird that is known for his signing.} with feathers new she sings,\\
The turtle to her make hath told her tale.
} 
\end{verse}

The sweet season, when the bud and flowers bloom and the hill is covered 
with green. The fresh nightingale sings and the turtledove declare her
love. In this stanza, the poet talks about the rebirth and renewal qualities
of Spring.

\newpage
\begin{verse}[\versewidth]
{\fontverse
Summer is come, for every spray now springs,\\
The hart\footnote{\, male deer.} hath hung his old head on the pale,\\
The buck in brake his winter coat he flings,\\
The fishes float with new repaired scale,
} 
\end{verse}

Summer has arrived, the branches grow and the deer has left his old horns in the fence.
The buck strip his winter coat, the fish float with fresh scale. In this stanza, the poet
talks about Summer and its renewal effect.

\begin{verse}[\versewidth]
{\fontverse
The adder all her slough away she slings,\\
The swift swallow\footnote{\, type of singing birds.} pursueth the flyës smale,\\
The busy bee her honey now she mings—\\
Winter is worn that was the flowers' bale.
} 
\end{verse}

The snake takes off her old skin and the swift swallow pursue little flies.
The busy bee mixes her honey and the winter that has worn the flowers is gone.

\begin{verse}[\versewidth]
{\fontverse
And thus I see, among these pleasant things\\
Each care decays, and yet my sorrow springs.
} 
\end{verse}

Even though I see all this beauty which make us forget our problems, my sadness and pain 
still grows.

\subsubhead{\textit{Spring} Summary}

In this poem, the poet describes Spring and Summer, two renewal seasons, and how 
everything comes to life in these seasons. In \textbf{Spring} the flowers bloom, the
hill is covered with green and the nightingale sings. In \textbf{Summer} the branches 
grow, the buck take off his winter coat, the fish swim with fresh scale, the snake
take its old skin off and the busy bee mixes her honey. Even though he sees all
this beauty before him, beauty that takes off the cares, his pain and sadness still
grows. The poet makes the \hl{first 3 quatrains as subject and the couplet as the predicate},
the whole sonnet serve as one statement. The poem theme is courtly love, its tone is 
sad and helpless, and its style is descriptive.

\head{Edmund Spenser}

He was widely known poet in the sixteenth century and was famous for inviting: 

\begin{itemize}

\item  \textbf{Spenserian  Stanza:} A nine-line stanza rhyming (abab bcbc c).
The first 8 lines are pentameter, and the last one is \underLine{hexameter}. Edmund used 
the Spenserian stanza in his epic poem \underLine{The Faerie Queene}.
\item \textbf{Spenserian Sonnet:} Is a poem of fourteen lines of iambic pentameter rhyming
(abab bcbc cdcd ee). This sonnet is made of 3 quatrains that are \underLine{interlocking} in
rhyme and a concluding couplet.
  
\end{itemize}

\subhead{AMORETTI} 

\hl{It is a sonnet sequence dedicated to one person}.
Edmund wrote these sonnets for his wife \underLine{Elizabeth}. In sonnet sequence the sonnets are 
\underLine{numbered not named}.
Amoretti means \underLine{little love messages or little cupid}.\medbreak


\textbf{Allegory: } Is a story with two levels of meaning, a surface one and a hidden (deep)
one.

\poemhead{Sonnet 1}
\settowidth{\versewidth}{Happy ye leaves when as those lilly hands,}
\begin{verse}[\versewidth]
{\fontverse
  Happy ye leaves\footnote{\, pages, specifically the pages of his poetry.} when as those lilly hands,\\
Which hold my life in their dead doing might\\
Shall handle you and hold in loves soft bands,\\
Lyke captives trembling at the victors sight.
} 
\end{verse}

Happy is the pages when those delicate hands touch them. Hands that are deadly and mighty 
because they hold my life. Those hands will hold you (the pages) gently. Those pages
are like the prisoners trembling before their conqueror. In this stanza, the poet envies the pages
for being hold in his beloved's hands, hands that are deadly because they control his life.
He compares the fear of the captives when they see their conquerors to his fear when he
sees his beloved. 

\begin{verse}[\versewidth]
{\fontverse
  And happy lines\footnote{\, verses of his poetry.}, on which with starry light,\\
Those lamping eyes will deigne sometimes to look\\
And reade the sorrowes of my dying spright,\\
Written with teares in harts close bleeding book.
} 
\end{verse}

Happy is those lines when your radiant and shining eyes look at them. 
And when your eyes read my lines, they can see the sorrows and my dying spirit.
Lines that are written with tears and bleeding heart. In this stanza, the poet
envies the lines for being seen by his beloved's bright and shining eyes. Lines
that are written with tears and are filled with sorrows.

\begin{verse}[\versewidth]
{\fontverse
And happy rymes bath’d in the sacred brooke,\\
Of Helicon\footnote{\, in Greek mythology it is a sacred river favored 
by the gods for its poetic inspiration.} whence she derived is,\\
When ye behold that Angels blessed looke,\\
My soules long lacked foode, my heavens blis.
} 
\end{verse}

And happy is those rhymes that are bathed in the sacred river of Helicon, the
same place that my beloved came from. That her angelic 
and blessed look is heaven itself. It is what i long for and 
what my soul needs. In this stanza, the poet envies the rhymes that are heard 
and enjoyed by his beloved. He says that this is food to his soul.


\begin{verse}[\versewidth]
{\fontverse
 Leaves, lines, and rymes, seeke her to please alone,\\
Whom if ye please, I care for other none. 
} 
\end{verse}

Pages, verses and rhymes, seek to please her alone, its the only thing that
I care about. 

\subsubhead{\textit{Sonnet 1} Summary}

In this poem, the poet envies the pages, the lines and the rhymes of his poems that his
beloved reads. He employees the three senses: \textbf{Touch}, That his beloved will touch the pages of his poem.
\textbf{Sight}, That his beloved will see the lines of his poem. \textbf{Hearing},
That his beloved will hear the rhymes of his poem. He describes his beloved's eyes
as flashing and radiant, her look as blessed and heavenly, her hands as delicate
and powerful because they control his life. The poem theme is courtly love,
its tone is longing and sad, and its style is descriptive.


\poemhead{Sonnet 34}
\settowidth{\versewidth}{Lyke as a ship that through the Ocean wyde,}
\begin{verse}[\versewidth]
{\fontverse
Lyke as a ship that through the Ocean wyde,\\
by conduct of some star doth make her way.\\
whenas a storme hath dimd her trusty guyde.\\
out of her course doth wander far astray:
}
\end{verse}

Like a ship sailing across the vast sea guided by a star to make her way.
When a storm hits it creates clouds that blocks the view of the star, making
the ship lose its way. In this stanza, the poet compare himself to a ship using
\textbf{simile}. He is the ship and his beloved is the guiding star, the storm represent
the problems between them, causing him to be astray from his course. 

\begin{verse}[\versewidth]
{\fontverse
So I whose star, that wont with her bright ray,\\
me to direct, with cloudes is ouercast,\\
doe wander now in darknesse and dismay,\\
through hidden perils round about me plast.
} 
\end{verse}

So I (the ship) once had a star that shine brightly but the clouds block its way.
So now I wonder without the guidance of the star, through dark and danger sea. In this
stanza, the poet says that he once had a beloved but problems separated between
them. Now he wonders alone and in despair.


\begin{verse}[\versewidth]
{\fontverse
Yet hope I well, that when this storme is past\\
My Helice\footnote{\, in Greek mythology she is said to be the nurse of Zeus.} the lodestar of my lyfe\\
will shine again, and looke on me at last,\\
with louely light to cleare my cloudy grief,
} 
\end{verse}

I still hopes that once the storm is over, my guiding star which is 
the center of my life will reappear and shine on me again. Her shining light
will make my cares and pains disappears. In this stanza, the poet says that
he hopes once their problems is over, his beloved will return to him again 
to take all of his grief and misery.

\begin{verse}[\versewidth]
{\fontverse
Till then I wander carefull comfortlesse,\\
in secret sorow and sad pensiuenesse.
} 
\end{verse}

Until then (when the storm ends) I will wonder without any comfort or hope.
In private sorrow and miserable reflections. In the couplet the poet
conclude that unless the storm is over and his star shines, he will be
miserable and sad.

\subsubhead{\textit{Sonnet 34} Summary}

In this poem, the poet uses the sea image to compare his life to a ship, his beloved
to a star, life to the sea and a storm to problems. In the sonnet 
a storm hits and causes the ship to lose its guide (the star) and go astray
without any course. He hopes that once the storm is over, his star
can finally shine again and make his pain go away. The poet use 
a simile to compare himself to a ship. The poem theme is courtly love,
its tone is hopeful and optimistic, and its style is descriptive


\poemhead{Sonnet 54}
\settowidth{\versewidth}{Of this worlds Theatre in which we stay,}
\begin{verse}[\versewidth]
{\fontverse
Of this worlds Theatre in which we stay,\\
My love lyke the Spectator ydly sits\\
Beholding me that all the pageants play,\\
Disguysing diversly my troubled wits.
} 
\end{verse}

In this world of theatre where we live, my beloved sits there with the audience watching idly.
She watches me act all these dramatic scenes while I'm  hiding my troubled mind in my disguise.
In this stanza, the poet compares life to a theatre, his beloved to the audience, and he to the actor.


\begin{verse}[\versewidth]
{\fontverse
Sometimes I joy when glad occasion fits,\\
And mask in myrth lyke to a Comedy:\\
Soone after when my joy to sorrow flits,\\
I waile and make my woes a Tragedy.
} 
\end{verse}

Sometimes i feel joy when a happy moment arrives and I put on a cheerful mask
like my life is a comedy. But soon after when my happiness turns into sadness,
I cry making my life like a tragedy. In this stanza, the poet compares his
happy times to comedy, his sad times to tragedy.

\begin{verse}[\versewidth]
{\fontverse
Yet she beholding me with constant eye,\\
Delights not in my merth nor rues my smart:\\
But when I laugh she mocks, and when I cry\\
She laughes, and hardens evermore her hart.
} 
\end{verse}

Yet she watches me with a steady eye, not taking joy in my happiness 
nor pity my sadness. And when i laugh she mocks me, when i cry she laughs
at me, and her heart becomes colder. 

\begin{verse}[\versewidth]
{\fontverse
What then can move her? if not merth nor mone,\\
She is no woman, but a sencelesse stone. 
} 
\end{verse}

If not my happiness nor my grief can move her, then what can? 
She is not a human but a senseless stone.

\subsubhead{\textit{Sonnet 54} Summary}

In this poem, the poet uses the \hl{stage image} to compare life to a theatre,
his beloved to the audience, he is the actor, his happy times as comedy
and his sad times as tragedy. He describe his beloved as a senseless stone because when he
preforms comedy she mocks him, when he preforms tragedy she laughs at him. The poem theme is courtly love,
its tone is depressed, and its style is descriptive. 


\head{Christopher Marlowe}

\textbf{Blank verse:} Is a poetry written with regular metrical but unrhymed lines, usually in
iambic pentameter. Christopher Marlowe was one of the earliest to use blank 
verse in the sixteenth country.\medbreak

\textbf{Pastoral Poetry:} Poetry which focuses on the simple country side life of 
shepherds and rustic people.


\poemhead{The Passionate Shepherd to His Love}
\settowidth{\versewidth}{Come live with me and be my love,}
\begin{verse}[\versewidth]
{\fontverse
Come live with me and be my love,\\
And we will all the pleasures prove,\\
That Valleys, groves, hills, and fields,\\
Woods, or steepy mountain yields.
} 
\end{verse}

Come live with me and be my love, we will experience every kind of pleasure in
those valleys, forests, hills and beautiful mountains. In this stanza, the poet
invite his beloved to live with him in the country side.

\begin{verse}[\versewidth]
{\fontverse
And we will sit upon the Rocks,\\
Seeing the Shepherds feed their flocks,\\
By shallow Rivers to whose falls\\
Melodious birds sing Madrigals.
} 
\end{verse}

And we will sit upon the rocks seeing shepherds feed their sheep by a calm
river where birds sing beautiful melodies.

\begin{verse}[\versewidth]
{\fontverse
And I will make thee beds of Roses\\
And a thousand fragrant posies,\\
A cap of flowers, and a kirtle\\
Embroidered all with leaves of Myrtle\footnote{\, a type of a plant.};
} 
\end{verse}

And I will make for you a bed of roses, and gather for you a thousand flower.
I will make for you a crown and dress made of flower and decorate the dress 
with myrtle leaves.

\begin{verse}[\versewidth]
{\fontverse
A gown made of the finest wool\\
Which from our pretty Lambs we pull;\\
Fair lined slippers for the cold,\\
With buckles of the purest gold;
} 
\end{verse}

I will give you a gown made from the best wool that is taken from our lambs.
And I will give you a beautiful and cozy slippers that has gold buckles.

\newpage
\begin{verse}[\versewidth]
{\fontverse
A belt of straw and Ivy buds,\\
With Coral clasps and Amber studs:\\
And if these pleasures may thee move,\\
Come live with me, and be my love.
} 
\end{verse}

I will make a belt made of straw and ivy leaf and decorate it
with coral and amber. If these pleasures delight you, then come
live with me and by my love.


\begin{verse}[\versewidth]
{\fontverse
The Shepherds’ Swains shall dance and sing\\
For thy delight each May-morning:\\
If these delights thy mind may move,\\
Then live with me, and be my love.
} 
\end{verse}

Young shepherds will dance and sing for you every morning in May.
If these pleasures move you, then come live with me and be my love.

\subsubhead{\textit{The Passionate Shepherd to His Love} Summary}

In this poem, the poet invites his beloved to live with him in the country
side. He tries to convince her by giving her an idealized view of the country
side. He promise her of: A bed made out of roses, crown made out of
flowers, gown made out of the finest wool, slippers made out of gold buckles
and a belt made of straw and ivy flowers. This poem is a \underLine{pastoral poem} 
that is written in \underLine{blank verse}. The poem theme is courtly love,
its tone is dreamy, and its style is direct.



\head{Walter Raleigh}

\textbf{Parody:} Is a poem which gives a comic imitation of another work for a satirical
purposes. When poets uses parody to critique another poem, they use the same rhyme, the
same stanza lines, and the same metre. A good \hl{example of parody is \textit{"The Nymph’s 
Reply to the Shepherd"}}.


\poemhead{What is our life?}
\settowidth{\versewidth}{WHAT is our life? The play of passion.}
\begin{verse}[\versewidth]
{\fontverse
What is our life? The play of passion.\\
Our mirth? The music of division:\\
Our mothers’ wombs the tiring-houses be,\\
Where we are dressed for life’s short comedy.
} 
\end{verse}

What is our life? It is a theatre of emotions, our happiness? Is
the music between scenes. Our mothers' wombs is like
the changing room of the play where we are dressed for this 
short comedy play. In this stanza, the poet compares life to a theatre,
happiness to the fleeting music that are played during scenes, our mothers'
wombs to the backstage of a theatre and our life to a short comedy. 

\newpage
\begin{verse}[\versewidth]
{\fontverse
The earth the stage; Heaven the spectator is,\\
Who sits and views whosoe’er doth act amiss.\\
The graves which hide us from the scorching sun\\
Are like drawn curtains when the play is done.
} 
\end{verse}

The earth is the stage and God is the audience. He watches us closely
to see if someone makes a mistake. The graves, which protect us from
sun, are like the curtains when the play is over. 

\begin{verse}[\versewidth]
{\fontverse
Thus playing post we to our latest rest,\\
And then we die in earnest, not in jest\footnote{\, joke.}.
} 
\end{verse}

After acting our part we head to our final rest (death) and then 
we die a real death. In the couplet the poet explains that our whole
life is a joke and that only death is real.

\subsubhead{\textit{What is our Life} Summary}

In this poem, the poet compares life to a theatre, earth to a
stage, God to the audience, our life to a comedy, our mothers'
womb to a changing room, the happy times to the music between
scenes and the grave to the curtains when the play is done. The poet
describes life as a ridicules play, happiness comes at short intervals, and
we are marching to one thing; death which is the only serious thing. This
poem is a philosophical poem that uses the images of the stage, its tone
is pessimistic, and its style is metaphoric.


\poemhead{The Nymph’s Reply to the Shepherd}
\settowidth{\versewidth}{If all the world and love were young,}
\begin{verse}[\versewidth]
{\fontverse
If all the world and love were young,\\
And truth in every Shepherd’s tongue,\\
These pretty pleasures might me move,\\
To live with thee, and be thy love.
} 
\end{verse}

If the world and love were eternal and every promise made by shepherds were true.
Then these pleasures delight and move me and I accept to live with you and be 
your love. In this stanza, the poet replies to Marlow's poem \textit{"The 
Passionate Shepherd to His Love"}. 


\begin{verse}[\versewidth]
{\fontverse
Time drives the flocks from field to fold,\\
When Rivers rage and Rocks grow cold,\\
And Philomel\footnote{\, is a woman in Greek mythology who is 
transformed into a nightingale.} becometh dumb,\\
The rest complains of cares to come.
} 
\end{verse}

When the season change the sheep flocks go back to the barn,
when the rivers rage the rocks grow cold and the nightingale becomes
quite. And when you finally rest at night, all the pains of the day
surface.


\newpage
\begin{verse}[\versewidth]
{\fontverse
The flowers do fade, and wanton fields,\\
To wayward winter reckoning yields,\\
A honey tongue, a heart of gall,\\
Is fancy’s spring, but sorrow’s fall.
} 
\end{verse}

The flowers and fields wilt when the harsh winter arrives. Sweet words
hide a cruel heart, your love is like the spring, it ends in fall.

\begin{verse}[\versewidth]
{\fontverse
Thy gowns, thy shoes, thy beds of Roses,\\
Thy cap, thy kirtle, and thy posies\\
Soon break, soon wither, soon forgotten:\\
In folly ripe, in reason rotten.
} 
\end{verse}

The gowns, the shoes, the bed of roses, the crown, the dress and the flowers
are all things that wilt in time and tear apart, like a ripe fruit that
rot in time.

\begin{verse}[\versewidth]
{\fontverse
Thy belt of straw and Ivy buds,\\
The Coral clasps and amber studs,\\
All these in me no means can move\\
To come to thee and be thy love.
} 
\end{verse}

The belt made of straw and ivy, the earrings with coral and amber.
All these does not move me to come and be your love.

\begin{verse}[\versewidth]
{\fontverse
But could youth last, and love still breed,\\
Had joys no date, nor age no need,\\
Then these delights my mind might move\\
To live with thee, and be thy love.
} 
\end{verse}

But if youth, love and happiness could be eternal, and aging did not
exist, then these delights might move me, and I might consider being your
love.

\subsubhead{\textit{The Nymph’s Reply to the Shepherd} Summary}

In this poem, the poet replies to Marlowe’s poem \textit{"The Passionate Shepherd to his
love"}. He gives a realistic view of the country side, saying that when the winter comes
flowers wilt, birds hide, the sheep goes to their barn, the gown and shoes and bed of roses
will be be broken and torn apart in time. The poet tries to paint an image of the real 
life of the country side, speaking through a Nymph, replying to each description of 
Marlow’s poem, in the couplet the Nymph conclude that if youth and happiness were
eternal, then \textit{maybe} she would accept to be his love. The poem theme is courtly love,
its tone is realistic, and its style is descriptive. 

\newpage
\head{Past Exam Sheets \& Answers}

\subhead{First Exam, morning study, A \& B}\bigbreak

\subsubhead{Q1/ Define the following: Metonymy, Terza rima, Ottava rima, Synecdoche.}

  \df{Metonymy:}Is a figure of speech where an object name is replaced by another
closely connected to it.\smallbreak

  \df{Synecdoche:}A figure of speech that uses a part of something to refer to the
whole.\smallbreak

  \df{Terza rima:}A rhyming stanza of three lines.\smallbreak

  \df{Ottava rima:}A rhyming stanza of eight lines.\smallbreak


\subsubhead{Q2/ Fill In the Blank}

\begin{enumerate}

  \item The most important poetic form which flourished with the Renaissance is \underLine{sonnet}.
  
  \item The sonnet is a poem of 14 \underLine{Iambic pentameter} lines of verse. (write the meter)

  \end{enumerate}


\subsubhead{Q3/ Name three major factors that helped the rise of Renaissance in the 16th century.}

\begin{enumerate}

  \item Humanism.

  \item Secularism.

  \item Naturalism.
  
\end{enumerate}

\subsubhead{Q4/ Name three major themes of English Renaissance poetry.}

\begin{enumerate}

  \item Courtly Love.

  \item Time.

  \item Death.
  
\end{enumerate}

\subsubhead{Q5/ Name three of the most common images in English Renaissance poetry.}

\begin{enumerate}

  \item Images of the sea. 

  \item Life as a stage.

  \item Nature.
  
\end{enumerate}

\subsubhead{Q6/ Name three inventions and discoveries of the Renaissance that changed the world.}

\begin{enumerate}

  \item Geographical Discoveries, such as Cape route to India and The New World (America).

  \item The Invention of Printing in 1445.

  \item The Copernican system.
  
\end{enumerate}


\subsubhead{Q7/ Name three main influences by Geoffrey Chaucer on the English Renaissance poetry.}

\begin{itemize}

\item His employment of rhyme and regular metre.
\item His huge contribution to middle English.
\item His impact on native literature.
  
\end{itemize}


\subsubhead{Q8/ In “Farewell”, Thomas Wyatt is taking a decision, what is that decision? How does he justify
it?}

In Farewell Thomas decided to leave his beloved because she was loving two people at the same time.
He calls her a hypocrite and accuses her of having a "double heart".


\subsubhead{Q9/ In “To His Lady”, Thomas Wyatt is making an offer to his lady, what is that offer? How does
he justify it?}

In To His Lady Thomas offers his beloved a simple question: do you want to be my love? Yes or no? If yes
then we be lovers, if no then we stay friends.


\subhead{First Exam, evening study, A \& B}\bigbreak

\subsubhead{Q1: Define the following: The Copernican System; English sonnet; Octave;
the Geographical discoveries; Italian sonnet; rhetorical question.}

\df{The Copernican System:}A system that is developed by Copernicus that advocated 
for the idea that the earth and other planets revolve around the sun.\medbreak

\df{English sonnet:}Is a sonnet consist of 3 quatrains and couplet rhyming
abab,cdcd and the climax is in the couplet.\medbreak

\df{Octave:}The first 8 lines of the sonnet and it is sometimes divided into 2 Quatrain.\medbreak

\df{The Geographical discoveries:}There were attempts to find new routs to find silk and precious stone,
this led to the discovery of both Cape route to India and The New World (America).\medbreak

\df{Italian Sonnet:}Is a sonnet composed of 14 lines of octave (first 8 lines) and sestet (last 6 lines) rhyming
abba,abba.\medbreak

\df{Rhetorical Question:}A question that doesn't need an answer and is asked to create dramatic effect.


\subsubhead{Q2: Write a brief note on the contributions of Chaucer in the development of Renaissance
poetry. (4 Marks)}

Chaucer is one of the greatest poets in the Middle Ages, he was called the father of English poetry.
He influenced later poets especially English poets in the sixteenth century. His influence 
appeared in his employment of rhyme and regular metre, his huge contribution to middle English 
and his impact on native literature.


\subsubhead{Q3: Read the following lines and answer the questions below: (10 Marks)}

\settowidth{\versewidth}{if it be yea, i shall be fain;mine. }
\begin{verse}[\versewidth]
{\fontsub
\textbf{
if it be yea, i shall be fain;\\
if it be nay, friends as before;\\
ye shall another man obtain,\\
and i mine own and yours no more.
}
} 
\end{verse}

\df{
1. Name the poet.\\
2. What is the title of the poem?\\
3. What is the tone of these lines?\\
4. Who is the speaker?\\
5. Explain the lines.
}

\begin{enumerate}

  \item Thomas Wyatt.
  \item To His Lady.
  \item Practical, Rational.
  \item A lover.
  \item In these lines the poet says: If yes then I will be happy,
    if no then we stay friends and you shall find someone else.
Then I will no longer bother you. The poet asks his beloved simply, Yes? We be lovers,
No? We stay friends and i go my way.
  
\end{enumerate}


\subsubhead{Q4: Read the following lines and answer the questions below: (10 Marks)}

\settowidth{\versewidth}{And you promised me, }
\begin{verse}[\versewidth]
{\fontsub
\textbf{
I promised you\\
And you promised me,\\
To be as true,\\
As I would be.\\
But since I see\\
Your double heart\\
Farewell my part!
}
} 
\end{verse}

\df{
1. Name the poet.\\
2. What is the title of the poem?\\
3. What is the tone of these lines?\\
4. What is the theme of the poem?\\
5. Explain the lines.
}

\begin{enumerate}

  \item Thomas Wyatt.
  \item Farewell.
  \item Sad, Angry, Betrayed.
  \item Courtly Love.
  \item In these lines the poet says: We promised each other to love one another. But now i see your 
deception, i will no longer stay. The poet say that he will leave her
because his beloved loves two people at the same time while lying to them, she is a hypocrite.

\end{enumerate}



\end{document}
