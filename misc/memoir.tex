\documentclass[12pt, a4paper]{article}
\usepackage{geometry}
\usepackage{lua-ul}
\usepackage{graphicx}
\usepackage{enumitem}
\usepackage{color,soul}
\usepackage{enumitem}
\usepackage{defaultpreamble}


\setmainfont{Bembo Book MT Pro}



\begin{document}

\phantomsection
\section*{\centering\fontsub{I \\ The Desert}}
\addcontentsline{toc}{section}{The Desert}\vspace{0.5cm}

It has been such a long time since I started living here
that I do not remember a time when I wasn't. So many days
have I longed for water, and so many days lie ahead. It is like
an ever-demanding mistress who is not easily pleased, but when
she is, it is as if you have won the lottery. I don't know 
why water is so scarce or why it is so expensive. Economists
might explain that by the rule of ''higher demand, higher value,''
but this answer does not satisfy me; the issue is deeper than this.

Sometimes when I see people get water, I cannot help but feel
envious of them, resent the world, and blame myself. My mind starts
to think: \textit{How did they get it? Why don't I have it?} And 
sometimes I'm afraid of the answer. What if I'm the reason for
my lack of water? Doubts enter my mind; it seems I can never
have the confidence of knowledge. Even though I do know that 
there are a lot of things that come between me and water, that 
it takes time, a lot of time. People devote their whole life
to the source of water, making such compromises just to
have it. And it makes me wonder about our nature: Are we just animals seeking
water no matter what? Is there any value that we will not compromise?

Some days are hot; others have clouds that
help lessen the beams of the sun. These are my lucky days. 
Have you ever wanted clouds to appear, not wished it but 
actually willed it? Even with the knowledge
of externals, it seems that to be conscious is not the endpoint, 
but an active discipline. Nights are the worst. Even though they hide the scorching
sun, I feel cold and detached, is there anything but darkness that reminds one of death?

This thirst! A never-ending thirst, sometimes overwhelms me.
I lose myself; I lose the desire itself; I begin to want... destruction!
Greed and frustration. Like a huge wave that takes me off the shore, 
fills my whole being with water, suffocates me but leaves me before I die.
Often, I see images -- a distorted, quick images -- of drinking which flash into my
mind. They serve to relieve the desire, as an outlet of some sort. But this only adds to 
the desire; it only enlarges the wave. And what else do I do but to scorn myself, for
allowing such images enter my mind, for contributing to this never-ending desire.

\newpage
\phantomsection
\section*{\centering\fontsub{II \\ The Hellfire}}
\addcontentsline{toc}{section}{The Hellfire}\vspace{0.5cm}

Since when did people start to live most of their lives in such a scorching place?
Is this really how life is lived now for most of us ? Or am I just not
capable enough to leave this place? This dread of never knowing, keeps me on the edge.
People spend years in this place,
making the highest compromises: lies, cheating, and corrupting their very essence, a soul
numbing pursuit, using all means necessary to get out. Some lucky
people, who think you are here by choice, blame you, shame you
or worse: they might pity you.
It is like it is not enough that you are already there, struggling, but you
also have to endure such condescending people. They will attack you, try to take the
very essence of you, an essence they built to make you align with 
their system. And until you have the proof, until you leave this place,
you are bound to these things.  It is like
being in the middle of the desert, clueless, with no idea where to go and what to do, and your best 
chance is to just move. But there is freedom in the desert; is there any place that grants such freedom?
It is often said that the price of freedom is high. And since we inherited such freedom, since it
is our destiny, why not enjoy it?



  
\end{document}
