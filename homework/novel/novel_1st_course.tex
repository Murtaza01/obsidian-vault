\documentclass[12pt, a4paper]{article}
\usepackage{geometry}
\usepackage{lua-ul}
\usepackage{graphicx}
\usepackage{enumitem}
\usepackage{color,soul}
\usepackage[fonthead={Goudy Oldstyle Std},
fontsub={Goudy Oldstyle Std}]{defaultpreamble}


\setmainfont{Bely}


\newgeometry{top=0.8in}

\begin{document}


\titlehead{Joseph Andrews}{Henry Fielding}


\head{Preface}

\textbf{The history of the adventures of Joseph Andrews and of his friend Mr. Abraham Adams} is the 
title of Henry Fielding's novel. The subtitle reads \textbf{"written in imitation of the manner of Cervantes,
the author of Don Quixote"}, which explains the main influence of the novel. 

\subhead{Henry Theory of Novel}

\ind In his preface, Henry explains that his work is different from the well known
literary forms of his time. Joseph Andrews is a \hl{comic romance}, a \hl{comic epic
poem in prose}. It has the length of the epic, but differs from serious romance
in being light and dealing with the ridiculous. Although the style sometimes 
includes burlesque\footnote{\, to use elegant style to describe or present 
inferior topics and ideas.} imitation, the novel is not a burlesque, as the characters are
based on those found in real life. Henry derives \hl{his influence from Homer's\footnote{
\, Greek poet.} lost satirical epic}. He explains that his work deals with inferior subjects
in an elevated style.

\subhead{Burlesque}

\ind Henry explains that he only uses burlesque in the level of \hl{diction}
and does not use it in any other way because it deforms human nature. He 
distinguishes burlesque from comic, for the two are widely different. He
explains the similarity between Carictura\footnote{\, paintings that aim to
exaggerate human features to show monsters.} and burlesque saying,
\underLine{"What Carictura in painting burlesque is in writing"}. Henry \hl{uses
burlesque to describe the character of Mrs. Slipslop.}


\subhead{Affectation and the Ridiculous}

\ind Henry says that his work will focus on the ridiculous rather than the sublime.
He explains that the ridiculous arises from affectation, and 
affectation itself arises either from vanity or hypocrisy.
He considers hypocrisy to be a much worse vice and more ridiculous than vanity.

\subsubhead{Vanity}

A vain man exaggerates his virtues. In the novel, vanity is shown in Parson Adams who
believes his learning makes him wiser than  others. Nevertheless, his vanity is fairly 
harmless when compared with other characters.

\subsubhead{Hypocrisy}

A hypocrite person hides his vices under an appearance of their opposite virtues. 
In the novel, hypocrisy is shown in 
Lady Booby and Mrs. Slipslop, who pretend to be chaste while pursuing Joseph Andrews.
They continually make themselves ridiculous because of their hypocrisy.







  
\end{document}
