\documentclass[12pt, a4paper]{article}
\usepackage{geometry}
\usepackage{lua-ul}
\usepackage{graphicx}
\usepackage{enumitem}
\usepackage{color,soul}
\usepackage{enumitem}
\usepackage[fonthead={Goudy Oldstyle Std},
fontsub={Goudy Oldstyle Std}]{defaultpreamble}


\setmainfont{Bely}


\begin{document}

\newgeometry{top=0.6in}

\enlargethispage{\baselineskip}
\titlehead{Joseph Andrews}{Henry Fielding}


\head{Preface}

\ind \textbf{The history of the adventures of Joseph Andrews and of his friend Mr. Abraham Adams} is the 
title of Henry Fielding's novel, which is a parody critiquing Pamela by Samuel Richardson.
Fielding's previous novel \hl{--Shamela-- was a parody} and a direct response to Richardson's Pamela.
The subtitle of \textit{Joseph Andrews} reads,\textbf{"Written in imitation of the manner of Cervantes,
  the author of Don Quixote,"} which explains the main influence of the novel. \hl{\textit{Don Quixote} is a picaresque
novel.} 

\subhead{Henry Theory of Novel}

\ind In his preface, Fielding explains that his work is different from the well known
literary forms of his time. \textit{Joseph Andrews} is a \hl{comic romance}, a \hl{comic epic
poem in prose}. It has the length of the epic, but differs from serious romance
in being light and dealing with the ridiculous. Although the style sometimes 
includes burlesque  imitation, the novel is not a burlesque, as the characters are
based on those found in real life. Fielding derives \hl{his influence from Homer's\footnote{
\, Greek poet.} lost comic epic\footnote{\, called Margites}}. He explains that his work deals with inferior subjects
in an elevated style.

\subhead{Burlesque}

\ind Burlesque or parody is the use of elegant style to present inferior topics and ideas. 
Fielding explains that he only uses burlesque in the level of \underLine{diction}
and does not use it in any other way because \hl{burlesque deforms human nature}. He 
distinguishes burlesque from comic, saying that comic deals with the inferior while
burlesque exhibits monsters. He
explains the similarity between Carictura\footnote{\, paintings that aim to
exaggerate human features to show monsters.} and burlesque saying,
"What Carictura in painting \underline{burlesque} is in writing". Fielding \hl{uses
burlesque to describe the character of Mrs. Slipslop.}


\subhead{Affectation and the Ridiculous}

\ind Fielding says that his work will focus on the ridiculous rather than the sublime.
He explains that the ridiculous arises from affectation, and 
affectation itself arises either from vanity or hypocrisy.
He considers hypocrisy to be a much worse vice and more ridiculous than vanity.
\hl{Fielding admires Ben Jonson the most} because Ben Jonson understood the ridiculous the best.

\subsubhead{Vanity}

\ind A vain man exaggerates his virtues. In the novel, vanity is shown the character of Parson Adams who
believes his learning makes him wiser than  others. Nevertheless, his vanity is fairly 
harmless when compared with other characters.


\restoregeometry

\subsubhead{Hypocrisy}

\ind A hypocrite person hides his vices under an appearance of their opposite virtues. 
In the novel, hypocrisy is shown in the characters of both
Lady Booby and Mrs. Slipslop, who pretend to be chaste while pursuing Joseph.
They continually make themselves ridiculous because of their hypocrisy.

\subhead{Questions and Quotes}

\quotehead{“The only Source of the true Ridiculous (as it appears to me) is Affectation.”}

These words are said by Fielding in his preface. He rejects burlesque because it shows monsters,
and seeks out comedy because it shows the forms of absurdity that exist in real life. 
To him the true ridiculous arises from the exposure of affectation.

\subsubhead{Q: What is the full title of the novel \textit{Joseph Andrews}?}

The history of the adventures of Joseph Andrews and of his friend Mr. Abraham Adams.

\subsubhead{Q: Why does Fielding satirize the character of Pamela in
Richardson's novel?}

Fielding believed that Richardson's Pamela sets a bad moral
example, turning sex into a commodity, and he found the heroine to be calculating and
manipulative.


\subsubhead{Q: What is affectation, and how is it exposed according to Fielding?
Elaborate with examples.}

Fielding explains that the ridiculous arises from affectation, and 
affectation itself arises either from vanity or hypocrisy.
He considers hypocrisy to be a much worse vice and more ridiculous than vanity.
He exemplifies affectation in the character of Parson Adams, whose vanity makes him 
believe that his learning makes him wiser than others, and in the character of Lady
Booby, whose hypocrisy makes her look ridiculous.

\subsubhead{Q: In his theory of novel, Fielding states that he is attempting a new species of writing. Explain.}

Answer page 1 under \textit{"Henry Theory of Novel"}.


\subsubhead{Q: Why does Fielding define \textit{Joseph Andrews} as a "comic epic-poem
in prose"? Discuss}

Fielding explains that Joseph Andrews has
the length of an epic but differs from the serious romance by focusing on the ridiculous rather than the sublime.
He also explains that his work lacks the meter found in all epics.

\subsubhead{Q: What are the characteristic of Fielding's theory 
of the Ridiculous?}

Fielding explains that the ridiculous arises from affectation, and 
affectation itself arises either from vanity or hypocrisy. He considers hypocrisy to be a 
much worse vice than vanity, because it is more surprising and more ridiculous 
when we discover that someone is the exact opposite of what they claims to be.


\subsubhead{Q: Discuss the theme of hypocrisy in \textit{Joseph Andrews}.}

Answer page 2 under \textit{"Hypocrisy"}.

\fillblank{Q: Fielding does not use \underLine[height=1pt]{burlesque} because it deforms human nature.}
{Q: Fielding does not use ..... because it deforms human nature.}

\falsetrue{Q: Cervantes's \textit{Don Quixote} is a serious romance. True or False?}

\fillblank{Q: Provide a short answer. Parody is defined as a
\underLine[height=1pt]{comic imitation of a literary work}.}
{Q: Provide a short answer. Parody is defined as a ....}

\falsetrue{Q: Vanity is more ridiculous when revelled than
hypocrisy. True or False?}

\circleoption{Q: The appropriate rendering of a character, action, speech or scene
is called .........}{satire}{parody}{\hl{decorum}}{the ridiculous}

\circleoption{Q: In his theory of fiction, Fielding claims 
that he uses burlesque at the level of}
{fable}{characters}{\hl{diction}}{sentiments}

\circleoption{Q: Who is among the following authors that
Fielding admires most?}
{Samuel Richardson}{Homer}{\hl{Ben Jonson}}{Colley Cibber}

\head{Chapter 1: Of Writing Lives in General}

\ind In his theory of fiction, Fielding points out the importance of biography\footnote{
\, writing about someone's life.}, comparing his work with other contemporary works.
He explains that a biography can be more useful to mankind than the great person's
life it records, and that the reader is improved by a mixture of instruction and delight.
He ironically mentions \textbf{Colley Cibber's} autobiography\footnote{\, self-written account of one's own life.}
and \textbf{Samuel Richardson's} Pamela\footnote{\, a biography which talks about female chastity.},
saying that his Joseph Andrews is the brother of Pamela and
that he is an example of \hl{male chastity}.

\subhead{Questions and Quotes}

\quotehead{“It is a trite but true Observation, that Examples work more forcibly on the
Mind than Precepts.”}

These words are said by Fielding explaining the moral utility of the novel:
it can embody virtue in the biographies of
exemplary characters, thereby inspiring our imitation. He ironically cites two contemporary works that 
that inspired imitation of virtue: Richardson’s Pamela and Cibber’s autobiography.

\subsubhead{Q: In his theory of fiction, Fielding points out the importance of Biography and “writing lives in
general”, comparing his work with other contemporary biographies. Explain.}

Fielding explains that a biography can be more
useful to mankind than the great person's life it records, and that the reader is
improved by a mixture of instruction and delight. He ironically cites two contemporary works: 
Colley Cibber’s autobiography and Samuel Richardson’s Pamela.

\circleoption{Q: Which one of the following genres is 
concerned with the task of "Of writing lives in general"? }
{romance}{epic}{history}{\hl{biography}}

\enlargethispage{2\baselineskip}
\headfoot{Chapter 2: Joseph's Genealogy}{\, genealogy is the study of a family tree.}

\ind The narrator says that Joseph is the brother of Pamela and the only son of \textbf{Gaffer} and \textbf{Gammer},
terms of respect for older people of low social rank, which implies that \hl{Joseph comes from low birth}.
He explains that there were many searches for Joseph’s parentage, with little success. He mentions
a \hl{great-grandfather who was a cudgel-player} and an epitaph that mentions \hl{merry man Andrew 
who was part of a sect called the laughing philosophers (Merry Andrews)}. He compares Joseph 
to the Athenians who sprang up from a dunghill (\hl{autokopros}\footnote{\, sprung from a dungill.}).

\subhead{Joseph's Jobs}

\ind At ten years old, Joseph became an apprentice to Sir Thomas Booby. The boy was employed in what 
was called \hl{keeping birds}. Later, he was moved to a dog-kennel\footnote{\, a shelter for dogs.
} where he worked as a \hl{whipper-in}\footnote{
\, a huntsman's assistant.}. He then worked in the stable where he rode in races. At age seventeen,
he became a \hl{footboy}\footnote{\, personal servant.} to Lady Booby.

\subhead{Questions and Quotes}

\quotehead{“Mr. Joseph Andrews, the Hero of our ensuing History, was esteemed to be
the only Son of Gaffar and Gammer Andrews, and Brother to the illustrious
Pamela, whose Virtue is at present so famous.”}

These words are said by Fielding while introducing his title character.
He makes a connection between his hero and Richardson's heroine --Pamela.
By saying that Joseph is the only son of Gaffer and Gammer, terms of respect for
older people of low social rank, Fielding emphasizes the hero's low birth
and signals the satirical, comical nature of his novel.


\subsubhead{Q: Why is Joseph's character considered mock-heroic?}

Because of his unknown and low birth. This is opposed to the epic convention,
where heroes descend from gods, kings, and noble ancestors.

\subsubhead{Q: Why does Joseph fail in his job as a bird-keeper?}

Because his voice was so musical that it allured the birds rather than terrify them.

\subsubhead{Q: What is the significance of Joseph's parents' names?}

They are terms of respect for older people of low social rank.

\subsubhead{Q: Who are the laughing philosophers?}

Is a sect found by one of Joseph's ancestor.

\subsubhead{Q: Who are the Autochthones with whom Joseph is compared?}

The Athenians who claimed to have sprung up from a dunghill.
\enlargethispage{\baselineskip}

\subsubhead{Q: Write down an essay about Joseph Andrews' mock-heroic
genealogy.}

In \textit{Joseph Andrews}, Fielding uses the mock-heroic style to describe Joseph's
family tree. In epic convention, heroes descend from gods, famous 
kings and other grand ancestors. However, the novel reverses this tradition: 
Joseph has no known ancestors; Fielding mentions only two based on a hearsay.
One who was a cudgel-player, and another who belonged to a sect named the Merry Andrews 
(laughing philosophers). Fielding goes further 
by mentioning that Joseph is the only son of Gaffer and Gammer,
terms of respect for older people of low social rank, to 
emphasize his low birth.


\truefalse{Q: The narrator claims that there are no reliable 
sources about Joseph's past. True or False?}

\fillblank{Q: The word "autokopros" means \underLine[height=1pt]{sprung from a dunghill}.}
{Q: The word "autokopros" means ....}

\enlargethispage{\baselineskip}
\head{Chapter 3: Characters Introduction}

\ind In this chapter we are introduced to the main characters: Parson\footnote{\, 
member of the clergy; a priest.}  Abraham Adams,
Mrs. Slipslop, Sir Thomas Booby and his wife Lady Booby. Parson Adams is a scholar from 
the country who wants to \hl{teach Joseph Latin}.
Sir Thomas Booby rarely sees Parson Adams; he values people only according to their
wealth and appearance, and Lady Booby considers country people brutes. This shows the snobbery\footnote{
\, a snob is a person who has a strong sense of class status.}of
the Booby family. Mrs. Slipslop the waiting-gentlewoman, has respect for Adams  but thinks she
is better than him only because she went to London. \hl{Mrs. Slipslop is described as a
\textit{"Mighty affecter of hard words"}}
because she uses words she does not understand only to show her superiority,
this reveals her vanity.


\subhead{Abraham Adams}

\ind Is an excellent scholar who knew Greek, Latin, French, Italian and Spanish.
He spent many years in learning, and is a man of good nature, but at the 
same time as entirely \hl{ignorant of the ways of this world as an infant just
entered to it}. He was generous, friendly and brave but \hl{simplicity} was 
his characteristic. At the age of fifty he worked as bishop with good income,
however, it was not enough to live well with his wife and six children. \hl{His name
is a biblical allusion} to the character of Abraham and Adam. \hl{Parson Adams is similar to 
Don Quixote} for the both are idealistic and eccentric.

\subhead{Theme of Class Distinction}

\ind In the novel we see a few characters showing a strong sense of class status. For example,
Sir Thomas Booby rarely sees Parson Adams because he values people only according to their
wealth and appearance. Lady Booby, who had a town education, considers country people to be brutes.
Mrs. Slipslop, a country woman who had been a few times to London, thinks that she knowns more about the world 
than her fellow countrymen.


\subhead{Questions and Quotes}

\quotehead{"He was generous, friendly and brave to an Excess; but
Simplicity was his Characteristic."}


These words are said by the narrator describing the character of Parson Adams.
He is described as generous and friendly but excessively brave, which is the result of
his lack of prudence. He is also described as naive and simple.

\quotehead{"She had in these disputes a [particular advantage]\footnote{\, this is a verbal irony.}over Adams for she was A
mighty affecter of hard words"}

These words are said by the narrator about Mrs. Slipslop. He says that when Parson Adams and 
Mrs. Slipslop argue, she uses jargon to show her superiority.

\subsubhead{Q: To which character does the term Quixotic apply in Joseph
Andrews, and why?}

To Abraham Adams because he and Don Quixote are idealistic and eccentric.

\subsubhead{Q: Why is Mrs. Slipslop described as "a mighty affector of hard
words"?}

Because she uses jargon to win arguments with Parson Adams.


\subsubhead{Q: How do Mr. Thomas Booby and Lady Booby act in a snobbish
manner?}

Sir Thomas Booby rarely sees Parson Adams; he values people only according to their wealth and appearance, and Lady Booby
considers country people brutes. This shows the snobbery the Booby family.


\subsubhead{Q: Why is parson Adams described as being naive?}

Because he is ignorant of the ways of this world as an infant just entered to it.

\subsubhead{Q: Draw a character sketch of Parson Adams.}

Answer on page 5 under \textit{"Abraham Adams"}.

\fillblank{Q: "Sir Thomas was too apt to estimate men merely
by their dress or fortune" These words are related to the
theme of \underLine[height=1pt]{class distinction} in Fielding's novel.}
{"Sir Thomas was too apt to estimate men merely
by their dress or fortune" These words are related to the
theme of .... in Fielding's novel.}

\circleoption{Q: Which one of these characters is described 
as "a mighty affector of hard words"?}
{Abraham Adams}{Lady Booby}{\hl{Mrs. Slipslop}}{None}


\enlargethispage{3\baselineskip}
\head{Chapter 4: Joseph in London}


\ind Joseph makes some friends in London who teach him how to dress and become 
fashionable. However, they could not teach him to game, swear, drink or
any other vice of the town. He becomes a \hl{connoisseur\footnote{\, 
expert in matters of tastes.}} in music. Lady Booby, who always thought that 
he lacked spirit, begins to change her mind after seeing the effect of the town
on Joseph, saying, "Aye, there is life in this fellow." She started taking walks
with him in Hyde Park and getting close to him. One morning, \hl{Lady Tittle and Lady
Tattle\footnote{\, both are tag-names that means gossip.}} saw them together and
\hl{defamed Lady Booby for being in love  with Joseph}.

\subhead{Theme of Countryside verses City}

\ind The city is always viewed as a place for vices and temptations, while the countryside
is seen as the place of moral goodness and innocence. In the novel, this is shown when Joseph meets friends in London
who try to teach him the vices of the city but fail. However, the city does have a positive influence on
Joseph's fashion and taste. In the novel, people from the city show snobbery and vanity,
thinking themselves superior. Nearly all characters from the city display this; even 
Mrs. Slipslop, who had visited London a few times, feels superior to Parson Adams.


\subhead{Question and Quotes}

\quotehead{"They could not, however, teach him to game, swear, drink, nor any other
genteel vice the town bounded with"}

These words are said by the narrator about Joseph's friends when he was in London.
They could not teach Joseph the vices of the city, which proves Joseph's virtue.

\truefalse{Q: Tittle and Tattle are characters who are 
associated with defamation. True or False?}

\fillblank{Q: The defamers who caught sight of Lady
Booby and Joseph in London are \underLine[height=1pt]{Lady Tattle}
and \underLine[height=1pt]{Lady Tittle.}
}
{Q: The defamers who caught sight of Lady
Booby and Joseph in London are .... and .... }


\head{Chapter 5: The Seduction}

\ind After the death of Sir Thomas Booby, Lady Booby pretends to mourn, while in fact she was
playing cards with her friends for six days. And on the seventh day she calls Joseph to her room,
this is a \hl{Biblical allusion to the creation of the world in seven days}\footnote{\, in the Bible, God has
made the world in six days and rested on the seventh.}. She calls him
a philanderer\footnote{\, general lover.} and tries to seduce him but fails. She tempts
him and assures him that he need not be afraid of their different class. However, Joseph shows himself 
to be chaste and reject her, Lady Booby then kicks him out of the room.
The \hl{seduction of Lady Booby to Joseph Andrews is a Biblical allusion to the advances
of Potiphar's wife to Joseph}.

\subhead{Question and Quotes}

\quotehead{"Then you are either a fool, or pretend to be so; I find I was
mistaken in you."}

These words are said by Lady Booby after Joseph rejects her advances over him.
She feels enraged and finds him to be either stupid or naive to reject her.

\quotehead{“You are a general lover. Indeed, you handsome fellows,
like handsome women, are long and very difficult in fixing; but yet you shall never
persuade me that your heart is so insusceptible of affection.”}

These words are said by Lady Booby to Joseph when she was trying to seduce him. She 
accuses him of being a general lover, a person who takes interest in many people. She
tries to test him to find out whether he is liable to be seduced.

\subsubhead{Q: Why is Joseph Andrews accused of being a philander?}

Because Joseph tells Lady Booby that all the women he had ever seen were
equally indifferent to him.

\circleoption{Q: The seduction episode in Fielding's novel
is take from ....}{Richardson}{Cervantes}
{Richardson \& the Bible}{\hl{the Bible}}

\fillblank{Q: Philanderer means \underLine[height=1pt]{general lover.}
}{Q: philanderer means ..... }

\head{Chapter 6: Letter to Pamela}

\ind Joseph writes a letter to his sister Pamela complaining to her that Lady Booby tried to seduce him.
This letter style of writing a novel is called \hl{epistolary technique\footnote{\, the epistolary technique is used by the writer 
to reveal the character's inner workings and mind. Epistle means letter.} which is used in Richardson's
Pamela}. After Joseph writes the letter he is approached by Mrs. Slipslop who tries to make
love to him violently, but Joseph is saved by the ringing of Lady Booby's bell. \hl{Mrs. Slipslop
is described in burlesque} manner and \hl{compared to a hungry tigress in a grotesque} way by the narrator. 

\subhead{Mrs. Slipslop}

\ind Mrs. Slipslop, the waiting-gentlewoman of Lady Booby, is described in a burlesque way as an old and ugly woman.
Like her mistress -- Lady Booby --
she feels lustful toward Joseph. Although single,
the narrator indicates that she is not a virgin. Her name, which implies she is morally lax, sloppy or careless, 
suggests that she made a "slip" in her past.
The narrator grotesquely compares her to a tiger when she tries to make a sexual advance on Joseph.


\subhead{Question and Quotes}

\quotehead{"As when a hungry tigress, who long has traversed the woods in
fruitless search, sees within the reach of her claws a lamb, she
prepares leap on her prey."}

These words are said by the narrator when Mrs. Slipslop tried to make an advance on Joseph. This is an instance
of epic simile, where the narrator grotesquely compares Slipslop's sexual advance on Joseph to a hungry tiger.

\quotehead{“She resolved to give a loose to her amorous inclinations, and to pay off the 
debt of pleasure which she found she owed herself.”}

These words are said by the narrator about Mrs. Slipslop. He explains that 
having already made amends for past and future mistakes, she is allowed the liberty to be with any man
and to indulge herself in pleasure.

\quotehead{“Don't tell anybody what I write, because I should not care to have
folks say I discover what passes in our family; but if it had not been so 
great a lady, I should have thought she had had a mind on me”}

These words are said by Joseph in his first letter to his sister Pamela.
He tells that if his mistress --Lady Booby-- was not of high rank, he would have
thought that she is infatuated by him. All the while, 
showing a great interest in preserving his family reputation.


\subsubhead{Q: How does Fielding make use of Grotesque in the novel?}

When he compares Mrs. Slopslop's sexual advance on Joseph to a hungry tiger.

\subsubhead{Q: To whom does Joseph write letters, and why?}

To his sister Pamela. To complain to her about Lady Booby's passion.

\subsubhead{Q: Give a character sketch of Mrs. Slipslop.}

Answer on page 9 under \textit{"Mrs. Slipslop"}.

\newpage
\head{Chapter 7: Lady Booby's Passion}

\ind Feeling rejected, Lady Booby wants to rid herself of Joseph. She calls
Mrs Slipslop to her room. The two disappointed women discuss the
young man. Mrs \hl{Slipslop falsely claims that Joseph is a womanizer}\footnote{\,
a male who has casual sex with several women.}. Lady Booby is divide against herself:
she wants to kick Joseph out, and then she wants to keep him nearby. It seems that Lady 
Booby has been hit by one of \hl{Cupid's\footnote{\, in Greek mythology, Cupid meaning passionate desire,
is the god of desire, erotic love and attraction.} arrows}. Lady Booby finally decides 
that she will insult then dismiss him.


\subhead{Question and Quotes}

\quotehead{“To my knowledge he games, drinks, swears and fights eternally; besides
he is horribly addicted to wenching.”}

These words are said by Mrs. Slipslop about Joseph when she was talking to Lady Booby.
She falsely tells Lady Booby that Joseph games, drinks, swears and has casual sex
with the servants of the house. 

\circleoption{Q: "O Love, what monstrous tricks dost thou play with thy votaries of both sexes! How dost thou deceive them, and make them deceive 
themselves" These words exemplify the use of ......}{burlesque}
{\hl{epic invocation}}{verbal irony}{epic simile}


\head{Chapter 8: Man's Virtue}

\ind The narrator describes Joseph's charms to make the 
reader understand and excuse Lady Booby. Joseph is interviewed by Lady Booby about his 
supposed misconduct with the maids of the house.
She further tries to seduce him but he tells her that he is virtuous. She does not believe that
a man can be virtuous, but he answers her that he is the brother of Pamela. She becomes outraged and
kicks Joseph out of the house. 

\subhead{Foreshadowing}

\ind Foreshadowing is a hint given earlier in the narrative 
about what is going to happen later in the story. A good example
is when the narrator talks about Joseph saying,\textit{"To those who have not
seen many noblemen, would give an idea of nobility"}. This foreshadows that
Joseph is not from a humble origin.

\subhead{Question and Quotes}

\quotehead{"Did ever mortal hear of a man's virtue? Did ever the greatest
or the gravest men pretend to any of this kind?"}

These words are said by Lady Booby to Joseph after she tries to seduce him. When Joseph
rejects her advances and declares himself a virtuous man, she is shocked because
she does not believe that a man can be virtuous and thinks that he is a hypocrite.

\head{Chapter 9: Lady Booby and Slipslop's Hypocrisy}

\ind Mrs. Slipslop, having listened through the keyhole to the conversation between 
Lady Booby and Joseph, is no longer afraid of her mistress -- Lady Booby -- and freely mocks her.
Lady Booby becomes worried about her reputation but thinks that she could bribe Mrs. Slipslop
into secrecy. What had hurt her the most was that she still had feelings for Joseph. 


\subhead{Question and Quotes}

\subsubhead{Q: Define and exemplify the term internal conflict.}

An internal conflict is the struggle that occurs within a character's mind. In the novel,
we see this in the character of Lady Booby when she struggles over whether to dismiss Joseph
or let him stay. Her passion for him seems to blind her reason. However, at the end she finally
decides to let him go.


\head{Chapter 10: Joseph Hits the Streets}

\ind Joseph writes a second letter to Pamela, complaining that Lady Booby has fallen in
love with him. In the letter he says that Parson Adams has told him that chastity 
is as great a virtue in a man as in a woman and promises his sister Pamela that he will
imitate her chastity. After receiving his remaining wages from Peter Pounce\footnote{\,
the steward of Lady Booby, he grows rich by robbing the servants of their salaries.}, Joseph ends up in
the street. From now on, \hl{Joseph is going to lead a picaresque life}, moving from inn\footnote{\,
similar to hotels but smaller.} to inn.


\subhead{Question and Quotes}

\quotehead{"Mr. Adams hath often told me that chastity is as great a
virtue in a man as in a woman."}

These words are said by Joseph in his second letter to his sister Pamela. After telling her
that his Lady has fallen in love with him and that he remained virtuous, he remembers that
Parson Adams has often told him that chastity is as good in a man as in a woman.

\head{Chapter 11: Fanny Goodwill}

\ind Instead of going to his parents or Pamela, Joseph leaves London for 
Lady Booby's country seat where Fanny lives. She and Joseph has been in
love for many years, but have not married as Parson Adams advised them
to wait until they had sufficient money and experience to live 
comfortably. During a storm Joseph takes shelter in an inn. Another traveller offers
to let Joseph use his extra horse as they are going in the same direction.
  

\head{Chapter 12: The Good Samaritan}


\ind The two travellers reach an inn. Joseph continues his journey on foot. He
is attacked and beaten unconscious on the street by two robbers who take his clothes
and money. Passengers do not help him, some fearing they will also be
robbed, and others objecting because Joseph is naked. When a stage-coach\footnote{\, 
a four-wheeled public transport coach that are led by horses.} passes by,
the postillion\footnote{\, the person who rides the stage-coach, \textcolor{blue}
{\href{https://westervillelibrary.org/wp-content/uploads/sites/116/2022/01/Drawing-of-Stagecoach.png}
{see image.}}} helps Joseph by giving him a coat so that he can enter the 
coach. The stage-coach arrives at an inn where Betty -- the maid -- helps
Joseph by giving him a shirt and a bed. The owner of the inn, Mr. Tow-wouse,  and his wife 
are bothered by the charity of Betty.

\subhead{The Story of the Good Samaritan}

\ind Jesus' parable\footnote{\, a simple story that is used to teach a moral lesson.}
about the traveler who was stripped of his clothes and wounded by thieves, 
supposedly many religious Jews pass by the wounded man, who is finally helped by 
a Samaritan. This biblical allusion is used  when Joseph is robbed 
and helped by the postillion.

\subhead{Questions and Quotes}

\subsubhead{Q: How does Fielding make use of the parable of the Good Samaritan?}

Fielding uses the parable of the Good Samaritan when Joseph is attacked
and beaten unconscious on the street by two robbers who take his clothes and money.
None of the passengers help him. However, when a stage-coach passes by, a postillion helps Joseph by
giving him a coat so that he can enter the coach. The postillion represents the Good Samaritan from
the Bible.


\head{Chapter 13: The Hypocrisy of the Clergymen}

\ind At the inn, Joseph is visited by a doctor who speaks medical jargon and
tells him to make his will. Barnabas, the local clergyman, tells Joseph
that this world is carnal and he must place all his hopes of happiness in
Heaven. Instead Joseph's thoughts are on Fanny. The clergyman advises
Joseph that he should forgive his robbers as a Christian. When Joseph
asks him what that forgiveness means, Barnabas is unable to answer him and
instead tells him, “In short, it is to forgive them as a Christian."

\subhead{Bad Clergymen}

\ind In the novel we see two examples of bad clergymen. The first is \textbf{Mr. Barnabas},
who tells Joseph it is okay for 
him to hate and even to kill the men who robbed him if he had the chance, because that is 
allowable by law. He gives him  shallow advice when Joseph asks him about forgiveness, showing
his lack of understanding of Christianity. The second is \textbf{Mr. Trulliber}, who refuses to loan Parson Adams 
14 shillings despite being the richest man in his parish. Trulliber accuses Adams of trying to rob 
him and throws him out of his house after Parson Adams says he is no Christian. Trulliber rules his parishioners 
using fear rather than love.

\subhead{Grace verses Charity}

\ind In the novel, we see two kinds of characters: the first are
the believers in grace, who believe that salvation is achieved by
their belief and God's foreknowledge of their 'goodness'. The second
is the believers in charity, who believe that it is through good works
that people go to Heaven. Fielding opposes Richardson and the Methodists,
who believed in the superiority of grace to good works. Fielding
believes that Christianity must be expressed in charity towards
others. In \textit{Joseph Andrews}, we see charity mostly in the lower class,
especially in the character of Parson Adams.

\subhead{Question and Quotes}

\quotehead{“Whoever therefore is void of Charity, I make no scruple of pronouncing 
that he is no Christian.”}

These words are said by Parson Adams during his argument with Parson Trulliber over 
the true nature of Christianity. Adams believes that it is through charity and good works, not 
grace, that people can be saved.


\quotehead{“Forgive them as—as—it is to forgive them as—in short, it is to forgive 
them as a Christian”}

These words are said by the local clergyman  Mr. Barnabas during a conversation with Joseph.
After Joseph asks him what forgiveness is, Barnabas gives Joseph shallow advice, showing
his lack of knowledge of Christianity.


\head{Book 2 Chapter 1}

\ind The narrator explains the theory of his novel. Using a metaphor, he compares
a novel to a journey where the spaces between chapters are inns for resting, the blank
pages are stages in a journey, and chapter titles allow the reader to skip what he does
not like. This is called the 'art of dividing', which has the advantage of enabling 
the reader to easily access the novel.

\enlargethispage{\baselineskip}
\head{Book 3 Chapter 1}

\ind Fielding further explains his theory of novel where he contrasts biography 
with history. He explains that while good literature is based on real people,
it is based on the species (meaning universal human types), not on specific
individuals. \hl{His work is concerned with manners
that are found in all ages}, making it universal.
The goal is to correct the behavior of individuals belonging to a specific group.
Fielding explains the difference between satire and lampoon\footnote{\,
a piece of writing that is intended to defame a person.}, saying that his work 
is concerned with satire.

\subhead{Biography verses History}

\ind Fielding elevates biography over history. According to him, historians report what
happened in the past accurately, but they are careless in their evaluations of
persons. Biographers -- or novelists -- on the other hand, may mistake
facts, but grasp human truth. The biographer should not
invent his own material, but should copy nature. Art should be based
on what is generally true to all nations and ages, and not the particulars
of one age or country.


\subhead{Satire verses Lampoon}

\ind Lampoon is intended to defame a specific individual.
While satire, being general, does not insult an individual but seek to correct
behavior. To laugh at a well-known fault common to a group of people 
does not mean that everyone in the group is guilty. Fielding explains that he
is a satirist rather than a libeller\footnote{\, a writer who defames others.}



\head{From and Style of \textit{Joseph Andrews}}

\subhead{The Picaresque Novel and \textit{Don Quixote}}


\ind The picaresque novel originated in Spain in the mid-16th century as a first-person 
narrative that relates the adventures of a lovable rascal as he travels from place to place in an 
episodic structure. The picaro, is a foil to the medieval knight who wanders the 
countryside looking for opportunities to prove his courage and good character. In 
contrast, a picaro may lie, cheat, and steal to survive, come from a lower station in life, and serve 
to reveal the corruption of society through irony or satire. The best example of a picaresque novel
is \textit{Don Quixote} by Cervantes, who depicts the adventures of Don Quixote and his sidekick Sancho Panza.
\textit{Joseph Andrews} is sometimes called a picaresque novel because it follows the same pattern as \textit{Don Quixote}.

\subsubhead{The Manner of Cervantes}

\ind The subtitle of \textit{Joseph Andrews} reads, ”Written in imitation of the
manner of Cervantes, the author of Don Quixote.” In \textit{Joseph Andrews}
we see this imitation in the following: 

\begin{itemize}[itemsep=2pt]

\item The journey of Adams and  Joseph through the countryside is 
like the journey of Don Quixote and Sancho Panza through the country.
  
\item the journey of Adams and the eccentric, idealistic 
Parson Adams is somewhat like the eccentric, idealistic Quixote.

\item The society shown in both books is hypocritical, scheming and
proud. 

\item the technical features such as the chapter headings and
the digressions within the main story are similar in the two books.

\item Several incidents occur in both novels and there is similarity in
the structure and movement.

\end{itemize}

\subhead{Biblical Allusions and Parallels}

\ind Fielding uses biblical allusions and parallels\footnote{\, 
a person or thing that is similar to another.} in the novel
to show Christian virtues in the secular world. 
The most obvious biblical allusions are:\smallbreak

\ind The Good Samaritan, Jesus’ parable about the traveler who was 
stripped of his clothes and wounded by thieves,
supposedly many religious Jews pass by the wounded man, who
is finally helped by a
Samaritan. This biblical allusion is used when Joseph is 
robbed and helped by the postillion.\smallbreak

\ind In Book IV, Chapter 8, there is an ironic use of 
Genesis. Adams has reminded Joseph that Abraham's faith in
God required him to accept the divine command that his son Isaac
be a sacrifice. Immediately afterwards, a messenger announces that 
Adams' son has drowned. Instead of accepting fate, Parson Adams
laments. However, just as God saved Isaac, So Adams' son is found to 
be alive. This reveals Parson Adams giving way to human emotions instead
of demonstrating faith.\smallbreak

\ind The names of the major character are biblical. Joseph's life
and character recall the biblical Joseph, especially in his chastity
when faced by sexual advances. The name Abraham Adams combines
Adams and Abraham, two major figures in the Bible.\smallbreak

\subhead{Characters Names}


\ind Fielding uses names to create broad comedy in the novel as well as 
indicate character. For example, Joseph Andrews alludes to the Joseph of the Bible.
Joseph's identity is hidden in the Bible story, 
just as Joseph Andrews's true identity is hidden. Both Josephs resist the advances of predatory 
mistresses. Parson Abraham Adams has a name that recalls the biblical characters of 
Abraham and Adam. Adam is the father of mankind, as the parson is the father of his 
parishioners. The parson attempts to mimic Abraham in his obedience. Other characters have 
names that represent their flaws: for example, Lady Booby (meaning Tempting), Mrs. Slipslop 
(meaning Morally Lax, Sloppy or Careless), Constable Suckbribe (meaning Corrupt), and the 
like. 


\subhead{The use of burlesque and grotesque}

\ind In his preface, Fielding explains that he will only use burlesque at the level 
of diction and will not use it in any other way because it deforms human nature.
However, in chapter 6, we see Fielding using burlesque to describe the character
of Mrs. Slipslop. Later in the chapter he describes her in a grotesque way by
comparing her to a hungry tigress. 


\end{document}
