\documentclass[12pt, a4paper]{article}
\usepackage{geometry}
\usepackage{lua-ul}
\usepackage{graphicx}
\usepackage{enumitem}
\usepackage{color,soul}
\usepackage[fonthead={Goudy Oldstyle Std},
fontsub={Goudy Oldstyle Std}]{defaultpreamble}


\setmainfont{Bely}


\begin{document}


\newgeometry{top=0.6in}

\enlargethispage{\baselineskip}
\titlehead{Joseph Andrews}{Henry Fielding}


\head{Preface}

\ind \textbf{The history of the adventures of Joseph Andrews and of his friend Mr. Abraham Adams} is the 
title of Henry Fielding's novel, which is a parody critiquing Pamela by Samuel Richardson.
Fielding's previous novel \hl{--Shamela-- was a parody} and a direct response to Richardson's Pamela.
The subtitle reads,\textbf{"Written in imitation of the manner of Cervantes,
the author of Don Quixote"}, which explains the main influence of the novel. \hl{Don Quixote is a picaresque
novel.} 

\subhead{Henry Theory of Novel}

\ind In his preface, Fielding explains that his work is different from the well known
literary forms of his time. Joseph Andrews is a \hl{comic romance}, a \hl{comic epic
poem in prose}. It has the length of the epic, but differs from serious romance
in being light and dealing with the ridiculous. Although the style sometimes 
includes burlesque  imitation, the novel is not a burlesque, as the characters are
based on those found in real life. Fielding derives \hl{his influence from Homer's\footnote{
\, Greek poet.} lost comic epic\footnote{\, called Margites}}. He explains that his work deals with inferior subjects
in an elevated style.

\subhead{Burlesque}

\ind Burlesque or parody is the use of elegant style to present inferior topics and ideas. 
Fielding explains that he only uses burlesque in the level of \underLine{diction}
and does not use it in any other way because \hl{burlesque deforms human nature}. He 
distinguishes burlesque from comic, saying that comic deals with the inferior while
burlesque exhibits monsters. He
explains the similarity between Carictura\footnote{\, paintings that aim to
exaggerate human features to show monsters.} and burlesque saying,
"What Carictura in painting \underline{burlesque} is in writing". Fielding \hl{uses
burlesque to describe the character of Mrs. Slipslop.}


\subhead{Affectation and the Ridiculous}

\ind Fielding says that his work will focus on the ridiculous rather than the sublime.
He explains that the ridiculous arises from affectation, and 
affectation itself arises either from vanity or hypocrisy.
He considers hypocrisy to be a much worse vice and more ridiculous than vanity.
\hl{Fielding admires Ben Jonson the most} because Ben Jonson understood the ridiculous the best.

\subsubhead{Vanity}

\ind A vain man exaggerates his virtues. In the novel, vanity is shown the character of Parson Adams who
believes his learning makes him wiser than  others. Nevertheless, his vanity is fairly 
harmless when compared with other characters.


\restoregeometry

\subsubhead{Hypocrisy}

\ind A hypocrite person hides his vices under an appearance of their opposite virtues. 
In the novel, hypocrisy is shown in 
Lady Booby and Mrs. Slipslop, who pretend to be chaste while pursuing Joseph Andrews.
They continually make themselves ridiculous because of their hypocrisy.

\subhead{Questions and Quotes}

\quotehead{“The only Source of the true Ridiculous (as it appears to me) is Affectation.”}

These words are said by Fielding in his preface. He rejects burlesque because it shows monsters,
and seeks out comedy because it shows the forms of absurdity that exist in real life. 
To him the true ridiculous arises from the exposure of affectation.


\head{Chapter 1: Of Writing Lives in General}

\ind In his theory of fiction, Fielding points out the importance of biography\footnote{
\, writing about someone's life.}, comparing his work with other contemporary biographies.
He explains that a biography can be more useful to mankind than the great person whose
life is recorded, and that the reader is improved by a mixture of instruction and delight.
He ironically mentions \textbf{Colley Cibber's} autobiography\footnote{\, self-written account of one's own life.}
and \textbf{Samuel Richardson's} Pamela\footnote{\, a biography which talks about female chastity.},
saying that his Joseph Andrews is the brother of Pamela and
that he is an example of \hl{male chastity}.

\subhead{Questions and Quotes}

\quotehead{“It is a trite but true Observation, that Examples work more forcibly on the
Mind than Precepts.”}

\headfoot{Chapter 2: Joseph's Genealogy}{\, genealogy is the study of a family tree.}

\ind The narrator says that Joseph is the brother of Pamela and the only son of \textbf{Gaffer} and \textbf{Gammer},
names that are used to refer to old people which implies that \hl{Joseph comes from low birth}.
He explains that there were many searches for Joseph’s parentage, with little success. He mentions
a \hl{great-grandfather who was a cudgel-player} and an epitaph that mentions \hl{merry man Andrew 
who was part of a sect called the laughing philosophers (Merry Andrews)}. He compares Joseph 
to the Athenians who sprang up from a dunghill (\hl{autokopros}\footnote{\, sprung from a dungill}).

\subhead{Joseph's Jobs}

\ind At ten years old, Joseph became an apprentice to Sir Thomas Booby. The boy was employed in what 
was called \hl{keeping birds}. Later, he was moved to a dog-kennel\footnote{\, a shelter for dogs.
} where he worked as a \hl{whipper-in}\footnote{
\, a huntsman's assistant.}. He then worked in the stable where he rode in races. At age seventeen,
he became a \hl{footboy}\footnote{\, personal servant.} to Lady Booby.

\subhead{Questions and Quotes}

\quotehead{“Mr. Joseph Andrews, the Hero of our ensuing History, was esteemed to be
the only Son of Gaffar and Gammer Andrews, and Brother to the illustrious
Pamela, whose Virtue is at present so famous.”}

\subsubhead{Q: Write down an essay about Joseph Andrews' mock-heroic
genealogy.}



\ind In Joseph Andrews, Fielding uses the mock-heroic style to describe Joseph's
family tree. In epic convention, heroes descend from gods, famous 
kings and other grand ancestors. However, the novel reverses this tradition: 
Joseph has no known ancestors; Fielding mentions only two based on a hearsay.
One who was a cudgel-player, and another who belonged to a sect named the Merry Andrews 
(laughing philosophers). Fielding goes further 
by mentioning that Joseph is the only son of Gaffer and Gammer, names that are 
used to refer to old people, which suggests his humble origin.



\head{Chapter 3: Characters Introduction}

\ind In this chapter we are introduced to the main characters: Parson\footnote{\, 
member of the clergy; a priest.}  Abraham Adams,
Mrs. Slipslop, Sir Thomas Booby and his wife Lady Booby. Parson Adams is a scholar from 
the country who wants to \hl{teach Joseph Latin}.
Sir Thomas Booby rarely sees Parson Adams; he values people only according to their
wealth and appearance, and Lady Booby considers country people brutes. This shows the snobbery\footnote{
\,a snob is a person who has a strong sense of class status.}of
the Booby family. Mrs. Slipslop the waiting-gentlewoman, has respect for Adams  but thinks she
is better than him only because she went to London. \hl{Mrs. Slipslop is described as a
\textit{"Mighty affecter of hard words"}}
because she uses words she does not understand only to show her superiority,
this reveals her vanity.


\subhead{Abraham Adams}

Is an excellent scholar who knew Greek, Latin, French, Italian and Spanish.
He spent many years in learning, and is a man of good nature, but at the 
same time as entirely \hl{ignorant of the ways of this world as an infant just
entered to it}. He was generous, friendly and brave but \hl{simplicity} was his 
his characteristic. At the age of fifty he worked as bishop with good income,
however, it was not enough to live well with his wife and six children. \hl{His name
is a biblical allusion} to the character of Abraham and Adam. \hl{Parson Adams is similar to 
Don Quixote} for the both are idealistic and eccentric.


\subhead{Questions and Quotes}

\subsubhead{Q: To which character does the term Quixotic apply in Joseph
Andrews, and why?}

To Abraham Adams because he and Done Quixote are idealistic and eccentric.

\head{Chapter 4: Joseph in London}

\ind Joseph makes some friends in London who teach him how to dress and become 
fashionable. However, they could not teach him to game, swear, drink or
any other vice of the town. He becomes a \hl{connoisseur\footnote{\, 
expert in matters of tastes.}} in music. Lady Booby, who always thought that 
he lacked spirit, begins to change her mind after seeing the effect of the town
on Joseph, saying, "Aye, there is life in this fellow." She started taking walks
with him in Hyde Park and getting close to him. One morning, \hl{Lady Tittle and Lady
Tattle\footnote{\, both are tag-names that means gossip.}} saw them together and
gossiped that Lady Booby was in love with Joseph.



  
\end{document}
