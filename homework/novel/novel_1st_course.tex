\documentclass[12pt, a4paper]{article}
\usepackage{geometry}
\usepackage{lua-ul}
\usepackage{graphicx}
\usepackage{enumitem}
\usepackage{color,soul}
\usepackage[fonthead={Goudy Oldstyle Std},
fontsub={Goudy Oldstyle Std}]{defaultpreamble}


\setmainfont{Bely}


\begin{document}


\newgeometry{top=0.6in}

\enlargethispage{\baselineskip}
\titlehead{Joseph Andrews}{Henry Fielding}


\head{Preface}

\ind \textbf{The history of the adventures of Joseph Andrews and of his friend Mr. Abraham Adams} is the 
title of Henry Fielding's novel, which is a parody critiquing Pamela by Samuel Richardson.
Fielding's previous novel \hl{--Shamela-- was a parody} and a direct response to Richardson's Pamela.
The subtitle of Joseph Andrews reads,\textbf{"Written in imitation of the manner of Cervantes,
the author of Don Quixote,"} which explains the main influence of the novel. \hl{Don Quixote is a picaresque
novel.} 

\subhead{Henry Theory of Novel}

\ind In his preface, Fielding explains that his work is different from the well known
literary forms of his time. Joseph Andrews is a \hl{comic romance}, a \hl{comic epic
poem in prose}. It has the length of the epic, but differs from serious romance
in being light and dealing with the ridiculous. Although the style sometimes 
includes burlesque  imitation, the novel is not a burlesque, as the characters are
based on those found in real life. Fielding derives \hl{his influence from Homer's\footnote{
\, Greek poet.} lost comic epic\footnote{\, called Margites}}. He explains that his work deals with inferior subjects
in an elevated style.

\subhead{Burlesque}

\ind Burlesque or parody is the use of elegant style to present inferior topics and ideas. 
Fielding explains that he only uses burlesque in the level of \underLine{diction}
and does not use it in any other way because \hl{burlesque deforms human nature}. He 
distinguishes burlesque from comic, saying that comic deals with the inferior while
burlesque exhibits monsters. He
explains the similarity between Carictura\footnote{\, paintings that aim to
exaggerate human features to show monsters.} and burlesque saying,
"What Carictura in painting \underline{burlesque} is in writing". Fielding \hl{uses
burlesque to describe the character of Mrs. Slipslop.}


\subhead{Affectation and the Ridiculous}

\ind Fielding says that his work will focus on the ridiculous rather than the sublime.
He explains that the ridiculous arises from affectation, and 
affectation itself arises either from vanity or hypocrisy.
He considers hypocrisy to be a much worse vice and more ridiculous than vanity.
\hl{Fielding admires Ben Jonson the most} because Ben Jonson understood the ridiculous the best.

\subsubhead{Vanity}

\ind A vain man exaggerates his virtues. In the novel, vanity is shown the character of Parson Adams who
believes his learning makes him wiser than  others. Nevertheless, his vanity is fairly 
harmless when compared with other characters.


\restoregeometry

\subsubhead{Hypocrisy}

\ind A hypocrite person hides his vices under an appearance of their opposite virtues. 
In the novel, hypocrisy is shown in the characters of both
Lady Booby and Mrs. Slipslop, who pretend to be chaste while pursuing Joseph Andrews.
They continually make themselves ridiculous because of their hypocrisy.

\subhead{Questions and Quotes}

\quotehead{“The only Source of the true Ridiculous (as it appears to me) is Affectation.”}

These words are said by Fielding in his preface. He rejects burlesque because it shows monsters,
and seeks out comedy because it shows the forms of absurdity that exist in real life. 
To him the true ridiculous arises from the exposure of affectation.

\subsubhead{Q: What is the full title of the novel Joseph Andrews?}

The history of the adventures of Joseph Andrews and of his friend Mr. Abraham Adams.

\subsubhead{Q: Why does Fielding satirize the character of Pamela in
Richardson's novel?}

Fielding believed that Richardson's Pamela sets a bad moral
example, turning sex into a commodity, and he found the heroine to be calculating and
manipulative.


\subsubhead{Q: What is affectation, and how is it exposed according to Fielding?
Elaborate with examples.}

Fielding explains that the ridiculous arises from affectation, and 
affectation itself arises either from vanity or hypocrisy.
He considers hypocrisy to be a much worse vice and more ridiculous than vanity.
He exemplifies affectation in the character of Parson Adams, whose vanity makes him 
believe that his learning makes him wiser than others, and in the character of Lady
Booby, whose hypocrisy makes her look ridiculous.

\subsubhead{Q: In his theory of novel, Fielding states that he is attempting a new species of writing. Explain.}

Answer page 1 under \textit{"Henry Theory of Novel"}.



\subsubhead{Q: Why does Fielding define Joseph Andrews as a "comic epic-poem
in prose"? Discuss}

Fielding explains that Joseph Andrews has
the length of an epic but differs from the serious romance by focusing on the ridiculous rather than the sublime.
He also explains that his work lacks the meter found in all epics.

\subsubhead{Q: What are the characteristic of Fielding's theory 
of the Ridiculous?}

Fielding explains that the ridiculous arises from affectation, and 
affectation itself arises either from vanity or hypocrisy. He considers hypocrisy to be a 
much worse vice than vanity, because it is more surprising and more ridiculous 
when we discover that someone is the exact opposite of what they claims to be.



\subsubhead{Q: Discuss the theme of hypocrisy in Joseph Andrews.}

Answer page 2 under \textit{"Hypocrisy"}.


\head{Chapter 1: Of Writing Lives in General}

\ind In his theory of fiction, Fielding points out the importance of biography\footnote{
\, writing about someone's life.}, comparing his work with other contemporary works.
He explains that a biography can be more useful to mankind than the great person's
life it records, and that the reader is improved by a mixture of instruction and delight.
He ironically mentions \textbf{Colley Cibber's} autobiography\footnote{\, self-written account of one's own life.}
and \textbf{Samuel Richardson's} Pamela\footnote{\, a biography which talks about female chastity.},
saying that his Joseph Andrews is the brother of Pamela and
that he is an example of \hl{male chastity}.

\subhead{Questions and Quotes}

\quotehead{“It is a trite but true Observation, that Examples work more forcibly on the
Mind than Precepts.”}

These words are said by Fielding explaining the moral utility of the novel:
it can embody virtue in the biographies of
exemplary characters, thereby inspiring our imitation. He ironically cites two contemporary works that 
that inspired imitation of virtue: Richardson’s Pamela and Cibber’s autobiography.

\subsubhead{Q: In his theory of fiction, Fielding points out the importance of Biography and “writing lives in
general”, comparing his work with other contemporary biographies. Explain.}

Fielding explains that a biography can be more
useful to mankind than the great person's life it records, and that the reader is
improved by a mixture of instruction and delight. He ironically cites two contemporary works: 
Colley Cibber’s autobiography and Samuel Richardson’s Pamela.

\headfoot{Chapter 2: Joseph's Genealogy}{\, genealogy is the study of a family tree.}

\ind The narrator says that Joseph is the brother of Pamela and the only son of \textbf{Gaffer} and \textbf{Gammer},
terms of respect for older people of low social rank, which implies that \hl{Joseph comes from low birth}.
He explains that there were many searches for Joseph’s parentage, with little success. He mentions
a \hl{great-grandfather who was a cudgel-player} and an epitaph that mentions \hl{merry man Andrew 
who was part of a sect called the laughing philosophers (Merry Andrews)}. He compares Joseph 
to the Athenians who sprang up from a dunghill (\hl{autokopros}\footnote{\, sprung from a dungill}).

\subhead{Joseph's Jobs}

\ind At ten years old, Joseph became an apprentice to Sir Thomas Booby. The boy was employed in what 
was called \hl{keeping birds}. Later, he was moved to a dog-kennel\footnote{\, a shelter for dogs.
} where he worked as a \hl{whipper-in}\footnote{
\, a huntsman's assistant.}. He then worked in the stable where he rode in races. At age seventeen,
he became a \hl{footboy}\footnote{\, personal servant.} to Lady Booby.

\subhead{Questions and Quotes}

\quotehead{“Mr. Joseph Andrews, the Hero of our ensuing History, was esteemed to be
the only Son of Gaffar and Gammer Andrews, and Brother to the illustrious
Pamela, whose Virtue is at present so famous.”}

These words are said by Fielding while introducing his title character.
He makes a connection between his hero and Richardson's heroine --Pamela.
By saying that Joseph is the only son of Gaffer and Gammer, terms of respect for
older people of low social rank, Fielding emphasizes the hero's low birth
and signals the satirical, comical nature of his novel.


\subsubhead{Q: Why is Joseph's character considered mock-heroic?}

Because of his unknown and low birth. This is opposed to the epic convention,
where heroes descend from gods, kings, and noble ancestors.

\subsubhead{Q: Why does Joseph fail in his job as a bird-keeper?}

Because his voice was so musical that it allured the birds rather than terrify them.

\subsubhead{Q: What is the significance of Joseph's parents' names?}

They are terms of respect for older people of low social rank.

\subsubhead{Q: Who are the laughing philosophers?}

Is a sect found by one of Joseph's ancestor.

\subsubhead{Q: Write down an essay about Joseph Andrews' mock-heroic
genealogy.}

\ind In Joseph Andrews, Fielding uses the mock-heroic style to describe Joseph's
family tree. In epic convention, heroes descend from gods, famous 
kings and other grand ancestors. However, the novel reverses this tradition: 
Joseph has no known ancestors; Fielding mentions only two based on a hearsay.
One who was a cudgel-player, and another who belonged to a sect named the Merry Andrews 
(laughing philosophers). Fielding goes further 
by mentioning that Joseph is the only son of Gaffer and Gammer,
terms of respect for older people of low social rank, to 
emphasize his low birth.


\head{Chapter 3: Characters Introduction}

\ind In this chapter we are introduced to the main characters: Parson\footnote{\, 
member of the clergy; a priest.}  Abraham Adams,
Mrs. Slipslop, Sir Thomas Booby and his wife Lady Booby. Parson Adams is a scholar from 
the country who wants to \hl{teach Joseph Latin}.
Sir Thomas Booby rarely sees Parson Adams; he values people only according to their
wealth and appearance, and Lady Booby considers country people brutes. This shows the snobbery\footnote{
\, a snob is a person who has a strong sense of class status.}of
the Booby family. Mrs. Slipslop the waiting-gentlewoman, has respect for Adams  but thinks she
is better than him only because she went to London. \hl{Mrs. Slipslop is described as a
\textit{"Mighty affecter of hard words"}}
because she uses words she does not understand only to show her superiority,
this reveals her vanity.


\subhead{Abraham Adams}

Is an excellent scholar who knew Greek, Latin, French, Italian and Spanish.
He spent many years in learning, and is a man of good nature, but at the 
same time as entirely \hl{ignorant of the ways of this world as an infant just
entered to it}. He was generous, friendly and brave but \hl{simplicity} was 
his characteristic. At the age of fifty he worked as bishop with good income,
however, it was not enough to live well with his wife and six children. \hl{His name
is a biblical allusion} to the character of Abraham and Adam. \hl{Parson Adams is similar to 
Don Quixote} for the both are idealistic and eccentric.


\subhead{Questions and Quotes}

\quotehead{"He was generous, friendly and brave to an Excess; but
Simplicity was his Characteristic."}


These words are said by the narrator describing the character of Parson Adams.
He is described as generous and friendly but excessively brave, which is the result of
his lack of prudence. He is also described as naive and simple.

\quotehead{"She had in these disputes a [particular advantage]\footnote{\, this is a verbal irony.}over Adams for she was A
mighty affecter of hard words"}

These words are said by the narrator about Mrs. Slipslop. He says that when Parson Adams and 
Mrs. Slipslop argue, she uses jargon to show her superiority.

\subsubhead{Q: To which character does the term Quixotic apply in Joseph
Andrews, and why?}

To Abraham Adams because he and Don Quixote are idealistic and eccentric.

\subsubhead{Q: Why is Mrs. Slipslop described as "a mighty affector of hard
words"?}

Because she uses jargon to win arguments with Parson Adams.


\subsubhead{Q: How do Mr. Thomas Booby and Lady Booby act in a snobbish
manner?}

Sir Thomas Booby rarely sees Parson Adams; he values people only according to their wealth and appearance, and Lady Booby
considers country people brutes. This shows the snobbery the Booby family.


\subsubhead{Q: Why is parson Adams described as being naive?}

Because he is ignorant of the ways of this world as an infant just entered to it.
\head{Chapter 4: Joseph in London}

\enlargethispage{2\baselineskip}

\ind Joseph makes some friends in London who teach him how to dress and become 
fashionable. However, they could not teach him to game, swear, drink or
any other vice of the town. He becomes a \hl{connoisseur\footnote{\, 
expert in matters of tastes.}} in music. Lady Booby, who always thought that 
he lacked spirit, begins to change her mind after seeing the effect of the town
on Joseph, saying, "Aye, there is life in this fellow." She started taking walks
with him in Hyde Park and getting close to him. One morning, \hl{Lady Tittle and Lady
Tattle\footnote{\, both are tag-names that means gossip.}} saw them together and
gossiped that Lady Booby was in love with Joseph.

\subhead{Question and Quotes}

\quotehead{"They could not, however, teach him to game, swear, drink, nor any other
genteel vice the town bounded with"}

These words are said by the narrator about Joseph's friends when he was in London.
They could not teach Joseph the vices of the city, which proves Joseph's virtue.

\head{Chapter 5: The Seduction}

\ind After the death of Sir Thomas Booby, Lady Booby pretends to mourn, while in fact she was
playing cards with her friends for six days. And on the seventh day she calls Joseph to her room,
this is a \hl{Biblical allusion to the creation of the world in seven days}\footnote{\, in the Bible, God has
made the world in six days and rested on the seventh.}. She calls him
a philanderer\footnote{\, general lover.} and tries to seduce him but fails. She tempts
him and assures him that he need not be afraid of their different class. However, Joseph shows himself 
to be chaste and reject her, Lady Booby then kicks him out of the room.
The \hl{seduction of Lady Booby to Joseph Andrews is a Biblical allusion to the advances
of Potiphar's wife to Joseph}.

\subhead{Question and Quotes}

\quotehead{"Then you are either a fool, or pretend to be so; I find I was
mistaken in you."}

These words are said by Lady Booby after Joseph rejects her advances over him.
She feels enraged and finds him to be either stupid or naive to reject her.


\subsubhead{Q: Why is Joseph Andrews accused of being a philander?}

Because Joseph tells Lady Booby that all the women he had ever seen were
equally indifferent to him.

\head{Chapter 6: Letter to Pamela}

\ind Joseph writes a letter to his sister Pamela complaining to her that Lady Booby tried to seduce him.
This letter style of writing a novel is called \hl{epistolary technique\footnote{\, the epistolary technique is used by the writer 
to reveal the character's inner workings and mind. Epistle means letter.} which is used in Richardson's
Pamela}. After Joseph writes the letter he is approached by Mrs. Slipslop who tries to make
love to him violently, but Joseph is saved by the ringing of Lady Booby's bell. \hl{Mrs. Slipslop
is described in burlesque} manner and \hl{compared to a hungry tigress in a grotesque} way by the narrator. 

\subhead{Mrs. Slipslop}

\ind Mrs. Slipslop, the waiting-gentlewoman of Lady Booby, is described in a burlesque way as an old and ugly woman.
Like her mistress -- Lady Booby --
she feels lustful toward Joseph. Although single,
the narrator indicates that she is not a virgin. Her name, which implies she is morally lax, sloppy or careless, also
suggests that she made a "slip" in her past.
The narrator grotesquely compares her to a tiger when she tries to make a sexual advance on Joseph.


\subhead{Question and Quotes}

\quotehead{"As when a hungry tigress, who long has traversed the woods in
fruitless search, sees within the reach of her claws a lamb, she
prepares leap on her prey."}

These words are said by the narrator when Mrs. Slipslop tried to make an advance on Joseph. This is an instance
of epic simile, where the narrator grotesquely compares Slipslop's sexual advance on Joseph to a hungry tiger.

\quotehead{“She resolved to give a loose to her amorous inclinations, and to pay off the 
debt of pleasure which she found she owed herself.”}

These words are said by the narrator about Mrs. Slipslop. He explains that 
having already made amends for past and future mistakes, she is allowed the liberty to be with any man
and to indulge herself in pleasure.


\subsubhead{Q: How does Fielding make use of Grotesque in the novel?}

When he compares Mrs. Slopslop's sexual advance on Joseph to a hungry tiger.

\subsubhead{Q: To whom does Joseph write letters, and why?}

To his sister Pamela. To complain to her about Lady Booby's passion.

\subsubhead{Q: Give a character sketch of Mrs. Slipslop.}

Answer on page 8 under \textit{"Mrs.slipslop"}.

\head{Chapter 7: Lady Booby's Passion}

\ind Feeling rejected, Lady Booby wants to rid herself of Joseph. She calls
Mrs Slipslop to her room. The two disappointed women discuss the
young man. Mrs \hl{Slipslop falsely claims that Joseph is a womanizer}\footnote{\,
a male who has casual sex with several women.}. Lady Booby is divide against herself:
she wants to kick Joseph out, and then she wants to keep him nearby. It seems that Lady 
Booby has been hit by one of \hl{Cupid's\footnote{\, in Greek mythology, Cupid meaning passionate desire,
is the god of desire, erotic love and attraction.} arrows}. Lady Booby finally decides 
that she will insult then dismiss him.

\head{Chapter 8: Man's Virtue}

\ind The narrator describes Joseph's charms to make the 
reader understand and excuse Lady Booby. Joseph is interviewed by Lady Booby about his 
supposed misconduct with the maids of the house.
She further tries to seduce him but he tells her that he is virtuous. She does not believe that
a man can be virtuous, but he answers her that he is the brother of Pamela. She becomes outraged and
kicks Joseph out of the house. 

\subhead{Question and Quotes}

\quotehead{"Did ever mortal hear of a man's virtue? Did ever the greatest
or the gravest men pretend to any of this kind?"}

These words are said by Lady Booby to Joseph after she tries to seduce him. When Joseph
rejects her advances and declares himself a virtuous man, she is shocked because
she does not believe that a man can be virtuous and thinks that he is a hypocrite.

\head{Chapter 9: Lady Booby and Slipslop's Hypocrisy}

\ind Mrs. Slipslop, having listened through the keyhole to the conversation between 
Lady Booby and Joseph, is no longer afraid of her mistress -- Lady Booby -- and freely mocks her.
Lady Booby becomes worried about her reputation but thinks that she could bribe Mrs. Slipslop
into secrecy. What had hurt her the most was that she still had feelings for Joseph. 


\head{Chapter 10: Joseph Hits the Streets}

\ind Joseph writes a second letter to Pamela, complaining that Lady Booby has fallen in
love with him. In the letter he says that Parson Adams has told him that chastity 
is as great a virtue in a man as in a woman and promises his sister Pamela that he will
imitate her chastity. After receiving his remaining wages from Peter Pounce\footnote{\,
the steward of Lady Booby, he grows rich by robbing the servants of their salaries.}, Joseph ends up in
the street. From now on, \hl{Joseph is going to lead a picaresque life}, moving from inn\footnote{\,
similar to hotels but smaller.} to inn.


\subhead{Question and Quotes}

\quotehead{"Mr. Adams hath often told me that chastity is as great a
virtue in a man as in a woman."}

These words are said by Joseph in his second letter to his sister Pamela. After telling her
that his Lady has fallen in love with him and that he remained virtuous, he remembers that
Parson Adams has often told him that chastity is as good in a man as in a woman.

\head{Chapter 11: Fanny}

Instead of going to his parents or Pamela, Joseph leaves London for 
Lady Booby's country seat where Fanny lives. She and Joseph has been in
love for many years, but have not married as Parson Adams advised them
to wait until they had sufficient money and experience to live 
comfortably. During a storm Joseph takes shelter in an inn. Another traveller offers
to let Joseph use his extra horse as they are going in the same direction.
  

\head{Chapter 12: The Good Samaritan}


The two travellers reach an inn. Joseph continues his journey on foot. He
is attacked and beaten unconscious on the street by two robbers who take his clothes
and money. Passengers do not help him, some fearing they will also be
robbed, and others objecting because Joseph is naked. When a stage-coach\footnote{\, 
a four-wheeled public transport coach that are led by horses.} passes by,
the postillion\footnote{\, the person who rides the stage-coach, \textcolor{blue}
{\href{https://westervillelibrary.org/wp-content/uploads/sites/116/2022/01/Drawing-of-Stagecoach.png}
{see image.}}} helps Joseph by giving him a coat so that he can enter the 
coach. The stage-coach arrives at an inn where Betty -- the maid -- helps
Joseph by giving him a shirt and a bed. The owner of the inn, Mr. Tow-wouse,  and his wife 
are bothered by the charity of Betty.

\head{Chapter 13: The Hypocrisy of the Clergyman}

\ind At the inn, Joseph is visited by a doctor who speaks medical jargon and
tells him to make his will. Barnabas, the local clergyman, tells Joseph
that this world is carnal and he must place all his hopes of happiness in
Heaven. Instead Joseph's thoughts are on Fanny. The clergyman advises
Joseph that he should forgive his robbers as a Christian. When Joseph
asks him what that forgiveness means, Barnabas is unable to answer him and
instead tells him, “In short, it is to forgive them as a Christian."




\end{document}
