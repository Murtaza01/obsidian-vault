\documentclass[12pt, a4paper]{article}
\usepackage{polyglossia}
\usepackage{fontenc}[T1]
\usepackage{geometry}

\geometry{top=0.6in}


\newfontfamily\englishfont{Recia}
\setmainfont{Recia}
\newfontfamily\fontpar{Synonym}

\newcommand{\exarrow}{{\Large{$\Rightarrow$}\,}}

\begin{document}
\begin{center}
\LARGE{Grammar Second Course}\\[0.5cm]
\Large{By: Birdman}\\[1cm]
\end{center}

\section*{\centering{\LARGE{Reporting\\[0.5cm]}}}

\subsection*{Converting to Indirect}
{\fontpar
When we convert from direct to indirect we need to follow 
few things:

\begin{enumerate}
\item We keep the expression "she said, he says, they asked..." untouched.
\item We change the pronoun, example:

  She said "i did my homework." 

 \exarrow She said \textbf{she} did \textbf{her} homework.

\item We make the tense past, if its already past we can either
  leave it or make it past perfect, both are right, example:

  She said "Its a strange story?"\\
\exarrow  she said It \textbf{was} a strange story.
\item We pay attention to time expression, because we do not have 
  future with past, example:

  "It rained yesterday and last week"

\exarrow He said that it rained the \textbf{day before} and the 
  \textbf{previous week}.

Because it might have been two days after or week after and 
im reporting his speech.

\item If we have a question that is started by an auxiliary we
  use if/whether in our indirect speech, example:

 "Is she a doctor or a nurse?"

\exarrow She asked \textbf{whether/if} she was a doctor 
  or a nurse

\item If the sentence is continous (expresses an on going event,
  or a current event, or something that is generally true) we keep the
  tense, example:

  "I go to the gym everyday"

  \exarrow He said he goes to the gym everyday.

  "Im going to study tomorrow"

  \exarrow She said she is going to study the day after
\end{enumerate}
}

 
\end{document}
