\documentclass[12pt, a4paper]{memoir}
\usepackage{polyglossia}
\usepackage{geometry}

\geometry{top=0.6in}
\setlength\parindent{0pt}

\newfontfamily\englishfont{Philosopher}
\setmainfont{Philosopher}
\newfontfamily\fontpar{BespokeSerif}

\begin{document}


\begin{center}
\huge{Phonetics Second Course}\\[0.5cm]
\Large{By: Birdman}\\[1cm]
\end{center}


\section*{\centering{\LARGE{The Syllable}\\[0.6cm]}}

{\fontpar
\textbf{Syllable}: Is a unit consisting of one vowel which may be 
proceeded or followed by one or more consonant.\bigbreak

Before we understand syllables, we need to understand the 
difference between vowels and consonants. \textbf{Vowels} produce none to 
little obstruction in the air flow. \textbf{Consonants} produce 
complete or partial obstruction in the air flow.\bigbreak

We also need to understand the difference between phonetics and phonology.
\textbf{Phonetics} studies the production of the speech sound, (whether
there is obstruction or not). \textbf{Phonology} studies the distribution
of the phonemes in a language. That's why syllables can be divided into
phonetic and phonology.

\section*{Syllable Phonetically}
Consisting of a centre which has little or no obstruction to the 
airflow.\medbreak

So phonetically we divide words by the amount of obstruction they have 
(how many consonant they have) and where they are in a word:


\begin{description}
  \item[1. Minimum syllable:]Is a single vowel in isolation, which has
    the least amount of obstruction:\medbreak

    \centerline{are /a:/ \hspace{2cm} err /3:/ }

  \item[2. Onset:]One or more consonant proceeding the centre of
    the syllable:\medbreak

    \centerline{bar /ba:/ \hspace{2cm} key /ki:/}

  \item[3. Coda:]One or more consonant comes after the centre of the 
    syllable:\medbreak

    \centerline{am /æm/ \hspace{2cm} ease /i:z/}

  \item[4. Onset and Code:]Syllable that has proceeding and following
    consonant which has the most obstruction:\medbreak

    \centerline{ran /ræn/ \hspace{2cm} sat /sæt/}
  
\end{description}

\section*{Syllable Phonologically}

Can be described as the possible combinations 
of English phonemes.\medbreak
\textbf{Phonotactics} is the study of the possible phoneme combinations
of a language.\bigbreak

\subsection*{Consonant Cluster}
Can be described as two or more consonant come together without a vowel
in between them.\medbreak

We divide consonant clusters (or sequences) based on their position:\medbreak

\textbf{Initial position}: In initial position a word can start with one,
two or three consonant, \underline{3 consonant sequence words only start 
with \textbf{/s/}} ex: \underline{Spr}ay , \underline{Str}ing.
\underline{There is no word that start with 
4 consonant}.\medbreak 

\textbf{Zero Onset} is when a word doesn't start with consonant 
(are /a:/).\bigbreak

\textbf{Final Position}: In final position a word can start with one,
two, three or four consonant, for example the word (pro\underline{mpts})
ends with 4 consonants \underline{no words ends with 5 consonant}.\medbreak

\textbf{Zero Coda} is When a word doesn't end with consonant.
}
\end{document}


