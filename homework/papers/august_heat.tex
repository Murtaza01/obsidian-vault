\documentclass[12pt, a4paper]{article}
\usepackage{polyglossia}
\usepackage{geometry}
\usepackage{lua-ul}
\usepackage{color,soul}
\usepackage{xcolor}
\usepackage{hyperref}

\geometry{bottom=1in, top=1in}
\setlength\parindent{0pt}

%fonts
\setmainfont{Alegreya}

\setotherlanguage{arabic}
\newfontfamily\arabicfont[Script=Arabic, Scale=1]{ِFustat}
\newfontfamily\arabichead[Script=Arabic, Scale=1]{Cairo}


%sections
\newcommand{\head}[1]{
  \phantomsection
  \section*{\centering{\arabichead{#1}}}
  \addcontentsline{toc}{section}{#1}
}


\newcommand{\subhead}[1]{
  \phantomsection
  \subsection*{#1}
  \addcontentsline{toc}{subsection}{#1}
}

\newcommand{\subsubhead}[1]{
  \phantomsection
  \subsubsection*{#1}
  \addcontentsline{toc}{subsubsection}{#1}
}


\begin{document}



\begin{otherlanguage}{arabic}

\head{حرارة اوغسطس}

{\arabichead{\centerline{بقلم ويليم فراير هارفي (0191)}}\bigbreak

طريق فينيستون، كلافام \bigbreak

العشرين من اغسطس 0091
}\bigbreak

اظن انني مررتُ باكثر يوم استثنائي في حياتي, وبينما الاحداث لاتزال حية في ذاكرتي, اردت ان اضعهم على الورق بكل وضوح. \medbreak

دعوني ابد بالقول ان اسمي جيمس كلارنس ويذرنكروفت. ابلغ اربعين من العمر, بكامل صحتي, لم اعرف قط المرض. 
بسياق العمل انا فنان ليس بناجح جدا, ولكني اكسب المال الكافي باعمالي الفنية بالأبيض والأسود لتلبية حاجاتي الاساسية.
اقرب الاقارب لدي هي اختي, ماتت قبل خمس سنوات, ولهذا السبب انا مستقل. \medbreak

فطرت هذا الصباح عند التاسعة, وبعد القاء نظرة سريعة على جريدة الصباح اشعلت غليوني واستمررت لاعطاء عقلي 
الفرصة ليجوم بالارجاء على امل ان احصل على موضوع لكتاتبه. الغرفة, على الرغم من ان الباب والنوافذ كانا مفتوحين, كانت 
حارة بخنق, وكنت قد قررت ان اكثر الاماكن برودة وراحة في الحي هوة اعمق نقطة في المسبح العام, عندها خطرت لي فكرة. \medbreak






\end{otherlanguage}




  
\end{document}
