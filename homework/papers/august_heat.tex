\documentclass[12pt, a4paper]{article}
\usepackage{polyglossia}
\usepackage{geometry}
\usepackage{lua-ul}
\usepackage{color,soul}
\usepackage{xcolor}
\usepackage{hyperref}
\usepackage{setspace}

\onehalfspacing
\setlength\parindent{0pt}

%fonts
\setotherlanguage{arabic}
\newfontfamily\arabicfont[Script=Arabic, Scale=1]{ِAmiri}
\newfontfamily\arabichead[Script=Arabic, Scale=1]{Fustat}

%sections
\newcommand{\head}[1]{
  \phantomsection
  \section*{\centering{\arabichead{#1}}}
  \addcontentsline{toc}{section}{#1}
}

\newcommand{\subhead}[1]{
  \phantomsection
  \subsection*{#1}
  \addcontentsline{toc}{subsection}{#1}
}

\newcommand{\subsubhead}[1]{
  \phantomsection
  \subsubsection*{#1}
  \addcontentsline{toc}{subsubsection}{#1}
}

\begin{document}

\begin{otherlanguage}{arabic}

\head{حرارة اغسطس}

{\arabichead{\centerline{بقلم ويليم فراير هارفي (0191)}}\bigbreak

طريق فينيستون، كلافام \bigbreak

العشرين من اغسطس, —91
}\bigbreak

اظن انني مررتُ باكثر يوم استثنائي في حياتي, وبينما الاحداث لاتزال حية في ذاكرتي, اردت ان اضعهم على الورق بكل وضوح. \medbreak

دعوني ابدأ بالقول ان اسمي جيمس كلارنس ويذرنكروفت. ابلغ اربعين من العمر, بكامل صحتي, لم اعرف قط المرض. 
بسياق العمل انا فنان ليس بناجح جدا, ولكني اكسب المال الكافي باعمالي الفنية بالأبيض والأسود لتلبية حاجاتي الاساسية.
اقرب الاقارب لدي هي اختي, ماتت قبل خمس سنوات, ولهذا السبب انا مستقل. \medbreak

فطرت هذا الصباح عند التاسعة, وبعد القاء نظرة سريعة على جريدة الصباح اشعلت غليوني واستمررت لاعطاء عقلي 
الفرصة ليحوم بالارجاء على امل ان احصل على موضوع لكتابته. الغرفة, على الرغم من ان الباب والنوافذ كانا مفتوحين, كانت 
حارة بخنق, وكنت قد قررت ان اكثر الاماكن برودة وراحة في الحي هوة اعمق نقطة في المسبح العام, عندها خطرت لي فكرة. بدأت في الرسم, عازماً كنت على عملي مما جعلني اترك غدائي غير ملموس, ولم اتوقف الا عندما دقت ساعة كنيسة القديس جودا الرابعة.
النتيجة النهائية, لهذه الرسمة المستعجلة, كانت, بكل تأكيد, احسن اعمالي. \medbreak

صَورت  مجرماً على قفص الاتهام لحظة نطق القاضي حكمه. الرجل كان
بدينًا, بدينا بشكل هائل. تدلت ثنيات من اللحم حول رقبته التي تجعدت حول عنقه الضخم الضئيل. كان حديث الحلاقه—او يجب علي القول
انه كان كذلك منذ ايام— وشبه اصلع. وقف في قفق الاتهام, واصابعه القصيرة المتحيرة ممسكة بقضيب القفص,  محدقًا أمامه مباشرةً. 
الشعور الذي اوحته تعابيره لم تكن تظهر رعباً, وانما, انهياراً تاماً ومطلق. بدا أن لا شيء فيه قادراً على حمل تلك الكتلة اللحمية. \medbreak

طويتُ الرسمة, ومن دون معرفة السبب بالتحديد, وضعتها في جيبي. بعد ذلك ومع شعور السعادة النادر الذي تمنحه المعرفة بانك قمت بعمل
متقن, غادرت المنزل. اعتقد أني شرعتُ بنية زيارة ترينتن, إذ اتذكر اني سرت عبر شارع لينتون ثم انعطفت يمينا على طريق جيلكريست عند
سفح التل حيث كان العمال يعملون على خطوط الترام الجديدة. \medbreak

ومن هناك فصاعداً لم يتبقى لدي سوى ذكرياتٌ مُقتمة عن اماكن ذهابي, الشي الوحيد الذي اذكره هوة الحر الشديد, الذي تصاعد
من الرصيف الاسمنتي المُغبر كموجه شبه ملموسة. تُقت للرعد الموعود به من قبل كتل الغيوم النحاسية الضخمة المتدلية بانخفاض 
في السماء الغربية. \medbreak



لابد اني مشيت خمسة او ستة اميالٍ, حين ايقظني صبي صغير من شرودي بسؤالي عن الوقت. كان الوقت السابعة الا عشرين دقيقة.
عندما غادر, بدإت اتأمل حولي, لقيت نفسي واقفاً امام بوابة تودي الى فناء يحيط به شريط من الارض الجافة, حيث تنتشر
ازهار الْـ ستوك الارجوانية والجيرانيوم القرمزية. كان هناك لوح مكتوب عليه:

\begin{center}
\arabichead{
تشاس أتكينسون \\
نحات نُصُب تذكارية \\
يتخصص في الرخام الإنجليزي والإيطالي \\
}
\end{center}

من الفناء نفسه انبثقت صفرة مرحة, وضجيج ضربات مطرقة, ورنين معدني حاد للتقاء الفولاذ بلحجر. نزوة مفاجئة دفعتني للدخول.
كان رجلٌ جالساً مديراً ظهرهُ باتجاهي. منهمكاً بنحت لوح ٍ  من الرخام ذي عروق غريبة. التفت نحوي عند سماع صوت خطواتي فتوقفت. \medbreak


كان الرجل نفسه الذي كنتُ  قد رسمتهُ واحتفظت ُ بصورته في جيبي. جلس هناك, ضخم كالفيل, والعرق الذي ينساب من فوق رأسه, يمسحه
بمنديل حريري احمر. عالرغم من ان الوجه كان نفسه, الا ان التعابير كانت مختلفة تماماً. \medbreak

حيأني مبتسماً كأننا اصدقاء قدامى, وصافح يدي. اعتذرت عن دخولي عنوة. ”كل شي حار وساطع في الخارج“ انا قلت ”يبدو  كـواحة في الصحراء“
”لا اعلم شيء عن الواحة“  هو اجاب ”لكنها بالتأكيد حارة كالجحيم في الخارج. تفضل بالجلوس, يا سيدي. “ \medbreak

أشار الى طرف شاهد القبر الذي كان يعمل عليه, فجلست. ”هذه قطعه رائعه من الحجر التي لديك“ قلت. هز رأسه ”الى حد ما هي كذلك“ 
اجاب ”السطح هنا هوة افضل ما يمكن ان تجده, لكن هناك عيب كبير في المؤخرة, لا اتوقع ان تلاحظه. يستحيل علي ان اصنع شئ جيد من قطعة 
رخام كهذه.  ستكون على مايرام في فصل الصيف ولن تتضرر في الحرارة, ولكن انتظر حتى يأتي الشتاء. ليس هناك شئ يضاهي الصقيع لكشف نقاط
ضعف الحجر.“  \medbreak


”اذاً فما الغرض منه؟“  سئلت. فانفجر الرجل ضاحكاً. ”سيكون عليك من الصعب تصديقي لو قلت لك انها لمعرض, لكنها الحقيقة. الفنانون يقيمون
معارض, وكذلك البقالون والجزارون, ونحن ايضاً لدينا معارض. احدث المستجدات في شواهد القبر, أتفهم؟“   \medbreak

أستمر يتحدث عن انواع الرخام, ايها اكثر تحملاً للمطر والريح, وايها اسهل في النحت. ثم تحول للحديث عن حديقته ونوع جديد من القرنفل اشتراه.
وبين كل دقيقة واخرى كان يلقي بادواته, يمسح جبهته الساطعة, ويلعن الحر. لم اقل كثيراً لاني شعرت بالاظطراب. كان هناك شئ غير طبيعي 
ومقلق حول هذا الرجل. حاولت في بادئ الامر ان اقنع نفسي بأني رايته من قبل, وان وجهه —رغم كونه غير مالؤف لي— قد وجد مكاناً
ما في زاوية من ذاكرتي, ولكني ادركت بأني اخدع نفسي بحجج واهية. \medbreak

انهى السيد أتكينسون عمله, وبصق في الارض ونهض وهوة يتنهد بارتياح. ”هيا! ماذا تظن في ذلك؟“ قال بنبره من الفخر. وكان النقش
الذي قرأته لاول مرة يحوي الاَتي:



\begin{center}
\arabichead{
مخصص لذكرى \\ 
جيمس كلارنس ويذنكروفت \\ 
وُلِد في 81 يناير 0681 \\
ووافته المنيه فجأه \\ 
في 02 اغسطس —91  \medbreak
}
\end{center}

\centerline{ ”في خضم الحياة نحن في الموت.“} 


جلست صامتاً لبعض الوقت. مرت قشعريرة من طرفي العلوي. سألته اين رأى الاسم. ”لم اره في اي مكان! كنت بحاجة لاسم ما, فكتبت اول ماخطر ببالي, 
لم تسأل؟“ رد السيد أتكينسون  ”انها مصادفة غريبة, ولكن حدث ان يكون نفس اسمي.“ قلت. اطلق صفيراً طويلاً وخافتاً.  \smallbreak

”والتواريخ؟“  

”استطيع ان اجيب عن واحد منها فقط, وهو صحيح.“  

”يا له من امر غريب!“  قال. \smallbreak

ولكني كنت اعلم اكثر منه. اخبرته عما قمت به في الصباح. اخرجتُ  الرسمه من جيبي واريته اياه. وبينما كان ينظر اليها، اخذت تعابير وجهه تتغير 
تدريجيًا حتى اصبحت تشبه تعابير الرجل الذي رسمته. ”وقبل يومين فقط!“ قال  ”كنتُ اخبر ماريا ان الاشباح غير موجودة“ 
لم يسبق لأي منا أن رأى شبحًا، لكنني فهمت ما يعنيه. ”لعلك سمعت اسمي“ قلت ”ولا بد أنك رأيتَني في مكانٍ ما ثم نسيتَ! هل كنتَ في 
كلاكتون-أون-سي في يوليو الماضي؟“ سئلني. لم أزر كلاكتون قط. كنا صامتين لبعض الوقت, وننظر لنفس الشئ: التاريخين المنقوشين على شاهد القبر,
وكان احدهما صحيحاً. \medbreak

”تعالَ إلى الداخل وتناول العشاء معنا“ قال  السيد أتكينسون. زوجته امرأة صغيرة ومفعمة بلحيوية,  ذات خدّين مُحمّرين مُتقشرين كأهل الريف.
قدمني زوجها كـ احد اصدقائه الفنانين. وكانت النتيجة مؤسفة,  فبعد أن أُزيح طبق السردين والجرجير, 
 أحضرتْ نسخةً من الإنجيل برسومات «دوري», واظطررتُ الى الجلوس والتظاهر باعجابي بالرسمات لما يقارب النصف ساعة. \medbreak


ذهبت  إلى الخارج، فوجدتُ أتكينسون جالسًا على شاهد القبر يُدخّن. استأنفنا الحديث من النقطة التي توقفنا عندها.
”عليك ان تعذرني لسؤالي “ قلت  ”لكن هل تذكر انك فعلت شئ يمكن ان تذهب من اجله للمحكمة؟“ هزَّ رأسه. 
”لستُ مفلساً, فالعمل مزدهر بشكل مرضي. قبل ثلاث سنوات، أهديتُ دُيوكًا روميةً لبعض المسؤولين في عيد الميلاد,
ولكن هذا كل مايمكنني تذكره. وكانت ديوكاً صغيرة ايضاً“ اضاف بعد ان فكر. \medbreak

نهض وأحضرَ دلواً من الشرفة، وبدأ يروي الأزهار.  ”مرتين يومياً بانتظام في الحر“  قال ”ومع ذلك تهلك الحرارة احياناً الرقيق منها.
اما السراخس —يا الهي!— فما كانت لتصمد. اين تسكن؟“  اخبرته بعنواني. سيتطلب العودةُ إلى منزلي ساعةً من المشي السريع.
”الامر هكذا“ قال  ”لننظر للمسأله بوضوح.  إذا عدتَ إلى منزلك الليلة، فأنت تُخاطر ان تكون عرضة للحوادث. قد تدهسك عربةٌ، وهناك دائمًا 
قشورُ الموزِ وقشرُ البرتقال، ناهيكَ عن السلالمِ الساقطة.“  \medbreak

تحدَّثَ عن المُستبعَدِ بجديةٍ مُفرطة كانت لتكون مضحكة قبلَ ست ساعات. لكنني لم أضحك.
”أفضل ما يمكننا فعله,“ واصلَ ”هو أن تبقى هنا حتى الساعة الثانية عشرة. سنصعد إلى الطابق العلوي ونُدخّن، فقد يكون الجو ألطفَ في الداخل.“ 
وبشكل مفاجىء وافقت. \medbreak

نحن جالسون الان في غرفة طويلة منخفضة تحت الجوانب.  أرسل أتكينسون زوجته إلى الفراش.
وهو نفسه منهمكٌ في شحذ بعض أدواته على مِسَنّة زيتية صغيرة، بينما يُدخن إحدى سجائري.
يبدو الهواء مُشحونًا برعد. أنا أكتب هذا على طاولة مهتزة قبيل النافذة المفتوحة. إحدى قوائمها مشقوقة, وأتكينسون 
 —الذي يبدو ماهرًا في استخدام أدواته— سيقوم بإصلاحها حالما ينتهي من شحذ حافة إزميله.
انها بعد الحادية عشرة الان. سأهم بالذهاب بعد ساعة من الان.
لكن الحر الخانق كافي لجعل المرء يُجن. \vspace{3em}



\arabichead{\centerline{النهاية}}

\end{otherlanguage}

  
\end{document}
