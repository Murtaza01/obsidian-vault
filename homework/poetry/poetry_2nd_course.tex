\documentclass[12pt, a4paper]{article}
\usepackage{polyglossia}
\usepackage{geometry}
\usepackage{lua-ul}
\usepackage{color,soul}
\usepackage{verse}
\usepackage{xcolor}
\usepackage{hyperref}

\geometry{bottom=1in, top=1in}
\setlength\parindent{0pt}
\newcommand{\attrib}[1]{\nopagebreak{\raggedleft\footnotesize #1\par}}

%fonts
\setmainfont{Erode}
\newfontfamily\fontverse{BespokeSerif}
\newfontfamily\fonthead{Charmonman}


%sections
\newcommand{\head}[1]{
  \phantomsection
  \section*{\centering{#1}}
  \addcontentsline{toc}{section}{#1}
}

\newcommand{\subhead}[1]{
  \phantomsection
  \subsection*{#1}
  \addcontentsline{toc}{subsection}{#1}
}

\newcommand{\subsubhead}[1]{
  \phantomsection
  \subsubsection*{#1}
  \addcontentsline{toc}{subsubsection}{#1}
}

\newcommand{\poemhead}[1]{
  \phantomsection
  \subsubsection*{\centering{\large{#1}}}
  \addcontentsline{toc}{subsubsection}{#1}
}

\begin{document}
\newgeometry{top=0.3in,bottom=1in}
{\fonthead
\begin{center}
\huge{\textbf{Poetry Second Course}}\\[0.2cm]
\Large{By: Birdman \& Fried Potato \& Mohamed}\\[0.3cm]
\end{center}
}

\phantomsection
\subsection*{\centering{\Large{William Shakespeare}}}
\addcontentsline{toc}{subsection}{William Shakespeare}

Was an English playwright, poet and actor. His work consisted
of 39 plays, \hl{154 sonnets}, 3 narrative poems. His sonnet sequence were
the most popular. The subject of his sonnets were about \hl{the dark lady} and his \hl{patron}.
William was \underLine{influenced by Christopher Marlowe} and his use of blank verse. His 
sonnets consisted of 3 quatrains and a couplet which he 
took from Henry Howard.

\poemhead{Sonnet 116}
\settowidth{\versewidth}{let me not to the marriage of true minds}
\begin{verse}[\versewidth]
{\fontverse
Let me not to the \underLine{m}arriage of true \underLine{m}inds\\
Admit impediments. Love is not love \\
Which \underLine{a}lters when it \underLine{a}lteration finds,\\ 
Or bends with the \underLine{r}emover to \underLine{r}emove.
}
\end{verse}
I do not accept that anything can come between 
true lovers. Its not true love that changes when bad things
happen.\medbreak

\textbf{\textcolor{red}{Literary Devices:}} \textbf{Alliteration}\footnote{
The repetition of the same letter in the beginning of words} 
in first line (repetition of m) the third line (repetition of a) and 
fourth line (repetition of r).  \textbf{Metaphor}\footnote{ comparing between
two things that isn't literally true in order to make a resemblance.}  
in \textit{"marriage of true minds"} compares physical union 
(marriage) to deep connection.

\begin{verse}[\versewidth]
{\fontverse
O no! it is an ever-fixed mark\\
That looks on tempests and is never shaken;\\
It is the star to every \underLine{w}and'ring bark,\\
\underLine{W}hose \underLine{w}orth's unknown, although his height be taken.
}
\end{verse}
No, instead love is constant and never changing,
like a lighthouse who is not moved by storms.
It is the star that guides ships. Its value 
is unmeasured. Here Shakespeare uses the images of the 
sea and the star to describe love.\medbreak

\textbf{\textcolor{red}{Literary Devices:}} \textbf{Alliteration} in third and
four line (repetition of w). \textbf{Hyperbole}\footnote{exaggeration for the sake of emphasis.}
in \textit{"ever-fixed"} and \textit{"never shaken"} 
exaggerate love.

\begin{verse}[\versewidth]
{\fontverse
Love's not Time's fool, though rosy lips and cheeks\\
Within his bending sickle's compass come; \\
Love alters not with his brief hours and weeks,\\
But bears it out ev'n to the edge of doom
}
\end{verse}
Love is not a waste of time or a game. It is not only physical
because youth is diminished by time. Love does not change or alter
it stays to end of time. Shakespeare tries to separate between physical
love which is temporary and spiritual love which is eternal.\smallbreak


\textbf{\textcolor{red}{Literary Devices:}} \textbf{Personification}\footnote{
describing a non-human entity with human attribute.} 
in \textit{"Love's not time's fool"} love rejects the wasting of time, fool here means 
jester (clown). \textbf{Hyperbole} in \textit{"edge of doom"} 
exaggerate the eternal and unchanging nature of love.

\restoregeometry

\begin{verse}[\versewidth]
{\fontverse
If this be error and upon me prov'd,\\
I never writ, nor no man ever lov'd.
}
\end{verse}
If I'm wrong and it was proved, then i have 
never written a poem, nor anyone had ever loved.Shakespeare shows his 
confidence about the message of this poem.

\subsubhead{\textit{Sonnet 116} Summary}
Shakespeare tries to separate between two kinds of love:
physical (temporary) love, spiritual
(eternal) love. He uses the images of the sea to describe love
saying its like a lighthouse, constant and never moved by hardships.
He also uses the star image to say that love is like a guide to people.
He conclude by expressing his strong belief about this message.

\poemhead{Sonnet 18}
\settowidth{\versewidth}{Shall I compare thee to a summer’s day?}
\begin{verse}[\versewidth]
{\fontverse
Shall I compare thee to a summer’s day?\\
Thou art more lovely and more temperate: \\
Rough winds do shake the darling buds of May, \\
And summer’s lease hath all too short a date; 
}
\end{verse}
Here Shakespeare asks a rhetorical question to 
express his wonder. He compares his beloved to a lovelier and
more mild summer day. He says that summer does not last too long.\medbreak
\textbf{\textcolor{red}{Literary Devices:}} \textbf{Personification} in \textit{"Rough winds do shake the darling buds"}.


\begin{verse}[\versewidth]
{\fontverse
Sometime too \underLine{h}ot the eye of \underLine{h}eaven shines, \\
And often is his gold complexion dimm'd; \\
And every fair from fair sometime declines, \\
By \underLine{ch}ance or nature’s \underLine{ch}anging course untrimm'd
}
\end{verse}

Shakespeare says that sometimes the sun is too hot in summer and his beloved 
golden face is dimmed. Beauty disappears in time either by 
chance (accidents) or by nature (age).\medbreak

\textbf{\textcolor{red}{Literary Devices:}} \textbf{Alliteration} in first line
(the repetition of h) and fourth line (the repetition of ch). \textbf{Metaphor} in 
\textit{"eye of the heaven"} meaning the sun. 

\begin{verse}[\versewidth]
{\fontverse
But thy eternal summer shall not \underLine{f}ade, \\
Nor lose possession of that \underLine{f}air thou ow’st; \\
Nor \underLine{sh}all death brag thou wander’st in his \underLine{sh}ade, \\
When in eternal lines to time thou grow’st:
}
\end{verse}
But your beauty and glow will not fade, not even death can
kill your memory, you will live eternally because of these
lines (poem).\medbreak

\textbf{\textcolor{red}{Literary Devices:}} \textbf{Alliteration} in first and second line
(repetition  of f) and third line (repetition of sh). \textbf{Personification} in
\textit{"nor shall death brag"}. \textbf{Metaphor} in \textit{"thy eternal summer"} his 
beloved compared to eternal summer. \textbf{Hyperbole} in \textit{"eternal summer shall not fade"}
exaggerate the eternity of his beloved's beauty.

\begin{verse}[\versewidth]
{\fontverse
So long as men can breathe or eyes can see,\\
So long \underLine{l}ives this, and this gives \underLine{l}ife to thee.
}
\end{verse}
As long as there is humans and they can see and read,
you will live eternally through this poem.


\subsubhead{Sonnet 18 Summary}
In this sonnet Shakespeare compares between summer and his beloved, saying that summer is lovely and temperate
but short and sometimes too hot. His beloved however is 
lovelier and more temperate and does not fade, as long as 
people can see and read, this sonnet will make his beloved eternal. The 
sonnet is \underLine{about Shakespeare patron}.

\input{poetry_first_exam.tex}

\subhead{Edmund Waller}
Edmund was part of the parliament and was against the king, he
wrote a poem in praise of Oliver Cromwell. However he changed sides
during the civil war and became part of the royalist (Cavalier poets).

\poemhead{Song: Go, Lovely Rose}
\settowidth{\versewidth}{When I resemble her to thee}
\begin{verse}[\versewidth]
{\fontverse
Go, lovely rose!\\
Tell her that wastes her time and me,\\
That now she knows,\\
When I resemble her to thee,\\
How sweet and fair she seems to be.
} 
\end{verse}

The poet sends the rose as a messenger to tell his beloved; to stop wasting his and her time
and that she is as beautiful as the rose.
\textbf{Apostrophe} is used to address the rose in \textit{"Go, lovely rose"}.
\textbf{Metaphor} in \textit{"resemble her to thee"} comparing a rose to a woman.

\begin{verse}[\versewidth]
{\fontverse
Tell her that’s young,\\
And shuns to have her graces spied,\\
That hadst thou sprung\\
In deserts, where no men abide,\\
Thou must have uncommended died.
} 
\end{verse}

The poet asks the rose to tell his beloved; that if she hides her beauty
like a rose in a dessert she will die unnoticed and without anyone praising her beauty.

\begin{verse}[\versewidth]
{\fontverse
Small is the worth\\
Of beauty from the light retired;\\
Bid her come forth,\\
Suffer herself to be desired,\\
And not blush so to be admired.
} 
\end{verse}

If you have beauty and its not seen, then it has no worth. The poet
tells the rose to invite his beloved to go out and show her beauty, she 
should not be shy to be desired and admired.\footnote{because this poem was 
written at the restoration time, it is a vulgar request for girls to show
their beauty.}

\begin{verse}[\versewidth]
{\fontverse
Then die! that she\\
The common fate of all things rare\\
May read in thee;\\
How small a part of time they share\\
That are so wondrous sweet and fair!
} 
\end{verse}

The poet tell the rose to die so that his beloved may learn that 
beauty only last for a brief time and is distend to fade.

\subsubhead{\textit{Go, Lovely Rose} Summary}

Edmund uses \hl{Apostrophe} to address the rose as 
his messenger. He tell the rose to encourage his beloved and invite her to show her beauty and to not be
shy to be admired, because beauty that is not seen has little worth. He tell the rose to die so 
that his beloved may learn that all this is the destiny of all beautiful things. This poem 
is \hl{carpe diem}.

\subhead{John Suckling}

One of the leaders of the Cavalier poets and part of the school
of Ben Jonson.

\poemhead{Song: Why so Pale?}
\settowidth{\versewidth}{Why so pale and wan fond lover?}
\begin{verse}[\versewidth]
{\fontverse
Why so pale and wan fond lover?\\
\vin Prithee why so pale?\\
Will, when looking well can't move her,\\
\vin Looking ill prevail?\\
\vin Prithee why so pale?
} 
\end{verse}

Why you are so pale foolish lover? If looking good 
did not move her, you think looking ill will? Here
the poet asks a \textbf{rhetorical} question that 
serve as a challenge to the lover action.

\begin{verse}[\versewidth]
{\fontverse
Why so dull and mute young sinner?\\
\vin Prithee why so mute?\\
Will, when speaking well can’t win her,\\
\vin Saying nothing do't?\\
\vin Prithee why so mute?
}
\end{verse}

Why so dull and mute? When you spoke well you did not
move her saying nothing does? \textit{"young sinner"} refers
to the idea of original sin\footnote{In Christian we are born 
sinners because of the first sin (Adam and Eve eating from the tree of knowledge).}

\begin{verse}[\versewidth]
{\fontverse
Quit, quit for shame, this will not move,\\
\vin This cannot take her;\\
If of herself she will not love,\\
\vin Nothing can make her;\\
\vin The devil take her.
} 
\end{verse}

Stop and have some dignity this will not work, this cannot move her.
If she did not love on her own then nothing will force her. Forget about
her "let her go to hell".

\subsubhead{\textit{Why so Pale} Summary}

In this poem the poet asks a \hl{rhetorical question} to challenge
the lover actions and make him think and reflect on them. He 
tells him that if she does not love you on her own then you are wasting
your time, just quit. This poem is taken from a play and is about 
\hl{unrequited love}

\subhead{Richard Lovelace}

One of the Cavalier poets and close friend to John Suckling and part
of the school of Ben Jonson. He was accustomed to court life and
spent some time in prison.

\poemhead{To Althea, from Prison}
\settowidth{\versewidth}{When Love with unconfinèd wings}
\begin{verse}[\versewidth]
{\fontverse
When Love with unconfinèd wings\\
Hovers within my Gates,\\
And my divine Althea brings\\
To whisper at the Grates;\\
When I lie tangled in her hair,\\
And fettered to her eye,\\
The Gods that wanton in the Air,\\
Know no such Liberty.
} 
\end{verse}

Even though I'm trapped in my cell, love hovers around me
and Althea comes to visit me. Not even the gods know such
freedom. Here the poet tries to say that even though
he is in prison cell, his imagination is not, his soul
is free. \textbf{Personification} in \textit{"Love with unconfinèd wings"}.

\begin{verse}[\versewidth]
{\fontverse
When flowing Cups run swiftly round\\
With no allaying Thames,\\
Our careless heads with Roses bound,\\
Our hearts with Loyal Flames;\\
When thirsty grief in Wine we steep,\\
When Healths and draughts go free,\\
Fishes that tipple in the Deep\\
Know no such Liberty.
} 
\end{verse}

When our cups of wine is passed around and we are in a careless state, when 
our heart is filed with love to the king, when our grief is drown in wine and
we make a toast for our health \textit{then} even the fishes that drink from the depth
of the ocean does not know such freedom.


\begin{verse}[\versewidth]
{\fontverse
Stone Walls do not a Prison make,\\
Nor Iron bars a Cage;\\
Minds innocent and quiet take\\
That for an Hermitage.\\
If I have freedom in my Love,\\
And in my soul am free,\\
Angels alone that soar above,\\
Enjoy such Liberty.
} 
\end{verse}

A cell in jail do not make a prison, nor the iron in the cell.
A mind that is quite and calm is like a Hermitage (A place where
religious people go in isolation to pray). If I'm free to love
and my soul is free, then only Angels have the same freedom that i have.

\subsubhead{\textit{To Althea, from Prison} Summary}

In this poem the poet tries to distinguish between physical 
and spiritual freedom. He says that even in jail he is still
free; his imagination/spirit can go anywhere. In the first stanza 
he talks about the sky and his love for Althea. In the second stanza
he talks about the sea and his love for the king. In the final
stanza he talks about his cell being like a Hermitage. He says that
not \textit{The gods in the sky} nor \textit{The fishes in the deep sea} 
are as free as he is but only the Angels that fly above.


\end{document}
