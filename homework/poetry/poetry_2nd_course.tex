\documentclass[12pt, a4paper]{article}
\usepackage{polyglossia}
\usepackage{geometry}
\usepackage{lua-ul}
\usepackage{color,soul}
\usepackage{verse}
\usepackage{tipa}
\usepackage{xcolor}

\geometry{top=0.6in}
\setlength\parindent{0pt}

\setmainfont{Erode}
\newfontfamily\fontverse{BespokeSerif}
%\newfontfamily\fontpar{Erode}
\newfontfamily\fonthead{Philosopher}

\newcommand{\attrib}[1]{\nopagebreak{\raggedleft\footnotesize #1\par}}

\begin{document}

{\fonthead
\begin{center}
\huge{Poetry Second Course}\\[0.5cm]
\Large{By: Birdman \& Fried Potato \& Mơ\textipa{h\r{a}m@\!d}}\\[1cm]
\end{center}
}

\section*{\centering{William Shakespeare}}
Was an English playwright, poet and actor. His work consisted
of 39 plays, \hl{154 sonnets}, 3 narrative poems. His sonnet sequence were
the most popular. The subject of his sonnets were about \hl{the dark lady} and his \hl{patron}.
William was \underLine{influenced by Christopher Marlowe} and his use of blank verse. His 
sonnets consisted of 3 quatrains and a couplet which he 
took from Henry Howard.


\poemtitle{Sonnet 116}
\settowidth{\versewidth}{let me not to the marriage of true minds}
\begin{verse}[\versewidth]
{\fontverse
Let me not to the \underLine{m}arriage of true \underLine{m}inds\\
Admit impediments. Love is not love \\
Which \underLine{a}lters when it \underLine{a}lteration finds,\\ 
Or bends with the \underLine{r}emover to \underLine{r}emove.
}
\end{verse}
I do not accept that anything can come between 
true lovers. Its not true love that changes when bad things
happen.\medbreak

\textbf{\textcolor{red}{Literary Devices:}} \textbf{Alliteration}\footnote{
The repetition of the same letter in the beginning of words} 
in first line (repetition of m) the third line (repetition of a) and 
fourth line (repetition of r).  \textbf{Metaphor}\footnote{ comparing between
two things that isn't literally true in order to make a resemblance.}  
in \textit{"marriage of true minds"} compares physical union 
(marriage) to deep connection.

\begin{verse}[\versewidth]
{\fontverse
O no! it is an ever-fixed mark\\
That looks on tempests and is never shaken;\\
It is the star to every \underLine{w}and'ring bark,\\
\underLine{W}hose \underLine{w}orth's unknown, although his height be taken.
}
\end{verse}
No, instead love is constant and never changing,
like a lighthouse who is not moved by storms.
It is the star that guides ships. Its value 
is unmeasured. Here Shakespeare uses the images of the 
sea and the star to describe love.\medbreak

\textbf{\textcolor{red}{Literary Devices:}} \textbf{Alliteration} in third and
four line (repetition of w). \textbf{Hyperbole}\footnote{exaggeration for the sake of emphasis.}
in \textit{"ever-fixed"} and \textit{"never shaken"} 
exaggerate love.

\begin{verse}[\versewidth]
{\fontverse
Love's not Time's fool, though rosy lips and cheeks\\
Within his bending sickle's compass come; \\
Love alters not with his brief hours and weeks,\\
But bears it out ev'n to the edge of doom
}
\end{verse}
Love is not a waste of time or a game. It is not only physical
because youth is diminished by time. Love does not change or alter
it stays to end of time. Shakespeare tries to separate between physical
love which is temporary and spiritual love which is eternal.\medbreak

\textbf{\textcolor{red}{Literary Devices:}} \textbf{Personification}\footnote{
describing a non-human entity with human attribute.} 
in \textit{"Love's not time's fool"} love rejects the wasting of time, fool here means 
jester (clown). \textbf{Hyperbole} in \textit{"edge of doom"} 
exaggerate the eternal and unchanging nature of love.

\begin{verse}[\versewidth]
{\fontverse
If this be error and upon me prov'd,\\
I never writ, nor no man ever lov'd.
}
\end{verse}
If I'm wrong and it was proved, then i have 
never written a poem, nor anyone had ever loved.Shakespeare shows his 
confidence about the message of this poem.

\subsubsection*{Sonnet 116 Summary}
Shakespeare tries to separate between two kinds of love:
physical (temporary) love, spiritual
(eternal) love. He uses the images of the sea to describe love
saying its like a lighthouse, constant and never moved by hardships.
He also uses the star image to say that love is like a guide to people.
He conclude by expressing his strong belief about this message.

\poemtitle{Sonnet 18}
\settowidth{\versewidth}{Shall I compare thee to a summer’s day?}
\begin{verse}[\versewidth]
{\fontverse
Shall I compare thee to a summer’s day?\\
Thou art more lovely and more temperate: \\
Rough winds do shake the darling buds of May, \\
And summer’s lease hath all too short a date; 
}
\end{verse}
Here Shakespeare asks a rhetorical question to 
express his wonder. He compares his beloved to a lovelier and
more mild summer day. He says that summer does not last too long.\medbreak
\textbf{\textcolor{red}{Literary Devices:}} \textbf{Personification} in \textit{"Rough winds do shake the darling buds"}.


\begin{verse}[\versewidth]
{\fontverse
Sometime too \underLine{h}ot the eye of \underLine{h}eaven shines, \\
And often is his gold complexion dimm'd; \\
And every fair from fair sometime declines, \\
By \underLine{ch}ance or nature’s \underLine{ch}anging course untrimm'd
}
\end{verse}

Shakespeare says that sometimes the sun is too hot in summer and his beloved 
golden face is dimmed. Beauty disappears in time either by 
chance (accidents) or by nature (age).\medbreak

\textbf{\textcolor{red}{Literary Devices:}} \textbf{Alliteration} in first line
(the repetition of h) and fourth line (the repetition of ch). \textbf{Metaphor} in 
\textit{"eye of the heaven"} meaning the sun. 

\begin{verse}[\versewidth]
{\fontverse
But thy eternal summer shall not \underLine{f}ade, \\
Nor lose possession of that \underLine{f}air thou ow’st; \\
Nor \underLine{sh}all death brag thou wander’st in his \underLine{sh}ade, \\
When in eternal lines to time thou grow’st:
}
\end{verse}
But your beauty and glow will not fade, not even death can
kill your memory, you will live eternally because of these
lines (poem).\medbreak

\textbf{\textcolor{red}{Literary Devices:}} \textbf{Alliteration} in first and second line
(repetition  of f) and third line (repetition of sh). \textbf{Personification} in
\textit{"nor shall death brag"}. \textbf{Metaphor} in \textit{"thy eternal summer"} his 
beloved compared to eternal summer. \textbf{Hyperbole} in \textit{"eternal summer shall not fade"}
exaggerate the eternity of his beloved's beauty.

\begin{verse}[\versewidth]
{\fontverse
So long as men can breathe or eyes can see,\\
So long \underLine{l}ives this, and this gives \underLine{l}ife to thee.
}
\end{verse}
As long as there is humans and they can see and read,
you will live eternally through this poem.


\subsubsection*{Sonnet 18 Summary}
In this sonnet Shakespeare compares between summer and his beloved, saying that summer is lovely and temperate
but short and sometimes too hot. His beloved however is 
lovelier and more temperate and does not fade, as long as 
people can see and read, this sonnet will make his beloved eternal. The 
sonnet is \underLine{about Shakespeare patron}.

\section*{\centering{Historical Background of the Seventeen Century}}
\subsection*{King x Parliament}

Both \textit{James I} and his son \textit{Charles I} thought that kings ruled 
by \underLine{Divine Right}. They abused the power of ruling, made illegal taxes
on working people. They were in constant disputes with the Parliament and tried 
many times to rule without them which lead to the civil war.

\subsection*{Civil War (1642-1649)}

The civil war happened between \textit{Charles I} and his supporters 
(\underLine{the Royalist}) against the Parliament and their supporters which were
merchants and tradesmen, they were of the common people and were called (\underLine{the Roundheads}). 
With the help of Scotland the Parliament defeated the king and executed him on 1649.

\subsection*{The Puritans}

In the seventeen century there were two major religious groups: the
\textit{Puritans} who were very strict and thought that all entertainment is
distasteful to God. And the \textit{Catholics} who wanted the Pope in Rome to 
be the head of the church.\medbreak

In (1649-1660) \underLine{Oliver Cromwell} who was a Puritan ruled England. He closed the 
theaters and other entertainments because he thought they were distraction
from the Bible. In the Restoration period people started to be more vulgar and 
indulgent as a reaction to the Puritans strict rules.


\section*{\centering{School of Ben Jonson}}

The school of Ben or the \textit{Tribe of Ben} were group of poets who imitated
Ben Jonson style. They were Royalist and were called \textbf{Cavalier Poets} because they
supported king \textit{Charles I} against the Parliament. This group included: 
\underLine{Thomas Carew}, \underLine{Richard Lovelace}, 
\underLine{John Suckling}, \underLine{Robert Herrick}.

\subsection*{Ben Jonson}

Playwright, critic and poet. Was part of the Royalist who supported 
king Charles I. He lived in the city, 
so most of \underLine{his work were about 
the city side} (urban side). He was influenced by classical 
literature (Greek and Roman) and imitated their work. He used
\textbf{classicism}\footnote{the following of Greek and Roman style and 
principles in literature.} 
in his form. His work were characterised by:

\begin{itemize}
  \item Clarity, order, simplicity, and plainness.
  \item Logic and wit.
  \item Realism and the use of controlled feelings
    \footnote{describing things in realistic  manner and staying away 
    from exaggeration.}
  \item Didacticism\footnote{is a type of literature that aims to teach}
    and instruction.
  \item Refinement of the classics. 
\end{itemize}

\subsubsection*{John Donne x Ben Jonson}

Unlike Ben's form which was simple and plain, Donne's form was complex, 
scientific and included metaphysical element. \underLine{Donne was Ben 
rival-artist}.

\subsubsection*{Ben Jonson Poems} 
Ben favored the \hl{shorter forms} in his work, such as: \underLine{the epigram}
, \underLine{the epitaph}, \underLine{the elegy} and 
\underLine{the epistle}.\medbreak


\textbf{Epigram:} A short poem consisting of 2 line verse that teaches a moral lesson.\medbreak

\textbf{Elegy:} A poem mourning the death of a loved one.\medbreak

\textbf{Epitaph:} A short poem often written on tombstone
to honor the dead.\medbreak

\textbf{Epistle:} A short poem in the form of a letter.




\poemtitle{On my First Son}
\settowidth{\versewidth}{Farewell, thou child of my right hand, and joy}
\begin{verse}[\versewidth]
{\fontverse
Farewell, thou child of my right hand, and joy;\\
My sin was too much hope of thee, lov'd boy. 
}
\end{verse}

Goodbye my child, you were my best thing and the thing that brought me joy.
My sin was that i had high expectation for you. Ben says "right 
hand" to mean best or favorite.

\begin{verse}[\versewidth]
{\fontverse
Seven years tho' wert lent to me, and I thee pay, \\
Exacted by thy fate, on the just day. 
}
\end{verse}

Seven years you were lent to me by fate, and then it took you on the 
exact day you were born (your birthday).

\begin{verse}[\versewidth]
{\fontverse
O, could I lose all father now! For why \\
Will man lament the state he should envy? 
}
\end{verse}

No one will ever call me "father" for you were my only child.
And why should i cry when i really should envy you.

\begin{verse}[\versewidth]
{\fontverse
To have so soon 'scap'd world's and flesh's rage, \\
And if no other misery, yet age?
}
\end{verse}

You have escaped the world too soon, escaped from your
body's trouble, and if you have lived you would suffer from old age.\bigbreak

\begin{verse}[\versewidth]
{\fontverse
Rest in soft peace, and, ask'd, say, "Here doth lie \\
Ben Jonson his best piece of poetry." 
}
\end{verse}

Here Ben says to his child "if they ask you (the angels), say that 
\textit{I (Ben)} lie here (in the grave)". His great pain makes him
feel as if he were dead.

\begin{verse}[\versewidth]
{\fontverse
For whose sake henceforth all his vows be such, \\
As what he loves may never like too much. 
}
\end{verse}

That i promise you my child i will never love anything like you 
again.\medbreak

\attrib{Poem is from \textit{Epigrams} (1616)}

\subsubsection*{\textit{On my First Son} Summary}

In his epigram Ben mourns the death of his favourite and only child, the best thing
that ever happened to him. He says that fate \textit{lent him} his child
only for seven years and then it took it from him. Because of his great
pain he envies the dead. He concludes by promising that he will never love
anything like his child. \hl{This poem is an elegy to Ben's
son}.

\subsubsection*{What \textit{`On my First Son`} is about and what
is the moral lesson?}

The poem is about \hl{honoring the father}. The moral lesson
is that \hl{nothing in life is ours}. Everything is given 
(lent) to us by God, whether it is money, children, or health.\bigbreak

\poemtitle{Song to Celia}
\settowidth{\versewidth}{Come, my Celia, let us prove}
\begin{verse}[\versewidth]
{\fontverse
Come, my Celia, let us prove\\
While we may, the sports of love;
}
\end{verse}

Come Celia let us experience the joys of love while we can.

\begin{verse}[\versewidth]
{\fontverse
Time will not be ours forever;\\
He at length our good will sever.
}
\end{verse}

Time will not be ours forever and will separate us from 
our health. \textit{"He"} refers to time, its a \textbf{personification}.

\begin{verse}[\versewidth]
{\fontverse
Spend not then his gifts in vain.\\
Suns that set may rise again;
}
\end{verse}

Don't waste the gift of time. The day that is over 
may be followed by a new day.

\begin{verse}[\versewidth]
{\fontverse
But if once we lose this light,\\
'Tis with us perpetual night.
}
\end{verse}

But if we lose the light of the sun (life), then 
we will face the night (death). The poet tries to warn 
his beloved about the importance of time.


\begin{verse}[\versewidth]
{\fontverse
Why should we defer our joys?\\
Fame and rumor are but toys.
}
\end{verse}

Why should we delay joys? (this is a \textbf{rhetorical question}
\footnote{a question that doesn't need an answer and asked to create dramatic effect.}).
Reputation is trivial. 

\begin{verse}[\versewidth]
{\fontverse
Cannot we delude the eyes\\
Of a few poor household spies,\\
Or his easier ears beguile,\\
So removed by our wile?
}
\end{verse}

Cannot we deceive the people around us? (a \textbf{rhetorical question}).
The poet tries to encourage his beloved to meet in secret.

\begin{verse}[\versewidth]
{\fontverse
'Tis no sin love's fruit to steal;\\
But the sweet theft to reveal.\\
To be taken, to be seen,\\
These have crimes accounted been.
}
\end{verse}

Its no sin to fall in love and act on it if it was in 
private. It is only a crime if we are seen.

\attrib{From \textbf{Volpone}, 1605}


\subsubsection*{\textit{Song to Celia} Summary}

In this poem Ben invite his beloved to enjoy love while 
they have the time, it is a \hl{carpe diem} poem (sieze the day).
He tries to convince her of the shortness of time and youth and how
they should size it in private while they can. This poem is from the play Volpone.


\subsection*{Robert Herrick}
A Royalist and cavalier poet, part of the school of Ben Jonson,
he is the \underLine{best representative of Ben's school}.

\poemtitle{Delight in Disorder}
\settowidth{\versewidth}{A sweet disorder in the dress. }
%i have some space here just to fix the width of the verse
\begin{verse}[\versewidth]
{\fontverse
A sweet disorder in the dress\\
Kindles in clothes a wantonness;\\
A lawn about the shoulders thrown\\
Into a fine distraction;\\
An erring lace, which here and there\\
Enthrals the crimson stomacher;\\
A cuff neglectful, and thereby\\
Ribands to flow confusedly;\\
A winning wave, (deserving note),\\
In the tempestuous petticoat;\\
A careless shoe-string, in whose tie\\
I see a wild civility:\\
Do more bewitch me, than when art\\
Is too precise in every part.
}
\end{verse}

\subsubsection*{\textit{Delight in Disorder} Summary}

In this poem Robert tells his beloved that there is
some joy in disorder. He tells her that she doesn't 
need to look perfect, that imperfection is charming and natural.
He expresses this in \underLine{subject/predicate} manner where
the first 12 lines is subject and the last two lines is the 
predicate that describe the subject.

\end{document}
