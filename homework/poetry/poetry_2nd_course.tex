\documentclass[12pt, a4paper]{article}
\usepackage{polyglossia}
\usepackage{geometry}
\usepackage{lua-ul}
\usepackage{color,soul}
\usepackage{verse}

\geometry{top=0.6in}
\setlength\parindent{0pt}

\newfontfamily\englishfont{Philosopher}
\setmainfont{Philosopher}
\newfontfamily\fontverse{BespokeSerif}
\newfontfamily\fontpar{Synonym}

\begin{document}

\begin{center}
\huge{Poetry Second Course}\\[0.5cm]
\Large{By: Birdman}\\[1cm]
\end{center}

\section*{\centering{William Shakespeare}}
{\fontpar
Was an English playwright, poet and actor. His work consisted
of 39 plays, \hl{154 sonnets}, 3 narrative poems. His sonnet sequence were
the most popular. The subject of his sonnets were about \hl{the dark lady} and his \hl{patron}.
William was \underLine{influenced by Christopher Marlowe} and his use of blank verse. His 
sonnets consisted of 3 quatrains and a couplet which he 
took from Henry Howard.
}

\poemtitle{Sonnet 116}
\settowidth{\versewidth}{Let me not to the marriage of true minds}
\begin{verse}[\versewidth]
{\fontverse
Let me not to the marriage of true minds \\
Admit impediments. Love is not love \\
Which alters when it alteration finds,\\ 
Or bends with the remover to remove.
}
\end{verse}
{\fontpar
I do not accept that anything can come between 
true lovers. Its not true love that changes when bad things
happen.
}
\begin{verse}[\versewidth]
{\fontverse
O no! it is an ever-fixed mark\\
That looks on tempests and is never shaken;\\
It is the star to every wand'ring bark,\\
Whose worth's unknown, although his height be taken.
}
\end{verse}
{\fontpar
No, instead love is constant and never changing,
like a lighthouse who is not moved by storms.
It is the star that guides ships. Its value 
is unmeasured.\medbreak

Here Shakespeare uses the images of the sea and the star
to describe love.
}
\begin{verse}[\versewidth]
{\fontverse
Love's not Time's fool, though rosy lips and cheeks\\
Within his bending sickle's compass come; \\
Love alters not with his brief hours and weeks,\\
But bears it out ev'n to the edge of doom
}
\end{verse}
{\fontpar
Love is not a waste of time or a game. It is not only physical
because youth is diminished by time. Love does not change or alter
it stays to end of time.\medbreak

\underLine{Here fool means clown}, Shakespeare tries to separate between physical
love which is temporary and spiritual love which is eternal.
}
\begin{verse}[\versewidth]
{\fontverse
If this be error and upon me prov'd,\\
I never writ, nor no man ever lov'd.
}
\end{verse}
{\fontpar
If I'm wrong and it was proved, then i have 
never written a poem, nor anyone had ever loved.\medbreak

Here Shakespeare shows his confidence about the message of this 
poem.
}

\subsection*{Sonnet 116 Summary}
{\fontpar
Shakespeare tries to separate between two kinds of love:
physical (temporary) love, spiritual
(eternal) love. He uses the images of the sea to describe love
saying its like a lighthouse constant and never moved by hardships.
He also uses the star image to say that love is like a guide to people.
}

\poemtitle{Sonnet 18}
\settowidth{\versewidth}{Shall I compare thee to a summer’s day?}
\begin{verse}[\versewidth]
{\fontverse
Shall I compare thee to a summer’s day?\\
Thou art more lovely and more temperate: \\
Rough winds do shake the darling buds of May, \\
And summer’s lease hath all too short a date; 
}
\end{verse}
{\fontpar
Here Shakespeare asks a rhetorical question to 
express his wonder. He compares his beloved to a lovelier and
more mild summer day. He says that summer does not last too long.
}

\begin{verse}[\versewidth]
{\fontverse
Sometime too hot the eye of heaven shines, \\
And often is his gold complexion dimm'd; \\
And every fair from fair sometime declines, \\
By chance or nature’s changing course untrimm'd
}
\end{verse}

{\fontpar
\underLine{Here eye of heaven means the sun}. Shakespeare says that sometimes
the sun is too hot in summer and his beloved golden face is dimmed.
Beauty disappears in time either by chance (accidents) or by
nature (age).
}

\begin{verse}[\versewidth]
{\fontverse
But thy eternal summer shall not fade, \\
Nor lose possession of that fair thou ow’st; \\
Nor shall death brag thou wander’st in his shade, \\
When in eternal lines to time thou grow’st:
}
\end{verse}
{\fontpar
But your beauty and glow will not fade, not even death can
kill your memory, you will live eternally because of these
lines (poem).
}
\begin{verse}[\versewidth]
{\fontverse
So long as men can breathe or eyes can see,\\
So long lives this, and this gives life to thee.
}
\end{verse}
{\fontpar
As long as there is humans and they can see and read,
you will live eternally through this poem.
}

\subsection*{Sonnet 18 Summery}
{\fontpar
In this sonnet Shakespeare compares between summer and his beloved, saying that summer is lovely and temperate
but is short and sometimes the sun is too hot, but his beloved is 
lovelier and more temperate and does not fade, as long as 
people can see and read this sonnet will make his beloved eternal. The 
sonnet is \underLine{about Shakespeare patron}.
}

\section*{\centering{Historical Background on the 17th Century}}

\subsection*{Puritans x Catholics}
\end{document}
