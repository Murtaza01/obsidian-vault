\documentclass[12pt, a4paper]{article}
\usepackage{polyglossia}
\usepackage{geometry}
\usepackage{lua-ul}
\usepackage{color,soul}
\usepackage{verse}
\usepackage{xcolor}
\usepackage{hyperref}

\geometry{bottom=1in, top=1in}
\setlength\parindent{0pt}
\newcommand{\attrib}[1]{\nopagebreak{\raggedleft\footnotesize #1\par}}

%fonts
\setmainfont{Alegreya}
\newfontfamily\fontverse{Alegreya-Bold.otf} 
\newfontfamily\fonthead{Cinzel Decorative}

%sections
\newcommand{\head}[1]{
  \phantomsection
  \section*{\centering{\fonthead{#1}}}
  \addcontentsline{toc}{section}{#1}
}

\newcommand{\subhead}[1]{
  \phantomsection
  \subsection*{#1}
  \addcontentsline{toc}{subsection}{#1}
}

\newcommand{\subsubhead}[1]{
  \phantomsection
  \subsubsection*{#1}
  \addcontentsline{toc}{subsubsection}{#1}
}

\newcommand{\poemhead}[1]{
  \phantomsection
  \subsubsection*{\centering{\large{#1}}}
  \addcontentsline{toc}{subsubsection}{#1}
}

\begin{document}

\small{\textcolor{red}{
* Please focus on the general meaning of poems and their literay devises that
i have highlighted with yellow (These were said by Dr.Haithm). Other literary 
devices were added just in case. Also I only added alliteration in few lines as an example,
but it is everywhere in poems. 
}}

\head{The School of John Donne}

Was not a real school but rather a group of poets whose poetry
had certain features in common. \underLine{Samuel Johnson}\footnote{\, an eighteenth century critic}
called them \textbf{Metaphysical poets} to indicate that these poets used \underLine{scientific 
images} in their poetry. They lived in a period of scientific, intellectual, political, and religious 
changes. Their poetry can be divided into parts: \underLine{the amatory}\footnote{\, relating to lovers or lovemaking.} 
and \underLine{the religious}, though these two aspects are sometimes found together.

\subhead{Characteristics of Metaphysical Poetry}

\begin{itemize}
  \item Metaphysical conceit.
  \item Scientific imagery.
  \item Political and religious themes.
  \item Fusion of mind and heart.
  \item Intellect and Controlled Sentiment\footnote{\, avoiding exaggeration when expressing emotions.}.
  \item Wit\footnote{\, to say something clever and funny.} and humor.
  \item Epigrammatic conciseness\footnote{\, short, clever and memorable statement.}.
\end{itemize}

\subsubhead{Metaphysical conceit}

Or simply \textbf{conceit}, is an extended metaphor that compares between two
seemingly unrelated things to create a surprising resemblance, often to explore 
abstract ideas (e.g., love, faith, death) through concrete imagery.
This unconventional comparison is used to startle and intellectually challenge the reader.
The \hl{most famous example is John Donne's comparison of
two lovers to the legs of a mathematical compass} in \textit{"A Valediction: Forbidding Mourning"}.

\subhead{John Donne}

Born in London in 1572, he was the most outstanding metaphysical poet. His poetry can be divided into: 
the \underLine{secular}\footnote{\, not related to religious or spiritual subjects.}
poems and the \underLine{religious} poems. The majority of his poems deal with 
death, disloyalty or unrequited love. One of the features of his poetry is the \underLine{consistent 
use of the first person}. Most of his poems often center upon himself.


\poemhead{A Valediction: Forbidding Mourning}
\settowidth{\versewidth}{As virtuous men pass mildly away, \vin \vin}
\begin{verse}[\versewidth]
{\fontverse
As virtuous men pass mildly away,\\
\vin And whisper to their souls to go,\\
Whilst some of their sad friends do say\\
\vin The breath goes now, and some say, No:
} 
\end{verse}

When virtuous people die, they pass away gently, their souls departing without struggle.
Their friends, grieving but confused, can’t even agree on the exact moment of death.
The poet compares virtuous people to the departure of lovers, saying that they should 
do it quietly and peacefully. 

\begin{verse}[\versewidth]
{\fontverse
So let us melt, and make no noise,\\
\vin No tear-floods, nor sigh-tempests move;\\
'Twere profanation of our joys\\
\vin To tell the laity our love.
} 
\end{verse}

So do not cry loudly and make a scene, it is profanation to make 
grief public and let laity\footnote{\, ordinary people (outsiders) that do not know us.}
(people) know. In these lines the poet tell his beloved that its 
a profanation to make our separation public and known to outsiders.

\begin{verse}[\versewidth]
{\fontverse
Moving of th' earth brings harms and fears,\\
\vin Men reckon what it did, and meant;\\
But trepidation of the spheres,\\
\vin Though greater far, is innocent.
} 
\end{verse}

Earthquacks brings fears and damage, people know how much 
damage it caused and the meaning of it (e.g, its a punishment from God).
But the plants and stars when crashed causes more damage yet no one notice them.
\textbf{Metaphysical conceit} is used to compare earthquack and trepidation of the spheres
to lovers separation. In physical love separation is like an earthquack, it causes harm and fear.
While in spiritual love separation is like planets and stars crashing, even though their damage is 
great, it goes unnoticed by us.

\begin{verse}[\versewidth]
{\fontverse
Dull sublunary lovers' love\\
\vin (Whose soul is sense) cannot admit\\
Absence, because it doth remove\\
\vin Those things which elemented it.
} 
\end{verse}

Earthly love rooted entirely in physical senses (touch, sight) 
cannot endure separation because absence removes these sensory elements.

\begin{verse}[\versewidth]
{\fontverse
But we by a love so much refined,\\
\vin That our selves know not what it is,\\
Inter-assured of the mind,\\
\vin Care less, eyes, lips, and hands to miss. 
} 
\end{verse}

But we have a love so refined and transcendent that we don't even know how to explain it.
It is mutual and rooted in the mind (spirit), that the absence of 
physical sense is insignificant. \textbf{Paradox}\footnote{
\, a literary device that contradicts itself but contains some truth.} 
is used in \textit{"So much refined..know not what it is"}
The poet says that the lovers love is so sublime yet they do not know what it is.

\begin{verse}[\versewidth]
{\fontverse
Our two souls therefore, which are one,\\
\vin Though I must go, endure not yet\\
A breach, but an expansion,\\
\vin Like gold to airy thinness beat. 
} 
\end{verse}

Our souls are bound as one. Though I must leave, our physical separation won’t break us.
Instead it will stretch and refine our connection, just like the gold that grows thinner and more 
expansive when hammered, yet remains unbroken. \textbf{Metaphysical conceit} is used to compare 
gold to the lovers bond in \textit{"like gold to airy thinness"}. \hl{\textbf{Paradox}
in \textit{"Our two souls therefore, which are one"} describes two souls as one.}

\begin{verse}[\versewidth]
{\fontverse
If they be two, they are two so\\
\vin As stiff twin compasses are two;\\
Thy soul, the fixed foot, makes no show\\
\vin To move, but doth, if the other do. 
} 
\end{verse}

Though we are two individuals, we are like the legs of a compass;
your soul is the fixed foot and the center of my life while
I am the foot that moves. \hl{\textbf{Metaphysical conceit} is used 
to compare the legs of a compass to two lovers.}

\begin{verse}[\versewidth]
{\fontverse
And though it in the center sit,\\
\vin Yet when the other far doth roam,\\
It leans and hearkens after it,\\
\vin And grows erect, as that comes home. 
} 
\end{verse}

Though one leg is in the center, when the other moves away the 
centered leg lean with it and when it comes close it 
becomes straight and adjust to its movements. 

\begin{verse}[\versewidth]
{\fontverse
Such wilt thou be to me, who must,\\
\vin Like th' other foot, obliquely run;\\
Thy firmness makes my circle just,\\
\vin And makes me end where I begun.
} 
\end{verse}

You, my beloved, are the fixed foot of the compass anchored at the center of my world.
While i keep moving to form a perfect circle by the help of your firmness, then i return
to you (this symbolizes the eternity of love).

\subsubhead{\textit{A Valediction: Forbidding Mourning} Summary}

In this poem John uses \underLine{metaphysical conceit} to compare (using scientific images)
two types of love: spiritual love and physical love. He tells his beloved that physical 
love is temporary and is entirely depended on sense. That eventually it will end 
(by separation of the bodies). However his love to
his beloved is transcendent and not effected by physical separation, it 
is eternal. In each stanza he describes spiritual lovers and their separation:

\begin{itemize}
  \item \textbf{First stanza}: Their separation is quite and peaceful. Comparing  it to virtuous men leaving the world.
  \item \textbf {Second stanza}: Their separation is private and without much noise.
  \item \textbf {Third stanza}: Their separation is innocent and harmless. Comparing it to the trepidation of the spheres using
    \textbf{metaphysical conceit}.
  \item \textbf {Fourth stanza}: Their love is transcendent and not depended on sense (touch, sight). 
  \item \textbf {Fifth stanza}: Their love is refined. Using a \textbf{paradox} in 
    \textit{"So much refined..know not what it is"}.
  \item \textbf {Sixth stanza}: Separation strengthen their love and it cannot be broken. Comparing their love 
    to gold using \textbf{metaphysical conceit} (\textit{"Like gold to airy thinness beat"}). 
    When gold is hammered it expand and remains unbroken.
  \item \textbf {Seventh, Eighth, Ninth stanza}: They are connected and depended on each other. 
    She is the center of his life. And their love is eternal.
    Using \textbf{metaphysical conceit} to compare themselves to two legs of a compass.
\end{itemize}

\poemhead{Death, be not proud}
\settowidth{\versewidth}{Death, be not proud, though some have called thee}
\begin{verse}[\versewidth]
{\fontverse
Death, be not proud, though some have called thee\\
Mighty and dreadful, for thou art not so;\\
For those whom thou think'st thou dost overthrow\\
Die not, poor Death, nor yet canst thou kill me.
} 
\end{verse}

The poet uses \hl{\textbf{apostrophe}\footnote{\, a figure of speech that addresses something or someone that cannot respond.}
to addresses death}, saying it is not scary nor powerful. He tells death that when you kill someone, his body 
dies, but his soul lives forever. The poet uses a \textbf{Paradox} in \textit{"canst thou kill me"} to say that 
death cannot kill him and to emphasize the soul' immorality.

\begin{verse}[\versewidth]
{\fontverse
From rest and sleep, which but thy pictures be,\\
Much pleasure; then from thee much more must flow,\\
And soonest our best men with thee do go,\\
Rest of their bones, and soul's delivery.
} 
\end{verse}

The poet uses \textbf{metaphor} to compare death to rest and sleep. He 
says if rest and sleep bring pleasure, death must bring more
pleasure. That righteous and good people die early, and death is nothing but the 
release of soul from body.

\begin{verse}[\versewidth]
{\fontverse
Thou art slave to fate, chance, kings, and desperate men,\\
And dost with poison, war, and sickness dwell,\\
And poppy or charms can make us sleep as well\\
And better than thy stroke; why swell'st thou then?
} 
\end{verse}

The poet addresses death using \textbf{apostrophe} to say that
it is controlled by fate, chance, kings (who order wars) and desperate men (who commit suicide).
That it is associated with poison, wars and sickness. That even drugs and magic spells make us
sleep and does it better than death. So why you (death) are proud and arrogant?

\begin{verse}[\versewidth]
{\fontverse
One short sleep past, we wake eternally\\
And death shall be no more; Death, thou shalt die.
} 
\end{verse}

The poet uses a \textbf{metaphor} to compare death to a short sleep.
He says that death is no more than a short sleep between our life 
and the after life. \hl{\textbf{Paradox} is used in \textit{"death, thou shalt die"}.
That in the after life there is no death, we will live eternally}. \textbf{Biblical Allusion}
in \textit{"Death, thou shalt die"} alluding to the Bible.

\subsubhead{\textit{Death, be not proud} Summary}

In this poem John addresses death using \hl{apostrophe} to tell it that
it is not great nor powerful. Death depend on external factors like chance, 
fate, diseases, and war. John compares death to sleep and rest saying it 
is more pleasurable. He also compares it to a short sleep between 
our life and the after life. And that it will no longer exist in the after life.
This is a \underLine{religious poem} (also called \textit{"Sonnet X"}), it is part of Donne's 
\underLine{Holy Sonnets}\footnote{\, also called Divine Meditations, 
are a series of 19 poems by John Donne.}.


\subhead{George Herbert}

Was a metaphysical poet born in 1593 into an aristocratic family. Educated at
\underLine{Trinity College, Cambridge} he become 
\underLine{public orator} there. In 1624 he become a \underLine{member of the parliament},
but soon lost the favor of the king. He got interested in religion and became a \underLine{priest}.
His volume of collected poems \textbf{The Temple} indicates how important religion was to him.

\poemhead{The Pulley}
\settowidth{\versewidth}{Having a glass of blessings standing by "let"}
\begin{verse}[\versewidth]
{\fontverse
When God at first made man,\\
Having a glass of blessings standing by,\\
“Let us,” said he, “pour on him all we can.\\
Let the world’s riches, which dispersèd lie,\\
Contract into a span.”
} 
\end{verse}

When God first made humans, He was holding a glass of blessings that
He poured to create humans with. All the riches of the world (strength, beauty, pleasure, etc) were gathered in 
one entity; human life. \hl{\textbf{Metaphor} is used in \textit{"glass of blessings"} to represent God grace.}
\textbf{Hyperbole}\footnote{\, exaggeration for the sake of emphasis.} 
in \textit{"pour on him all we can"} to exaggerate God generosity. \textbf{Biblical Allusion} used
in \textit{"Let us, said he"}

\begin{verse}[\versewidth]
{\fontverse
So strength first made a way;\\
Then beauty flowed, then wisdom, honour, pleasure.\\
When almost all was out, God made a stay,\\
Perceiving that, alone of all his treasure,\\
Rest in the bottom lay.
} 
\end{verse}

God first gave us strength, then beauty, wisdom, honour and pleasure.
When the glass was almost poured fully, God made a pause, withholding 
fulfilment in the bottom of the glass. \textbf{Personification}\footnote{\,
describing a non-human entity with human attribute.}
used in \textit{"strength first made a way"} giving strength human attributes
(making a way). \textbf{Metaphor} used in \textit{"treasure"} to compare
a treasure to God's blessings.

\begin{verse}[\versewidth]
{\fontverse
“For if I should,” said he,\\
“Bestow this jewel also on my creature,\\
He would adore my gifts instead of me,\\
And rest in Nature, not the God of Nature;\\
So both should losers be.
} 
\end{verse}

God says that if He gave his creation fulfilment then they would 
forget all about Him. Adoring the gifts and forgetting 
about the one who gave it to them. Then both God and humans would
lose; God by not having people to worship and glorify Him. Humans by
losing eternal fulfilment and knowing their creator.  \textbf{Metaphor}
used in \textit{"Bestow this jewel"} to refer to fulfilment.

\begin{verse}[\versewidth]
{\fontverse
“Yet let him keep the rest,\\
But keep them with repining restlessness;\\
Let him be rich and weary, that at least,\\
If goodness lead him not, yet weariness\\
May toss him to my breast.”
} 
\end{verse}

Let humans keep earthly blessings (strength, beauty, wisdom) to
themselves, but keep them with constant dissatisfaction. Let them 
be rich but weary. So that if moral goodness wouldn't lead them
to me, weariness would. \textbf{Metaphor} used in \textit{"weary"}
comparing physical tiredness to spiritual exhaustion. \textbf{Personification}
used in \textit{"weariness May toss him"} giving weariness human attributes \textit{"toss"}.

\subsubhead{\textit{The Pulley} Summary}

In the poem \textit{The Pulley} God blesses people with all worldly riches
(e.g, strength, wisdom, beauty, etc) expect \underLine{spiritual fulfilment}. Withholding
it so that people would return to him. The poem shows that earthly blessings will not
bring spiritual fulfilment, and that this need was made in us by God so that we would not forget
about him. George uses \textbf{metaphysical conceit} in the poem title to compare a pulley\footnote{
\, a machine that used to lift heavy objects.
\href{https://i.pinimg.com/736x/08/3b/ad/083badbb5a73ae0b9ae40093063f61e1.jpg}{\textcolor{blue}{See image}} }
to God's plan to lift humans to Him. Human restlessness (dissatisfaction) 
act as a weight which God uses to draw people to Him.
The poem is part of George volume \underLine{The Temple}.

\subhead{Henry Vaughan}

Was a metaphysical poet and a \underLine{medical physician} born in 1622.
He was influenced by Donne but his \underLine{major influence 
came from George Herbert}. His best poetry appeared in (1650-1655) in his
religious volume (\textit{Silex Scintillans}).


\poemhead{The Retreat}
\settowidth{\versewidth}{Happy those early days! when I}
\begin{verse}[\versewidth]
{\fontverse
Happy those early days! when I\\
Shined in my angel infancy.
} 
\end{verse}

Happy were the days of my childhood when i was pure like an angel.
\hl{\textbf{Metaphor} used in \textit{"angel infancy"} to compare divine purity to childhood}.
The poet expresses a nostalgia\footnote{\, a longing for a period in the past.}
for his childhood.

\begin{verse}[\versewidth]
{\fontverse
Before I understood this place\\
Appointed for my second race, 
} 
\end{verse}

Before i understand that this earth is a temporary place which 
i came to from another life (first existence). \textbf{Allusion}
to Platonism\footnote{\, a philosophy based on Plato's ideas.}
philosophy which asserts that the soul existed before this life.

\begin{verse}[\versewidth]
{\fontverse
Or taught my soul to fancy aught\\
But a white, celestial thought;
} 
\end{verse}

Or before my soul desired anything expect pure and divine thoughts.
The poet says that in childhood his desires were innocent and pure.
\textbf{Metaphor} used in \textit{"celestial thought"} to mean
transcendent thinking. \textbf{Hyperbole} used in \textit{"celestial thought"}
exaggerating how pure he once was. \textbf{Symbolism} in \textit{"white"} to symbolize 
purity, innocence, and divinity.

\begin{verse}[\versewidth]
{\fontverse
When yet I had not walked above\\
A mile or two from my first love, 
} 
\end{verse}

When i had not walked away from God, when i was pure and innocent.
\textbf{Metaphor} in \textit{"walked above a mile"} to compare
between physical walking to separation from God. Also 
in \textit{"first love"} to refer to God. \textbf{Paradox} used in \textit{"walked above"}
the upward movement is contrasted in the falling away from divine grace (God).


\begin{verse}[\versewidth]
{\fontverse
And looking back, at that short space,\\
Could see a glimpse of His bright face;
} 
\end{verse}

And when i look back at my childhood (the days of innocence), i could
briefly remember being close to God. \textbf{Metaphor} used in \textit{"bright face"} 
to refer to God's presence. 

\begin{verse}[\versewidth]
{\fontverse
When on some gilded cloud or flower\\
My gazing soul would dwell an hour, 
} 
\end{verse}

When i was a child i would look at nature for hours and admire it.
\textbf{Personification} \textit{"soul would dwell"} giving the soul
human attributes. 


\begin{verse}[\versewidth]
{\fontverse
and in those weaker glories spy\\
some shadows of eternity;
} 
\end{verse}

Earthly beauties are mere reflection of heaven, they are 
but a tiny thing compared to heaven. \textbf{Oxymoron}\footnote{
\, a figure of speech that combines contradictory words with opposing meanings.}
in \textit{"weaker glories"}.
\textbf{Metaphor} in \textit{"weaker glories"} to refer to nature and earthly beauty. 
\textbf{Allusion} to Platonism philosophy which emphasizes that the material world is imperfect 
reflection of a higher spiritual world.

\begin{verse}[\versewidth]
{\fontverse
before i taught my tongue to wound\\
my conscience with a sinful sound, 
} 
\end{verse}

Before i learned to use my tongue in a sinful way that hurts
my conscience (cause guilt). The poet remembers a time where he was free of guilt 
from speaking sinfully and hurtfully. \textbf{Personification} in \textit{"tongue to wound"}
giving the tongue a human attribute, also in \textit{"wound my conscience"} making 
his conscience as something that can be wounded. \textbf{Metaphor} in \textit{"wound"} to refer
to hurtful speech. \textbf{Metonymy}\footnote{\, a figure of speech where an object 
is referred to by something closely associated with it.}
in \textit{"tongue"} to represent speech, also in \textit{"sound"} to represent spoken words.

\begin{verse}[\versewidth]
{\fontverse
Or had the black art to dispense\\
A \underLine{s}everal \underLine{s}in to every \underLine{s}ense, 
} 
\end{verse}

If i had black magic i would assign a sin to each sense (sight, hearing, touch, etc.).
The poet is exaggerating his capacity for evil and assigning every part of the body to a sin.
\textbf{Personification} in \textit{"dispense a several sin"}, here sin is a substance that
can be dispensed (distributed). \textbf{Metaphor} in \textit{"black art"} to refer
to evil. \textbf{Alliteration} by repeating the 's' sound.

\newpage
\begin{verse}[\versewidth]
{\fontverse
But felt through all this fleshly dress\\
Bright shoots of everlastingness.
} 
\end{verse}

Despite being in this physical body, i still feel bursts of spiritual divinity.
 \textbf{Metaphor} in \textit{"fleshly dress"} to refer to physical body.
 \textbf{Paradox} in \textit{"shoots of everlastingness"} that eternal divinity is
felt through a mortal body. \textbf{Allusion} to Platonism which asserts
that the physical body is a temporary container of the soul. \textbf{Metaphysical conceit} 
in \textit{"fleshly dress..shoots of everlastingness"} to compare 
Physical body to eternal existence.


\begin{verse}[\versewidth]
{\fontverse
O, how i long to \underLine{t}ravel back,\\
and \underLine{t}read again that ancient \underLine{t}rack!
} 
\end{verse}

How i long to get back to that state (of being pure, innocent and close to God).
To the original path of God. \textbf{Alliteration}\footnote{\, 
the repetition of the same letter in the beginning of words} by repeating the 't' sound.
\textbf{Metaphor} in \textit{"ancient track"} to refer to the path of God.

\begin{verse}[\versewidth]
{\fontverse
That I might once more reach that plain\\
Where first I left my glorious train, 
} 
\end{verse}

That i might once again get back to a simpler and purer time
where i left my true path to God. \textbf{Metaphor} in \textit{"glorious train"} 
to refer to divine grace, also in \textit{"that plain"} to refer to spiritual peacefulness.


\begin{verse}[\versewidth]
{\fontverse
From whence th’ enlightened \underLine{s}pirit \underLine{s}ees\\
That \underLine{s}hady city of palm trees.
} 
\end{verse}

From there the enlightened spirits can see a city which is peaceful and heavenly.
The poet suggest that when we enter spiritual clarity, we are no longer blind
and can see a heavenly place. \textbf{Metaphor} in \textit{"Enlightened spirit"} 
to mean spiritual clarity, also in \textit{"Shady city of palm trees"} to refer
to a heavenly place. \textbf{Biblical Allusion} in \textit{"city of palm trees"} 
represent God's promised land. \textbf{Alliteration} by repeating the 's' sound.

\begin{verse}[\versewidth]
{\fontverse
But, ah! my \underLine{s}oul with too much \underLine{s}tay\\
Is drunk, and \underLine{s}taggers in the way. 
} 
\end{verse}

But sadly my soul is in too much sin. My soul has become confused 
and lost its way. The poet expresses his despair about his moral corruption
and his inability to change it. \textbf{Metaphor} in \textit{"is drink"} to refer to
lack of control or moral corruption. \textbf{Personification} in \textit{"my soul...Is drunk"} 
giving his soul human attributes. \textbf{Alliteration} by repeating the 's' sound. 
\textbf{Metaphysical conceit} in \textit{"drunk, and staggers in the way"} 
comparing Spiritual decline to drunkenness.



\begin{verse}[\versewidth]
{\fontverse
Some men a forward motion love;\\
But I by backward steps would move,
} 
\end{verse}

Some men seek earthly progress. But I seek to go backward (to spiritual purity
and being close to God). \textbf{Metaphor} in \textit{"forward motion"} 
to refer to earthly progress, also in \textit{"backward steps"} to refer 
to spiritual purity. \textbf{Antithesis}\footnote{\,
a figure of speech that position opposing ideas within one statement.}
in \textit{"forward motion..backward steps"} creates a contrast to emphasize
the rejection of worldly advancement in favor of the spiritual one. \textbf{Paradox} in 
\textit{"forward motion"} which is considered as regression, leading to moral corruption. Also
in \textit{"backward steps"} which is considered to be spiritual advancement.

\begin{verse}[\versewidth]
{\fontverse
And when this dust falls to the urn,\\
In that state I came, return.
} 
\end{verse}

And when this physical body dies and is buried. I will return to the state 
which i came in. The poet asserts that after death his soul will revert back
to its divine place and eternal from. \textbf{Metaphor} in \textit{"dust"} 
to refer to human body, also in \textit{"urn"} to refer to grave. \textbf{Biblical Allusion}
is used in the description of the return to God. \textbf{Allusion} to Platonism 
which emphasizes that death is the separation of soul from body and the return of soul to
its original place.


\subsubhead{\textit{The Retreat} Summary}

In this \underLine{nostalgic poem} Henry lament a time where he was once pure, innocent and close to God.
Those times were his childhood and the life before. Being older he begun to sin
and walk away from God's path. He longs for divine closeness and wants to relive the blessings he had.
In the last couplet he wishes to die so that  he could return to God and his
pure and innocent state. Throughout the poem Henry uses \underLine{Platonic} and \underLine{Biblical} allusions.
The poem is written in iambic tetrameter\footnote{\, four stressed syllables in a line} and
rhyming couplets\footnote{\, two lines of verse that is rhymed and metered.}. 


\subhead{Andrew Marvell}

A politician and a metaphysical poet who was born in 1621. He was 
friend to John Milton. He was influenced by both the 
metaphysical school and the school of Ben Jonson.

\poemhead{A Dialogue between the Soul and the Body}
\settowidth{\versewidth}{O who shall, from this dungeon, raise }
\centerline{\textbf{\large{Soul}}}
\begin{verse}[\versewidth]
{\fontverse
O who shall, from this dungeon, raise\\
a soul enslav’d so many ways?\\
with bolts of bones, that fetter’d stands\\
in feet, and manacled in hands;\\
here blinded with an eye, and there\\
deaf with the drumming of an ear;
} 
\end{verse}

Oh who shall free me from this painful prison (body)?
Whose bones (ribs) feels like prison bars and hands
and feet feels like chains. Whose eyes blind me 
and ears makes me deaf (symbolizing that sight can 
makes us blind at times and our ears are easily distracted). 
\textbf{Metaphor} in \textit{"bolts of bones"} to refer to prison bars. 
\textbf{Paradox} in \textit{"blinded with an eye"}, that
having eyesight blinds our soul. Also in \textit{"deaf with
drumming of an ear"}, that having ears deafens and distract our
soul. \textbf{Personification} the soul is given a human attribute
(a prisoner). \hl{\textbf{Rhetorical question}\footnote{\, 
a question that doesn't need an answer. It is asked to create dramatic effect.}.
in \textit{"a soul enslav’d so many ways?"}}. 

\begin{verse}[\versewidth]
{\fontverse
A soul hung up, as ’twere, in chains\\
Of nerves, and arteries, and veins;\\
Tortur’d, besides each other part,\\
In a vain head, and double heart\footnote{\, hypocrisy}. 
} 
\end{verse}

A soul that is tortured by the body and its organs and nerves.
Inside a prideful and arrogant head, and a heart that experience
mixed emotions (love/hate). \textbf{Personification} in \textit{"tortur'd"} 
given the soul a human attribute (being tortured), also in \textit{"a soul.. in chains"}.\bigbreak

\newpage
\centerline{\textbf{\large{Body}}}
\begin{verse}[\versewidth]
{\fontverse
O who shall me deliver whole\\
From bonds of this tyrannic soul?\\
Which, stretch’d upright, impales me so\\
That mine own precipice I go;\\
And warms and moves this needless frame,\\
(A fever could but do the same)
} 
\end{verse}

Who shall set me free from this cruel and oppressive soul? Which compel 
me to do things that i don't like (like being virtuous and moral). It 
heats and warms (a task that a fever can do). The body mocks the soul 
for doing the same job a fever do. \hl{\textbf{Rhetorical question} in
\textit{"who shall..tyrannic soul?"}}. \textbf{Personification} the soul
is given human attributes (being a tyrant). \textbf{Paradox} in \textit{"tyrannic soul"}
the soul which is a symbol for freedom is a jailer.

\begin{verse}[\versewidth]
{\fontverse
And, wanting where its spite to try,\\
Has made me live to let me die.\\
A body that could never rest,\\
Since this ill spirit it possest. 
} 
\end{verse}

The soul forces me to live, only to torture me and eventually let me die.
It won't let me rest because it is evil and tyrant. \textbf{Metaphor} in \textit{"ill spirit"} 
the soul is compared to a demon. \textbf{Hyperbole} in \textit{"A body that could never rest"}.
\bigbreak

\centerline{\textbf{\large{Soul}}}
\begin{verse}[\versewidth]
{\fontverse
What magic could me thus confine\\
Within another’s grief to pine?\\
Where whatsoever it complain,\\
I feel, that cannot feel, the pain;\\
And all my care itself employs;\\
That to preserve which me destroys;
} 
\end{verse}

What magic traps me like this and make me suffer the body's saddens? That
whatever the body feels I feel it too even though I shouldn't. And all my 
care goes to keep the body alive, even though that destroys me (by neglecting 
my spiritual needs and paying all my attention to the body). \hl{\textbf{Rhetorical question} in
\textit{"What magic..to pine?"}}. \textbf{Personification} in which the soul is given 
human attributes (the soul feels, pine and care). \textbf{Metaphor} in \textit{"magic"}
to mean a hidden and wondrous force. \textbf{Synecdoche}\footnote{\,
a figure of speech that uses a part of something to refer to the whole.} 
in \textit{"another's grief"} referring to the body.


\begin{verse}[\versewidth]
{\fontverse
Constrain’d not only to endure\\
Diseases, but, what’s worse, the cure;\\
And ready oft the port to gain,\\
Am shipwreck’d into health again. 
} 
\end{verse}

Am forced not only to endure the diseases of the body but also the cure.
And when Am ready to die and leave this body, am pulled back by health.
The soul expresses its frustration with the body's sickness and painful cures.
And when there is hope of the soul leaving this body, physical health destroys it. 
\textbf{Paradox} in \textit{"what's worse, the cure"} asserting that the cure is worse
than the disease. Also in \textit{"shipwreck’d into health again"} asserting that health
is a bad thing. \textbf{Metaphor} in \textit{"port to gain"} referring to death,
also \textit{"shipwreck'd into health again"} to mean "getting back to life again".\bigbreak

\centerline{\textbf{\large{Body}}}
\begin{verse}[\versewidth]
{\fontverse
But physic yet could never reach\\
The maladies thou me dost teach;\\
Whom first the cramp of hope does tear,\\
And then the palsy shakes of fear;\\
The pestilence of love does heat,\\
Or hatred’s hidden ulcer eat; 
} 
\end{verse}

But medicine cannot cure the emotional and mental pain that the soul
causes. The cramping hope, the paralyzing fear, the overwhelming love
that feels like a plague and the hatred that eats
the body. \textbf{Paradox} in \textit{"cramp of hope"} hope which is a positive
thing is being described negatively. Also in \textit{"pestilence of love"} 
framing love negatively. \textbf{Personification} in
\textit{"hope does tear"}, \textit{"palsy shakes of fear"}, \textit{"love does heat"} and 
\textit{"hatred..eat"} given emotions a human attribute. 

\begin{verse}[\versewidth]
{\fontverse
Joy’s cheerful madness does perplex,\\
Or sorrow’s other madness vex;\\
Which knowledge forces me to know,\\
And memory will not forego.\\
What but a soul could have the wit\\
To build me up for sin so fit? 
} 
\end{verse}

Joy is overwhelming and confusing, sorrow is annoying. I 
(the body is speaking here) cannot forget these emotions because 
of the souls knowledge of them. Who other then the soul that have 
the intellect to make me perfect for sinning. \hl{\textbf{Rhetorical question}
in \textit{"What...for sin so fit?"}}. \textbf{Oxymoron} in \textit{"Joy cheerful madness"}
framing a good emotion (joy) as a negative one. \textbf{Personification}
in \textit{"knowledge forces me to know"} the body is given human attribute.


\begin{verse}[\versewidth]
{\fontverse
So architects do square and hew\\
Green trees that in the forest grew.
} 
\end{verse}

So the architects (the soul) cut and shape the wood from
the green trees that grow in the forest. The Body is saying
that he was once natural like the green trees but the soul
came and changed his natural shape. \textbf{Metaphor} 
in \textit{"architects"} to refer to soul
and \textit{"Green trees"} to refer to body.

\subsubhead{\textit{A Dialogue between the Soul and the Body} Summary}

In this poem the Soul and Body have a debate, each complaining about the 
other. The Soul describes the \underLine{Body as a prison}, filed with diseases and
pains. The Body describes the \underLine{Soul as a tyrant}, filling the Body
with overwhelming emotions . The poem contains allusion to Platonism.
Andrew offers no resolution in the debate, symbolizing the \hl{constant 
struggle between physical desires and spiritual needs}. 


\end{document}

