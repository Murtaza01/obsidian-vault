\documentclass[12pt, a4paper]{article}
\usepackage{polyglossia}
\usepackage{geometry}
\usepackage{lua-ul}
\usepackage{color,soul}
\usepackage{verse}
\usepackage{xcolor}
\usepackage{hyperref}

\geometry{bottom=1in, top=1in}
\setlength\parindent{0pt}
\newcommand{\attrib}[1]{\nopagebreak{\raggedleft\footnotesize #1\par}}

%fonts
\setmainfont{Erode}
\newfontfamily\fontverse{BespokeSerif}
\newfontfamily\fonthead{Charmonman}


%sections
\newcommand{\head}[1]{
  \phantomsection
  \section*{\centering{#1}}
  \addcontentsline{toc}{section}{#1}
}

\newcommand{\subhead}[1]{
  \phantomsection
  \subsection*{#1}
  \addcontentsline{toc}{subsection}{#1}
}

\newcommand{\subsubhead}[1]{
  \phantomsection
  \subsubsection*{#1}
  \addcontentsline{toc}{subsubsection}{#1}
}

\newcommand{\poemhead}[1]{
  \phantomsection
  \subsubsection*{\centering{\large{#1}}}
  \addcontentsline{toc}{subsubsection}{#1}
}

\begin{document}
\newgeometry{top=0.3in,bottom=1in}
{\fonthead
\begin{center}
\huge{\textbf{Seventeenth Century Poetry}}\\[0.2cm]
\Large{By: Birdman \& Luffy \& Mohamed}\\[0.3cm]
\end{center}
}

\head{The School of John Donne}

Was not a real school but rather a group of poets whose poetry
had certain features in common. \underLine{Samuel Johnson}\footnote{\, an eighteenth century critic}
called them \textbf{Metaphysical poets} to indicate that these poets used scientific 
images in their poetry. They lived in a period of scientific, intellectual, political, and religious 
changes. Their poetry can be divided into parts: \underLine{the amatory}\footnote{\, relating to lovers or lovemaking.} 
and \underLine{the religious}, though these two aspects are sometimes together.

\subhead{Characteristics of Metaphysical Poetry}

\begin{itemize}
  \item Metaphysical conceit.
  \item Scientific imagery.
  \item Political and religious themes.
  \item Fusion of mind and heart.
  \item Intellect and Controlled Sentiment\footnote{\, avoiding exaggeration when expressing emotions.}
  \item Wit and humor.
  \item Epigrammatic conciseness\footnote{\, short, clever and memorable statement.}
\end{itemize}

\subsubhead{Metaphysical conceit}

Is an extended metaphor that is intellectually complex, comparing two
seemingly unrelated things to create a surprising resemblance, often to explore 
abstract ideas (e.g., love, faith, death) through concrete (physical) imagery.
This comparison is used to startle and intellectually challenge the reader.
The \hl{most famous example is John Donne's comparison of
two lovers to the legs of a mathematical compass} in \textit{"A Valediction: Forbidding Mourning"}.

\subhead{John Donne}

\end{document}

