\documentclass[12pt, a4paper]{article}
\usepackage{polyglossia}
\usepackage{geometry}
\usepackage{lua-ul}
\usepackage{color,soul}
\usepackage{verse}
\usepackage{xcolor}
\usepackage{hyperref}

\geometry{bottom=1in, top=1in}
\setlength\parindent{0pt}
\newcommand{\attrib}[1]{\nopagebreak{\raggedleft\footnotesize #1\par}}

%fonts
\setmainfont{Alegreya}
\newfontfamily\fontverse{Alegreya-Bold.otf} 
\newfontfamily\fonthead{Cinzel Decorative}

\begin{document}
  
\phantomsection
\subsection*{\centering{\fonthead{\Large{{William Shakespeare}}}}}\bigbreak
\addcontentsline{toc}{subsection}{William Shakespeare}

Was an English playwright, poet and actor. His work consisted
of 39 plays, \hl{154 sonnets}, 3 narrative poems. His sonnet sequence were
the most popular. The subject of his sonnets were about \hl{the dark lady} and his \hl{patron}.
William was \underLine{influenced by Christopher Marlowe} and his use of blank verse. His 
sonnets consisted of 3 quatrains and a couplet which he 
took from Henry Howard.

\poemhead{Sonnet 116}
\settowidth{\versewidth}{let me not to the marriage of true minds}
\begin{verse}[\versewidth]
{\fontverse
Let me not to the \underLine{m}arriage of true \underLine{m}inds\\
Admit impediments. Love is not love \\
Which \underLine{a}lters when it \underLine{a}lteration finds,\\ 
Or bends with the \underLine{r}emover to \underLine{r}emove.
}
\end{verse}
I do not accept that anything can come between 
true lovers. Its not true love that changes when bad things
happen. \textbf{Alliteration}\footnote{
The repetition of the same letter in the beginning of words} 
in first line (repetition of m) the third line (repetition of a) and 
fourth line (repetition of r).  \textbf{Metaphor}\footnote{ comparing between
two things that isn't literally true in order to make a resemblance.}  
in \textit{"marriage of true minds"} compares physical union 
(marriage) to deep connection.

\begin{verse}[\versewidth]
{\fontverse
O no! it is an ever-fixed mark\\
That looks on tempests and is never shaken;\\
It is the star to every \underLine{w}and'ring bark,\\
\underLine{W}hose \underLine{w}orth's unknown, although his height be taken.
}
\end{verse}
No, instead love is constant and never changing,
like a lighthouse who is not moved by storms.
It is the star that guides ships. Its value 
is unmeasured. Here Shakespeare uses the images of the 
sea and the star to describe love. \textbf{Alliteration} in third and
four line (repetition of w). \textbf{Hyperbole}\footnote{exaggeration for the sake of emphasis.}
in \textit{"ever-fixed"} and \textit{"never shaken"} 
exaggerate love.

\begin{verse}[\versewidth]
{\fontverse
Love's not Time's fool, though rosy lips and cheeks\\
Within his bending sickle's compass come; \\
Love alters not with his brief hours and weeks,\\
But bears it out ev'n to the edge of doom
}
\end{verse}
Love is not a waste of time or a game. It is not only physical
because youth is diminished by time. Love does not change or alter
it stays to end of time. Shakespeare tries to separate between physical
love which is temporary and spiritual love which is eternal. \textbf{Personification}\footnote{
describing a non-human entity with human attribute.} 
in \textit{"Love's not time's fool"} love rejects the wasting of time, fool here means 
jester (clown). \textbf{Hyperbole} in \textit{"edge of doom"} 
exaggerate the eternal and unchanging nature of love.


\begin{verse}[\versewidth]
{\fontverse
If this be error and upon me prov'd,\\
I never writ, nor no man ever lov'd.
}
\end{verse}
If I'm wrong and it was proved, then i have 
never written a poem, nor anyone had ever loved. the poet show his 
confidence about the message of his poem.


\subsubhead{\textit{Sonnet 116} Summary}
Shakespeare tries to separate between two kinds of love:
physical (temporary) love, spiritual
(eternal) love. He uses the images of the sea to describe love
saying its like a lighthouse, constant and never moved by hardships.
He also uses the star image to say that love is like a guide to people.
He conclude by expressing his strong belief about this message.

\poemhead{Sonnet 18}
\settowidth{\versewidth}{Shall I compare thee to a summer’s day?}
\begin{verse}[\versewidth]
{\fontverse
Shall I compare thee to a summer’s day?\\
Thou art more lovely and more temperate: \\
Rough winds do shake the darling buds of May, \\
And summer’s lease hath all too short a date; 
}
\end{verse}
Here Shakespeare asks a rhetorical question to 
express his wonder. He compares his beloved to a lovelier and
more mild summer day. He says that summer does not last too long. \textbf{Personification} in
\textit{"Rough winds do shake the darling buds"} given the wind human attribute.


\begin{verse}[\versewidth]
{\fontverse
Sometime too \underLine{h}ot the eye of \underLine{h}eaven shines, \\
And often is his gold complexion dimm'd; \\
And every fair from fair sometime declines, \\
By \underLine{ch}ance or nature’s \underLine{ch}anging course untrimm'd
}
\end{verse}

Shakespeare says that sometimes the sun is too hot in summer and his beloved 
golden face is dimmed. Beauty disappears in time either by 
chance (accidents) or by nature (age). \textbf{Alliteration} in first line
(the repetition of h) and fourth line (the repetition of ch). \textbf{Metaphor} in 
\textit{"eye of the heaven"} meaning the sun. 

\begin{verse}[\versewidth]
{\fontverse
But thy eternal summer shall not \underLine{f}ade, \\
Nor lose possession of that \underLine{f}air thou ow’st; \\
Nor \underLine{sh}all death brag thou wander’st in his \underLine{sh}ade, \\
When in eternal lines to time thou grow’st:
}
\end{verse}
But your beauty and glow will not fade, not even death can
kill your memory, you will live eternally because of these
lines (poem). \textbf{Alliteration} in first and second line
(repetition  of f) and third line (repetition of sh). \textbf{Personification} in
\textit{"nor shall death brag"}. \textbf{Metaphor} in \textit{"thy eternal summer"} his 
beloved compared to eternal summer. \textbf{Hyperbole} in \textit{"eternal summer shall not fade"}
exaggerate the eternity of his beloved's beauty.

\begin{verse}[\versewidth]
{\fontverse
So long as men can breathe or eyes can see,\\
So long \underLine{l}ives this, and this gives \underLine{l}ife to thee.
}
\end{verse}
As long as there is humans and they can see and read,
you will live eternally through this poem.


\subsubhead{Sonnet 18 Summary}
In this sonnet Shakespeare compares between summer and his beloved, saying that summer is lovely and temperate
but short and sometimes too hot. His beloved however is 
lovelier and more temperate and does not fade, as long as 
people can see and read, this sonnet will make his beloved eternal. The 
sonnet is \underLine{about Shakespeare patron}.


\end{document}
