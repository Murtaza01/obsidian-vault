\documentclass[12pt, a4paper]{article}
\usepackage{polyglossia}
\usepackage{geometry}
\usepackage{lua-ul}
\usepackage{color,soul}
\usepackage{verse}
\usepackage{xcolor}
\usepackage{hyperref}

\geometry{bottom=1in, top=1in}
\setlength\parindent{0pt}
\newcommand{\attrib}[1]{\nopagebreak{\raggedleft\footnotesize #1\par}}

%fonts
\setmainfont{Alegreya}
\newfontfamily\fontverse{Alegreya-Bold.otf} 
\newfontfamily\fonthead{Cinzel Decorative}


%sections
\newcommand{\head}[1]{
  \phantomsection
  \section*{\centering{\fonthead{#1}}}
  \addcontentsline{toc}{section}{#1}
}


\newcommand{\subhead}[1]{
  \phantomsection
  \subsection*{#1}
  \addcontentsline{toc}{subsection}{#1}
}

\newcommand{\subsubhead}[1]{
  \phantomsection
  \subsubsection*{#1}
  \addcontentsline{toc}{subsubsection}{#1}
}

\newcommand{\poemhead}[1]{
  \phantomsection
  \subsubsection*{\centering{\large{#1}}}
  \addcontentsline{toc}{subsubsection}{#1}
}

\begin{document}

\head{Historical Background of the Sixteenth Century}

\subhead{Major Events}

\begin{enumerate}

  \item \textbf{The fall of Constantinople}\smallbreak
The fall of the city lead many Greek scholars and their manuscripts to settle down
in Italy, because Italy was a great place for learning. The scholars and their teaching 
had a huge impact on Italy and its revival of
classical learning.
  
  \item \textbf{Geographical Discoveries}\smallbreak
The fall of Constantinople led to the loss of the old routes for spices, silk and
precious stones. As a result there were attempts to find new routs, this led to the discovery
of both \underLine{Cape route to India} and \underLine{The New World (America)}.

  \item \textbf{The Invention of Printing}\smallbreak
In the late 15th books were written by hand and they were expensive, only rich
people was able to afford them, \hl{in 1445} this changed. The Invention of printing allowed 
the widespread of knowledge to most people, books were cheap and available without much effort.

  \item \textbf{The Copernican system}\smallbreak
Copernicus work changed the idea held; that the earth is the center of the universe
and that the sun and other planets revolve around it. Copernicus instead advocated 
that the earth and other planets revolve around the sun.

  \item \textbf{The Reformation}\smallbreak
By the end of the Middle Ages many thought the catholic church needed reformation
because of the growing wealth of the clergy and the moral shortcomings to some of them.
The Reformation of the church included: 

\begin{itemize}
\item Humanism and the Renaissance.
\item The invention of printing.
\item The reaction of princes against the authority of the pope
\end{itemize}

\subhead{The Renaissance}

It is the most significant movement of the 16th century, unlike the middle ages, it
was conscious of itself, it knew that it was the Renaissance.
It was marked by the growing interest of the part of scholars in the language and
literature of the classical worlds of Greece and Roman. Italy was the center
of the Renaissance because:

\begin{itemize}

\item its geographical position close to Greece Egypt and Arabic empire.
\item its traditional Roman law and government.
\item its material prosperity and peace
  
\end{itemize}

\end{enumerate}

\head{Sixteenth Century Poetry}

In the 16th century lyrics in its all kind were popular especially songs, madrigals and
\hl{lyrical poems}. The major poets at that time were: \underLine{Sir Thomas Wyatt.},
\underLine{Henry Howard (Earl of Surrey)}, \underLine{Philip Sidney}, \underLine{William Shakespeare},
\underLine{Edmund Spenser} and \underLine{Christopher Marlowe}.\medbreak

\textbf{Lyrical Poetry:}Type of poetry that expresses personal feelings in a rhyming and short way.
It is \hl{personal, short and musical}.

\subhead{Geoffrey Chaucer}

One of the greatest poets in the Middle Ages. Was called the father of English poetry.
He influenced later poets especially English poets in the 16th century. His influence 
appeared in:

\begin{itemize}

\item His employment of rhyme and regular metre.
\item His huge contribution to middle English.
\item His impact on native literature.
  
\end{itemize}


\subhead{Sonnet}

\hl{The sonnet is 14 line poem written in iambic pentameter}. It was the
most important form of poetry in the 16th century. The sonnet originated in Italy in the 13th century by
\textbf{Petrarch} and \textbf{Dante}. The
English sonnet was credited to \textbf{Sir Thomas Wyatt} and \textbf{Henry Howard} in the early
16th century. Sonnets were mostly about love, Petrarch wrote sonnets to his mistress \underLine{Laura}.

\subsubhead{Petrarch Sonnet}

Or Italian sonnet. Were composed of 14 lines of \underLine{octave and sestet} rhyming
\underLine{abba,abba}. The climax was in the octave and sometimes the octave 
were divided into two quatrain. The structure would look like this:

\begin{itemize}
  
\item \textbf{Octave:} The first 8 lines of the sonnet and it is sometimes divided into 2
  \textbf{Quatrain} (4 lines).
\item \textbf{Sestet:} The last 6 lines of the sonnet, it is sometimes divided into 2 \textbf{tercets}
(3 lines).

\end{itemize}

\subsubhead{English Sonnet}

Henry Howard was one of the first to modify the Italian Scheme of the sonnet,
but Shakespeare was the best to apply it. This is why the  English Sonnet was called
\underLine{The Shakespearean Sonnet}. English sonnet consisted of \underLine{3 quatrains and couplet}
rhyming \underLine{abab,cdcd} and the climax or the solution were in the couplet.

\subhead{Themes of The Sixteenth Century}\bigbreak

\begin{enumerate}

  \item \textbf{Courtly Love}\smallbreak
Love was treated as a kind of god to be worshiped, poets idealized their lovers and
would indulge themselves in \underLine{unrequited love}.
These poets would love unattainable females which were of high class and sometimes married. 

  \item \textbf{Time}\smallbreak
Time was an important theme, it manifested in the idea of youth and the saying
“carpe florem” meaning (size the flower), it represent the poet desire to for youth
and the beauty that comes from it.

  \item \textbf{Death}\smallbreak
Closely connected to time, is the theme of immortality and the fear of death,
there were two reaction from poets about it: It drove some poets to live in the 
moment following the motto \underLine{“carpe
diem”} meaning (size the day), other took reckless risk and attacked the wicked.

  \item \textbf{Dreams and Sleep}
  
\end{enumerate}


\subhead{Poetic Devices of The Sixteenth Century}\bigbreak


\begin{itemize}

\item \textbf{Metaphor:} Is a figure of speech that compare between two things that isn’t
literally true in order to make a resemblance.\smallbreak
\textbf{Ex:} She is flower.

\item \textbf{Metonymy: } Is a figure of speech where an object name is replaced by an-
other closely connected to it.\smallbreak
\textbf{Ex:} What is your favourite \textit{dish}?\smallbreak
Here dish means food.

\item \textbf{Synecdoche:} A figure of speech that uses a part of something to refer to the
whole.\smallbreak
\textbf{Ex:} We Need more \textit{hands} to finish the job.\smallbreak
Here hands refers to men.

\item  \textbf{Apostrophe:} A figure of speech that is used to address a non human entity
(object) or someone that cannot reply.\smallbreak
In Sonnet 1 Edmund uses Apostrophe to describe the pages and rhymes.

\item  \textbf{Oxymoron:} A figure of speech that combines contradictory words with
opposing meanings.\smallbreak
\textbf{Ex:} Bitter sweet.

\item  \textbf{Alliteration:} The repetition of the same letter in the beginning of words in
a single line.\smallbreak
\textbf{Ex:} “Yet \underLine{m}ay I, by no \underLine{m}eans, \underLine{m}y wearied \underLine{m}ind”.

\item  \textbf{Consonance: } The repetition of the same letter in the mid or end of words
in a single line.\smallbreak

\textbf{Ex:} “Who lis\underLine{t} her hun\underLine{t}, I put him ou\underLine{t} of doub\underLine{t}”.

\end{itemize}


\subhead{Images of The Sixteenth Century}\bigbreak

\begin{enumerate}

  \item \textbf{Images of the sea}\smallbreak
The sea was the most important image of the 16th century because of the discovery
that were made through the sea especially the discovery of the new world
(America).
  \item \textbf{Life as a stage}\smallbreak
Stage was a very important image in the 16th century, poets would compare our
life to a play, happiness and sadness to comedy and tragedy, it was widely used
because of the popularity of plays and how similar life can be to a play.

  \item \textbf{Stars}\smallbreak
Stars was an important image of the 16th century, people back then depended
on stars to guide them through their joineries through the wide open seas.

  \item \textbf{War}.
  \item \textbf{Imprisonment}.
  \item \textbf{Diseases}.
  \item \textbf{Nature}.

\end{enumerate}

\head{Sir Thomas Wyatt}

Thomas was responsible of introducing the Italian sonnet to English poetry, his
influence was major especially in his use of two forms of poetry:

\begin{itemize}
  \item \textbf{ottava rima:} A rhyming stanza of eight lines.
  \item  \textbf{terza rima:} A rhyming stanza of three lines.
  
\end{itemize}

\textbf{Stanza:} A group of rhyming lines separated from others in a poem.


%\poemhead{To His Lady}
%\settowidth{\versewidth}{}
%\begin{verse}[\versewidth]
%{\fontverse
%
%} 
%\end{verse}

\end{document}
