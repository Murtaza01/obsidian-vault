\documentclass[12pt, a4paper]{article}
\usepackage{polyglossia}
\usepackage{geometry}
\usepackage{lua-ul}
\usepackage{color,soul}
\usepackage{verse}
\usepackage{xcolor}
\usepackage{hyperref}

\geometry{bottom=1in, top=1in}
\setlength\parindent{0pt}
\newcommand{\attrib}[1]{\nopagebreak{\raggedleft\footnotesize #1\par}}

%fonts
\setmainfont{Alegreya}
\newfontfamily\fontverse{Alegreya-Bold.otf}
\newfontfamily\fonthead{Cinzel Decorative}
\newfontfamily\fontsub{Literata}

%sections
\newcommand{\head}[1]{
  \phantomsection
  \section*{\centering{\fonthead{#1}}}
  \addcontentsline{toc}{section}{#1}
}


\newcommand{\poemhead}[1]{
  \phantomsection
  \subsection*{\centering{\large{\fontsub{#1}}}}
  \addcontentsline{toc}{subsection}{#1}
}


\newcommand{\subhead}[1]{
  \phantomsection
  \subsection*{\fontsub{#1}}
  \addcontentsline{toc}{subsection}{#1}
}

\newcommand{\subsubhead}[1]{
  \phantomsection
  \subsubsection*{\fontsub{#1}}
  \addcontentsline{toc}{subsubsection}{#1}
}

\begin{document}

\head{General Characteristics of the Sixteenth Century}

\subhead{Major Events}

\begin{enumerate}

  \item \textbf{The fall of Constantinople}\smallbreak
The fall of the city lead many Greek scholars and their manuscripts to settle down
in Italy, because Italy was a great place for learning. The scholars and their teaching 
had a huge impact on Italy and its revival of
classical learning.
  
  \item \textbf{Geographical Discoveries}\smallbreak
The fall of Constantinople led to the loss of the old routes for spices, silk and
precious stones. As a result there were attempts to find new routs, this led to the discovery
of both \underLine{Cape route to India} and \underLine{The New World (America)}.

  \item \textbf{The Invention of Printing}\smallbreak
In the late 15th books were written by hand and they were expensive, only rich
people was able to afford them, \hl{in 1445} this changed. The Invention of printing allowed 
the widespread of knowledge to most people, books were cheap and available without much effort.

  \item \textbf{The Copernican system}\smallbreak
Copernicus work changed the idea held; that the earth is the center of the universe
and that the sun and other planets revolve around it. Copernicus instead advocated 
that the earth and other planets revolve around the sun.

  \item \textbf{The Reformation}\smallbreak
By the end of the Middle Ages many thought the catholic church needed reformation
because of the growing wealth of the clergy and the moral shortcomings to some of them.
The Reformation of the church included: 


\begin{itemize}
\item Humanism and the Renaissance.
\item The invention of printing.
\item The reaction of princes against the authority of the pope
\end{itemize}

\end{enumerate}

\subhead{The Renaissance}

It is the most significant movement of the 16th century, unlike the middle ages, it
was conscious of itself, it knew that it was the Renaissance.
It was marked by the growing interest of the part of scholars in the language and
literature of the classical worlds of Greece and Roman. \underLine{Italy was the center
of the Renaissance} because of:

\begin{itemize}

\item Its geographical position close to Greece Egypt and Arabic empire.
\item Its traditional Roman law and government.
\item Its material prosperity and peace.
  
\end{itemize}


\head{Sixteenth Century Poetry}

In the 16th century lyrics in its all kind were popular especially songs, madrigals and
\hl{lyrical poems}. The major poets at that time were: \underLine{Sir Thomas Wyatt.},
\underLine{Henry Howard (Earl of Surrey)}, \underLine{Philip Sidney}, \underLine{William Shakespeare},
\underLine{Edmund Spenser} and \underLine{Christopher Marlowe}.\medbreak

\textbf{Lyrical Poetry:}Type of poetry that expresses personal feelings in a rhyming and short way.
It is \hl{personal, short and musical}.\medbreak

\textbf{Iambic  Pentameter:} A line of a poem composed of 10 syllables ( \underLine{five iambic feet}).\medbreak
\textbf{Iambic Hexameter (\underLine{Alexandrine}):} A line of a poem composed of 12 syllables
( \underLine{six iambic feet}).\medbreak

\subhead{Geoffrey Chaucer}

One of the greatest poets in the Middle Ages. Was called the father of English poetry.
He influenced later poets especially English poets in the 16th century. His influence 
appeared in:

\begin{itemize}

\item His employment of rhyme and regular metre.
\item His huge contribution to middle English.
\item His impact on native literature.
  
\end{itemize}


\subhead{Sonnet}

\hl{The sonnet is 14 line poem written in iambic pentameter}. It was the
most important form of poetry in the 16th century. The sonnet originated in Italy in the 13th century by
\textbf{Petrarch} and \textbf{Dante}. The
English sonnet was credited to \textbf{Sir Thomas Wyatt} and \textbf{Henry Howard} in the early
16th century. Sonnets were mostly about love, Petrarch wrote sonnets to his mistress \underLine{Laura}.

\subsubhead{Petrarch Sonnet}

Or Italian sonnet. Were composed of 14 lines of \underLine{octave and sestet} rhyming
\underLine{abba,abba}. The climax was in the octave and sometimes the octave 
were divided into two quatrain. The structure would look like this:

\begin{itemize}
  
\item \textbf{Octave:} The first 8 lines of the sonnet and it is sometimes divided into 2
  \textbf{Quatrain} (4 lines).
\item \textbf{Sestet:} The last 6 lines of the sonnet, it is sometimes divided into 2 \textbf{tercets}
(3 lines).

\end{itemize}

\subsubhead{English Sonnet}

Henry Howard was one of the first to modify the Italian Scheme of the sonnet,
but Shakespeare was the best to apply it. This is why the  English Sonnet was called
\underLine{The Shakespearean Sonnet}. English sonnet consisted of \underLine{3 quatrains and couplet}
rhyming \underLine{abab,cdcd} and the climax or the solution were in the couplet.

\subhead{Themes of The Sixteenth Century}\bigbreak

\begin{enumerate}

  \item \textbf{Courtly Love}\smallbreak
Love was treated as a kind of god to be worshiped, poets idealized their lovers and
would indulge themselves in \underLine{unrequited love}\footnote{\, one-sided love.}.
These poets would love unattainable females which were of high class and sometimes married. 
Most of the poets in the sixteenth century wrote about courtly love. 

  \item \textbf{Time}\smallbreak
Time was an important theme, it manifested in the idea of youth and the saying
“carpe florem\footnote{\, seize the flower. }", it represent the poet desire to for youth
and the beauty that comes from it.

  \item \textbf{Death}\smallbreak
Closely connected to time, is the theme of immortality and the fear of death,
there were two reaction from poets about it: It drove some poets to live in the 
moment following the motto \underLine{“carpe
diem\footnote{\, seize the day/moment.}"}, other took reckless risk and attacked the wicked.

  \item \textbf{Dreams and Sleep}
  
\end{enumerate}


\subhead{Poetic Devices of The Sixteenth Century}\bigbreak


\begin{itemize}

\item \textbf{Metaphor:} Is a figure of speech that compare between two things that isn’t
literally true in order to make a resemblance.\smallbreak
\textbf{Ex:} She is flower.

\item \textbf{Metonymy: } Is a figure of speech where an object name is replaced by an-
other closely connected to it.\smallbreak
\textbf{Ex:} What is your favourite \textit{dish}?\smallbreak
Here dish means food.

\item \textbf{Synecdoche:} A figure of speech that uses a part of something to refer to the
whole.\smallbreak
\textbf{Ex:} We Need more \textit{hands} to finish the job.\smallbreak
Here hands refers to men.

\item  \textbf{Apostrophe:} A figure of speech that is used to address a non human entity
(object) or someone that cannot reply.\smallbreak
In Sonnet 1 Edmund uses Apostrophe to describe the pages and rhymes.

\newpage
\item  \textbf{Oxymoron:} A figure of speech that combines contradictory words with
opposing meanings.\smallbreak
\textbf{Ex:} Bitter sweet.

\item  \textbf{Alliteration:} The repetition of the same letter in the beginning of words in
a single line.\smallbreak
\textbf{Ex:} “Yet \underLine{m}ay I, by no \underLine{m}eans, \underLine{m}y wearied \underLine{m}ind”.

\item  \textbf{Consonance: } The repetition of the same letter in the mid or end of words
in a single line.\smallbreak

\textbf{Ex:} “Who lis\underLine{t} her hun\underLine{t}, I put him ou\underLine{t} of doub\underLine{t}”.

\end{itemize}


\subhead{Images of The Sixteenth Century}\bigbreak

\begin{enumerate}

  \item \textbf{Images of the sea}\smallbreak
The sea was the most important image of the 16th century because of the discovery
that were made through the sea especially the discovery of the new world
(America). The most popular poems that used images of the sea were:

\begin{itemize}
  
  \item The Galley By Thomas Wyatt.
  \item Sonnet 34 By Edmund Spenser.
  
\end{itemize}

  \item \textbf{Life as a stage}\smallbreak
Stage was a very important image in the 16th century, poets would compare our
life to a play, happiness and sadness to comedy and tragedy, it was widely used
because of the popularity of plays and how similar life can be to a play.
The most popular poems that used the images of stage were:


\begin{itemize}
  
  \item Sonnet 54 By Edmund Spenser.
  \item What is our Life By Walter Raleigh.
  
\end{itemize}

  \item \textbf{Stars}\smallbreak
Stars was an important image of the 16th century, people back then depended
on stars to guide them through their joineries through the wide open seas.

  \item \textbf{War}.
  \item \textbf{Imprisonment}.
  \item \textbf{Diseases}.
  \item \textbf{Nature}.

\end{enumerate}

\head{Sir Thomas Wyatt}

Thomas was responsible of introducing the Italian sonnet to English poetry, his
influence was major especially in his use of two forms of poetry:

\begin{itemize}
  \item \textbf{ottava rima:} A rhyming stanza of eight lines.
  \item  \textbf{terza rima:} A rhyming stanza of three lines.
  
\end{itemize}

\textbf{Stanza:} A group of rhyming lines separated from others in a poem.


\poemhead{To His Lady}
\settowidth{\versewidth}{Madam, withouten many words}
\begin{verse}[\versewidth]
{\fontverse
Madam, withouten many words \\
\vin Once I am sure ye will or no ...\\
And if ye will, then leave your bourds\\
\vin And use your wit and show it so,
} 
\end{verse}

Madam, without many words and playing around, tell me if you are into 
me or not. If you are, show it to me. The poet is addressing his beloved
and asking her to just tell him if she is interested or not.

\begin{verse}[\versewidth]
{\fontverse
  And with a beck\footnote{\, node or gesture.} ye shall me call; \\
\vin   And if of one that burneth alway \\
Ye have any pity at all, \\
\vin  Answer him fair with yea or nay.
} 
\end{verse}

And with a node from you i will come. And if you have any pity for someone
who is burning for your love, then answer him straight; yes or no. 



\begin{verse}[\versewidth]
{\fontverse
  If it be yea, I shall be fain\footnote{\, happy or glad.}\\
\vin   If it be nay, friends as before;\\
Ye shall another man obtain,\\
\vin   And I mine own and yours no more.
} 
\end{verse}

If yes then I will be happy, if no then we stay friends
and you shall find someone else. Then I will no longer want you.


\subsubhead{\textit{To His Lady} Summary}

In this poem the poet asks his beloved a simple question: yes or no. He is asking her
if she wants him or not, expressing this in a very simple and direct way. The theme 
of the poem is \hl{courtly love}.

\poemhead{Farewell}
\settowidth{\versewidth}{What should I say,}
\begin{verse}[\versewidth]
{\fontverse
What should I say, \\
Since faith is dead,\\
And truth away\\
From you is fled?\\
Should I be led\\
With doubleness\footnote{\, hypocrisy. Being two-faced.}?\\
Nay\footnote{\, stronger form of no.}, nay, mistress! 
} 
\end{verse}

I have no words for you since you have abandoned honesty and trust.
Should i be led by your hypocrisy? No way mistress. The poet is expressing
his frustration with his beloved after she killed the trust between them 
and lost her honesty. He tells her that he is no longer fooled by her 
deception. 

\begin{verse}[\versewidth]
{\fontverse
I promised you,\\
And you promised me,\\
To be as true\\
As I would be.\\
But since I see\\
Your double heart\footnote{\, loving two people at the same time, being a hypocrite.},\\
Farewell my part! 
} 
\end{verse}

We promised each other to love one another. But now i see your 
deception, i will no longer stay. The poet say that he will leave 
because his beloved love two people at the same time and lying to them.

\begin{verse}[\versewidth]
{\fontverse
Though for to take\\
It is not my mind,\\
But to forsake\\
One so unkind \\
And as I find,\\
So will I trust:\\
Farewell, unjust! 
} 
\end{verse}

I no longer want to be with someone so cruel and unkind. Goodbye unjust.


\begin{verse}[\versewidth]
{\fontverse
Can ye say nay?\\
But you said\\
That I alway\\
Should be obeyed?\\
And thus betrayed\\
Or that I wiste—\\
Farewell, unkissed. 
} 
\end{verse}

How could you betray me? You said you will always love me. Goodbye! You don't even deserve a kiss.
The poet asks a \textbf{rhetorical question}\footnote{\, a question that doesn't need an answer and is asked to create dramatic effect.}
to express his amazement of his beloved
betrayal.

\subsubhead{\textit{Farewell} Summary}

In this poem the poet says goodbye to his beloved because she 
betrayed him, she loved someone else and lied to him. The poet 
express his pain in a frustrated and sad tone. The theme
of the poem is \hl{courtly love}.


\poemhead{An Appeal}
\settowidth{\versewidth}{And wilt thou leave me thus?}
\begin{verse}[\versewidth]
{\fontverse
And wilt thou leave me thus?\\
Say nay, say nay, for shame,\\
To save thee from the blame\\
Of all my grief and grame;\\
And wilt thou leave me thus?\\
Say nay, say nay!
} 
\end{verse}

Will you really leave me like this? Say no, because its a shameful act and
you will be guilty of being the cause of my misery. The poet asks his beloved a \textbf{rhetorical question}
in \textit{"And wilt thou leave me thus?"} to express his wonder and frustration.

\begin{verse}[\versewidth]
{\fontverse
And wilt thou leave me thus,\\
That hath loved thee so long\\
In wealth and woe among?\\
And is thy heart so strong\\
As for to leave me thus?\\
Say nay, say nay!

} 
\end{verse}

Will you really leave me like this? I have loved you for so long, in good time and bad ones.
Is your heart really that strong to leave me like this? Say no.

\begin{verse}[\versewidth]
{\fontverse
And wilt thou leave me thus,\\
That hath given thee my heart\\
Never for to depart,\\
Nother for pain nor smart;\\
And wilt thou leave me thus?\\
Say nay, say nay!
} 
\end{verse}

Will you leave me like this? I have given you my heart and promised you to never leave, not
for pain nor for suffering. 

\begin{verse}[\versewidth]
{\fontverse
And wilt thou leave me thus\\
And have no more pity\\
Of him that loveth thee?\\
Alas, thy cruelty!\\
And wilt thou leave me thus?\\
Say nay, say nay!
} 
\end{verse}

Will you leave me like this? And have no pity for someone that love you so much?
Alas how cruel you are. The poet expresses his pain and heartbreak because of
his beloved abandonment. He decided to leave her by saying \textit{"alas,they cruelty"}.

\subsubhead{\textit{An Appeal} Summary}


In this poem the poet asks a rhetorical question, saying: \textit{“after all, you leave me
like this?”} to express his wonder and surprise. He feels heartbroken and frustrated.
The poet uses reputation to emphasis his feelings and message and for music.
In the end of the poem the poet leaves his beloved. The theme of the poem is
\hl{courtly love}.

\poemhead{The Galley}
\settowidth{\versewidth}{My galley, chargèd with forgetfulness,}
\begin{verse}[\versewidth]
{\fontverse
My galley, chargèd with forgetfulness,\\
Thorough sharp seas in winter nights doth pass\\
'Tween rock and rock; and eke mine en'my, alas,\\
That is my lord, steereth with cruelness;
} 
\end{verse}

My ship is weighed down with neglect, it struggles through cold nights
and dangerous seas. It sail through the deadly rocks and between my
enemy. This cruel enemy is my master, he controls the ship. The poet
uses a \textbf{metaphor (Images of the sea)} to compare his life to a ship.
He says that he suffers through cold nights and lonely times. His 
cruel beloved controls his life (the ship).

\begin{verse}[\versewidth]
{\fontverse
And every owre a thought in readiness,\\
As though that death were light in such a case.\\
An endless wind doth tear the sail apace\\
Of forced sighs and trusty fearfulness.
} 
\end{verse}

And every oar is an urgent call, as if death would be easier than enduring
this journey. A storm never ending attack the ship and rip off the sail.
The poet says that he is never rested through this journey of love. He 
feels that his suffering is endless and death would be better.


\begin{verse}[\versewidth]
{\fontverse
A rain of tears, a cloud of dark disdain,\\
Hath done the weared cords great hinderance;\\
Wreathèd with error and eke with ignorance.\\
The stars be hid that led me to this pain;
} 
\end{verse}

A cloudy storm that rains over my ship damages the ropes 
and hide the star that guide me through my journey. The poet says
that problems between him and his beloved damages their relationship
and takes her away from him (she is his guiding star).


\begin{verse}[\versewidth]
{\fontverse
Drownèd is Reason that should me comfort,\\
And I remain despairing of the port.
} 
\end{verse}

My mind (reason) which should comfort me is drowned. And i
will remain desperate and without hope.

\subsubhead{\textit{The Galley} Summary}

In this poem the poet uses a \hl{metaphor (images of the sea)} to compare
the journey of the sea to the journey of love. The sea is dangerous,
filed with deadly rocks and storms, even worse there is enemies in the 
sea. Similarly the journey of love is filed with emotional pain and 
obstacles and the beloved is sometimes the enemy. The poet says that
he is the ship that is lost in the sea and the star that guides him is hid by a cloud
(a metaphor for problems). He concludes the poem in despair and hopelessness.
This poem theme is \hl{courtly love}.

\poemhead{The Hind}
\settowidth{\versewidth}{Whoso list to hunt, I know where is an hind,}
\begin{verse}[\versewidth]
{\fontverse
  Whoso list to hunt, I know where is an hind\footnote{\, female deer.},\\
But as for me, alas, I may no more.\\
The vain travail hath wearied me so sore,\\
I am of them that farthest cometh behind.
} 
\end{verse}

Who are in the mood to hunt? I know where is a female. As for me
I no longer want to, the pointless pursuit has exhausted me.
The poet says that he no longer want to purist females because 
it is pointless. 


\begin{verse}[\versewidth]
{\fontverse
Yet may I by no means my wearied mind\\
Draw from the deer, but as she fleeth afore\\
Fainting I follow. I leave off therefore,\\
Sithens in a net I seek to hold the wind.
} 
\end{verse}

Even though I try, I keep thinking about her. As I chase her I faint and
give up, because trying to catch her is like trying to catch the wind in a net.


\begin{verse}[\versewidth]
{\fontverse
Who list her hunt, I put him out of doubt,\\
As well as I may spend his time in vain.\\
And graven with diamonds in letters plain\\
There is written, her fair neck round about:
} 
\end{verse}

Who wants to purist her? I assure you it is a waste of time
because there is a writing around her neck engraved with diamond.


\begin{verse}[\versewidth]
{\fontverse
  Noli me tangere\footnote{\, Do not touch me}, for Caesar's I am,\\
And wild for to hold, though I seem tame. 
} 
\end{verse}

Do not touch me because i belong to Caesar. And even though 
i seem gentle, i am wild. 

\subsubhead{\textit{The Hind} Summary}

In this poem the poet refer to his beloved as Hind. He says that he is tired of
chasing her and that it is a pointless pursuit, because she already belongs
to someone else (Caesar). He says that even though he stopped the chase he still
thinks about her. He used Hind as a metaphor for females 
because they are beautiful and graceful, and because they used to hunt
them in his time. 

\head{Henry Howard}

\poemhead{To His Lady}
\settowidth{\versewidth}{Set me whereas the sun doth parch the green}
\begin{verse}[\versewidth]
{\fontverse
Set me whereas the sun doth parch the green\\
Or where his beams do not dissolve the ice,\\
In temperate heat where he is felt and seen;\\
In presence prest of people, mad or wise; 
} 
\end{verse}

\begin{verse}[\versewidth]
{\fontverse
Set me in high or yet in low degree,\\
In longest night or in the shortest day,\\
In clearest sky or where clouds thickest be,\\
In lusty youth or when my hairs are gray.
} 
\end{verse}

\begin{verse}[\versewidth]
{\fontverse
Set me in heaven, in earth, or else in hell;\\
In hill, or dale, or in the foaming flood;\\
Thrall or at large, alive whereso I dwell,\\
Sick or in health, in evil fame or good:
} 
\end{verse}

\begin{verse}[\versewidth]
{\fontverse
Hers will I be, and only with this thought\\
Content myself although my chance be nought.
} 
\end{verse}


\poemhead{Spring}
\settowidth{\versewidth}{The soote season, that bud and bloom forth brings,}
\begin{verse}[\versewidth]
{\fontverse
The soote season, that bud and bloom forth brings,\\
With green hath clad the hill and eke the vale;\\
The nightingale with feathers new she sings,\\
The turtle to her make hath told her tale.
} 
\end{verse}

\begin{verse}[\versewidth]
{\fontverse
Summer is come, for every spray now springs,\\
The hart hath hung his old head on the pale,\\
The buck in brake his winter coat he flings,\\
The fishes float with new repaired scale,
} 
\end{verse}

\begin{verse}[\versewidth]
{\fontverse
The adder all her slough away she slings,\\
The swift swallow pursueth the flyës smale,\\
The busy bee her honey now she mings—\\
Winter is worn that was the flowers' bale.
} 
\end{verse}

\begin{verse}[\versewidth]
{\fontverse
And thus I see, among these pleasant things\\
Each care decays, and yet my sorrow springs.
} 
\end{verse}

\head{Edmund Spenser}

He was widely known poet in the sixteenth century and was famous for inviting: 

\begin{itemize}

\item  \textbf{Spenserian  Stanza:} A nine-line stanza rhyming (abab bcbc c).
The first 8 lines are pentameter, and the last one is \underLine{hexameter}. Edmund used 
the Spenserian stanza in his epic poem \underLine{The Faerie Queene}.
\item \textbf{Spenserian Sonnet:} Is a poem of fourteen lines of iambic pentameter rhyming
(abab bcbc cdcd ee). This sonnet is made of 3 Quatrains that are \underLine{interlocking} in
rhyme and a concluding couplet.
  
\end{itemize}

\subhead{AMORETTI} 

\hl{It is a sonnet sequence dedicated to one person}.
Edmund wrote these sonnets for his wife \underLine{Elizabeth}. In sonnet sequence the sonnets are \underLine{
numbered not named}.
Amoretti means \underLine{little love messages or little cupid}.\medbreak


\textbf{Allegory: } Is a story with two levels of meaning, a surface one and a hidden (deep)
one.


\poemhead{Sonnet 1}
\settowidth{\versewidth}{Happy ye leaves when as those lilly hands,}
\begin{verse}[\versewidth]
{\fontverse
Happy ye leaves when as those lilly hands,\\
Which hold my life in their dead doing might\\
Shall handle you and hold in loves soft bands,\\
Lyke captives trembling at the victors sight.
} 
\end{verse}

\begin{verse}[\versewidth]
{\fontverse
And happy lines, on which with starry light,\\
Those lamping eyes will deigne sometimes to look\\
And reade the sorrowes of my dying spright,\\
Written with teares in harts close bleeding book.
} 
\end{verse}

\begin{verse}[\versewidth]
{\fontverse
And happy rymes bath’d in the sacred brooke,\\
Of Helicon whence she derived is,\\
When ye behold that Angels blessed looke,\\
My soules long lacked foode, my heavens blis.
} 
\end{verse}

\begin{verse}[\versewidth]
{\fontverse
 Leaves, lines, and rymes, seeke her to please alone,\\
Whom if ye please, I care for other none. 
} 
\end{verse}


\poemhead{Sonnet 34}
\settowidth{\versewidth}{Lyke as a ship that through the Ocean wyde,}
\begin{verse}[\versewidth]
{\fontverse
Lyke as a ship that through the Ocean wyde,\\
by conduct of some star doth make her way.\\
whenas a storme hath dimd her trusty guyde.\\
out of her course doth wander far astray:
} 
\end{verse}

\begin{verse}[\versewidth]
{\fontverse
So I whose star, that wont with her bright ray,\\
me to direct, with cloudes is ouercast,\\
doe wander now in darknesse and dismay,\\
through hidden perils round about me plast.
} 
\end{verse}

\begin{verse}[\versewidth]
{\fontverse
Yet hope I well, that when this storme is past\\
My Helice the lodestar of my lyfe\\
will shine again, and looke on me at last,\\
with louely light to cleare my cloudy grief,
} 
\end{verse}

\begin{verse}[\versewidth]
{\fontverse
Till then I wander carefull comfortlesse,\\
in secret sorow and sad pensiuenesse.
} 
\end{verse}



\poemhead{Sonnet 54}
\settowidth{\versewidth}{Of this worlds Theatre in which we stay,}
\begin{verse}[\versewidth]
{\fontverse
Of this worlds Theatre in which we stay,\\
My love lyke the Spectator ydly sits\\
Beholding me that all the pageants play,\\
Disguysing diversly my troubled wits.
} 
\end{verse}

\begin{verse}[\versewidth]
{\fontverse
Sometimes I joy when glad occasion fits,\\
And mask in myrth lyke to a Comedy:\\
Soone after when my joy to sorrow flits,\\
I waile and make my woes a Tragedy.
} 
\end{verse}

\begin{verse}[\versewidth]
{\fontverse
Yet she beholding me with constant eye,\\
Delights not in my merth nor rues my smart:\\
But when I laugh she mocks, and when I cry\\
She laughes, and hardens evermore her hart.
} 
\end{verse}

\begin{verse}[\versewidth]
{\fontverse
What then can move her? if not merth nor mone,\\
She is no woman, but a sencelesse stone. 
} 
\end{verse}


\head{Walter Raleigh}


\poemhead{What is our life?}
\settowidth{\versewidth}{WHAT is our life? The play of passion.}
\begin{verse}[\versewidth]
{\fontverse
What is our life? The play of passion.\\
Our mirth? The music of division:\\
Our mothers’ wombs the tiring-houses be,\\
Where we are dressed for life’s short comedy.
} 
\end{verse}

\begin{verse}[\versewidth]
{\fontverse
The earth the stage; Heaven the spectator is,\\
Who sits and views whosoe’er doth act amiss.\\
The graves which hide us from the scorching sun\\
Are like drawn curtains when the play is done.
} 
\end{verse}

\begin{verse}[\versewidth]
{\fontverse
Thus playing post we to our latest rest,\\
And then we die in earnest, not in jest.
} 
\end{verse}



\poemhead{The Nymph’s Reply to the Shepherd}
\settowidth{\versewidth}{If all the world and love were young,}
\begin{verse}[\versewidth]
{\fontverse
If all the world and love were young,\\
And truth in every Shepherd’s tongue,\\
These pretty pleasures might me move,\\
To live with thee, and be thy love.
} 
\end{verse}

\begin{verse}[\versewidth]
{\fontverse
Time drives the flocks from field to fold,\\
When Rivers rage and Rocks grow cold,\\
And Philomel becometh dumb,\\
The rest complains of cares to come.
} 
\end{verse}

\begin{verse}[\versewidth]
{\fontverse
The flowers do fade, and wanton fields,\\
To wayward winter reckoning yields,\\
A honey tongue, a heart of gall,\\
Is fancy’s spring, but sorrow’s fall.
} 
\end{verse}

\begin{verse}[\versewidth]
{\fontverse
Thy gowns, thy shoes, thy beds of Roses,\\
Thy cap, thy kirtle, and thy posies\\
Soon break, soon wither, soon forgotten:\\
In folly ripe, in reason rotten.
} 
\end{verse}

\begin{verse}[\versewidth]
{\fontverse
Thy belt of straw and Ivy buds,\\
The Coral clasps and amber studs,\\
All these in me no means can move\\
To come to thee and be thy love.
} 
\end{verse}

\begin{verse}[\versewidth]
{\fontverse
But could youth last, and love still breed,\\
Had joys no date, nor age no need,\\
Then these delights my mind might move\\
To live with thee, and be thy love.
} 
\end{verse}


\head{Christopher Marlowe}


\poemhead{The Passionate Shepherd to His Love}
\settowidth{\versewidth}{Come live with me and be my love,}
\begin{verse}[\versewidth]
{\fontverse
Come live with me and be my love,\\
And we will all the pleasures prove,\\
That Valleys, groves, hills, and fields,\\
Woods, or steepy mountain yields.
} 
\end{verse}

\begin{verse}[\versewidth]
{\fontverse
And we will sit upon the Rocks,\\
Seeing the Shepherds feed their flocks,\\
By shallow Rivers to whose falls\\
Melodious birds sing Madrigals.
} 
\end{verse}

\begin{verse}[\versewidth]
{\fontverse
And I will make thee beds of Roses\\
And a thousand fragrant posies,\\
A cap of flowers, and a kirtle\\
Embroidered all with leaves of Myrtle;
} 
\end{verse}

\begin{verse}[\versewidth]
{\fontverse
A gown made of the finest wool\\
Which from our pretty Lambs we pull;\\
Fair lined slippers for the cold,\\
With buckles of the purest gold;
} 
\end{verse}


\begin{verse}[\versewidth]
{\fontverse
A belt of straw and Ivy buds,\\
With Coral clasps and Amber studs:\\
And if these pleasures may thee move,\\
Come live with me, and be my love.
} 
\end{verse}


\begin{verse}[\versewidth]
{\fontverse
The Shepherds’ Swains shall dance and sing\\
For thy delight each May-morning:\\
If these delights thy mind may move,\\
Then live with me, and be my love.
} 
\end{verse}



\end{document}
