\head{Historical Background of the Seventeen Century}

\subhead{King x Parliament}

Both \textit{James I} and his son \textit{Charles I} thought that kings ruled 
by \underLine{Divine Right}. They abused the power of ruling, made illegal taxes
on working people. They were in constant disputes with the Parliament and tried 
many times to rule without them which led to the civil war.

\subhead{Civil War (1642-1649)}

The civil war happened between \textit{Charles I} and his supporters 
(\underLine{the Royalist}) against the Parliament and their supporters which were
merchants and tradesmen, they were of the common people and were called (\underLine{the Roundheads}). 
With the help of Scotland the Parliament defeated the king and executed him in 1649.

\subhead{The Puritans}

In the seventeen century there were two major religious groups: the
\textit{Puritans} who were very strict and thought that all entertainment is
distasteful to God. And the \textit{Catholics} who wanted the Pope in Rome to 
be the head of the church.\medbreak

In the period (1649-1660) \underLine{Oliver Cromwell} (who was a Puritan) ruled England. He closed the 
theaters and other entertainments because he thought they were distraction
from the Bible. In the \underLine{Restoration period} after Oliver rule was over, 
people started to be more vulgar and indulgent as a reaction to the Puritan's strict rules.

\head{School of Ben Jonson}

The school of Ben or the \textit{Tribe of Ben} were group of poets who imitated
Ben Jonson style. They were Royalist and were called \underLine{Cavalier Poets} because they
supported king \textit{Charles I} against the Parliament. This group included: 
\underLine{Thomas Carew}, \underLine{Richard Lovelace}, 
\underLine{John Suckling}, \underLine{Robert Herrick}. \hl{Ben and his followers often wrote 
carpe diem\footnote{\, seize the day/moment. Its a philosophy that recognizes the brevity of life
and the need to live in the moment.} poems}.\medbreak

\subhead{Ben Jonson}

Playwright, critic and poet. Was part of the Royalist who supported 
king Charles I. He lived in the city, 
so most of \underLine{his work were about 
the city side} (urban side). He was influenced by classical 
literature (Greek and Roman) and imitated their style. He used
\textbf{classicism}\footnote{\, the following of Greek and Roman style and 
principles in literature.} 
in his form. \underLine{\textbf{John Donne} was his rival-artist}, his 
form was complex, scientific and included metaphysical elements, unlike
Ben's form which was simple and plain.


\subsubhead{Characteristics of Ben's School}

\begin{itemize}
  \item Clarity, brevity, simplicity and order.
  \item Realism and the use of controlled feelings.\footnote{\, describing things in realistic  manner and staying away 
    from exaggeration.}
  \item Logic and wit.
  \item Didacticism\footnote{\, is a type of literature that aims to teach}
    and instruction.
  \item Refinement of the classics. 
\end{itemize}


\subsubhead{Ben Jonson Poems} 
Ben favored the \hl{shorter forms} in his work, such as: \underLine{the epigram}
, \underLine{the epitaph}, \underLine{the elegy} and 
\underLine{the epistle}.\medbreak


\textbf{Epigram:} A short poem consisting of 2 line verse that teaches a moral lesson.\medbreak

\textbf{Elegy:} A poem mourning the death of a loved one.\medbreak

\textbf{Epitaph:} A short poem often written on tombstone
to honor the dead.\medbreak

\textbf{Epistle:} A short poem in the form of a letter.

\poemhead{On my First Son}
\settowidth{\versewidth}{Farewell, thou child of my right hand, and joy}
\begin{verse}[\versewidth]
{\fontverse
Farewell, thou child of my right hand, and joy;\\
My sin was too much hope of thee, lov'd boy. 
}
\end{verse}

Goodbye my child, you were my best thing and the thing that brought me joy.
My sin was that i had high expectation for you. Ben says "right 
hand" to mean best or favorite.

\begin{verse}[\versewidth]
{\fontverse
Seven years tho' wert lent to me, and I thee pay, \\
Exacted by thy fate, on the just day. 
}
\end{verse}

Seven years you were lent to me by fate, and then it took you on the 
exact day you were born (your birthday).

\begin{verse}[\versewidth]
{\fontverse
O, could I lose all father now! For why \\
Will man lament the state he should envy? 
}
\end{verse}

No one will ever call me "father" for you were my only child.
And why should i cry when i really should envy you.

\begin{verse}[\versewidth]
{\fontverse
To have so soon 'scap'd world's and flesh's rage, \\
And if no other misery, yet age?
}
\end{verse}

You have escaped the world too soon, escaped from your
body's trouble, and if you have lived you would suffer from old age.\bigbreak

\begin{verse}[\versewidth]
{\fontverse
Rest in soft peace, and, ask'd, say, "Here doth lie \\
Ben Jonson his best piece of poetry." 
}
\end{verse}

Here Ben says to his child "if they ask you (the angels), say that 
\textit{I (Ben)} lie here (in the grave)". His great pain makes him
feel as if he were dead.

\begin{verse}[\versewidth]
{\fontverse
For whose sake henceforth all his vows be such, \\
As what he loves may never like too much. 
}
\end{verse}


That i promise you my child i will never love anything like you 
again.\medbreak


\subsubhead{\textit{On my First Son} Summary}

In his epigram Ben mourns the death of his favourite and only child, the best thing
that ever happened to him. He says that fate \textit{lent him} his child
only for seven years and then it took it from him. Because of his great
pain he envies the dead. He concludes by promising that he will never love
anything like his child. \hl{This poem is an elegy to Ben's
son}. It is about \hl{honoring the father} and the moral lesson
is that \hl{nothing in life is ours}. Everything is given 
(lent) to us by God, whether it is money, children or health.

\poemhead{Song to Celia}
\settowidth{\versewidth}{Come, my Celia, let us prove}
\begin{verse}[\versewidth]
{\fontverse
Come, my Celia, let us prove\\
While we may, the sports of love;
}
\end{verse}

Come Celia let us experience the joys of love while we can.

\begin{verse}[\versewidth]
{\fontverse
Time will not be ours forever;\\
He at length our good will sever.
}
\end{verse}

Time will not be ours forever and will separate us from 
our health.  \textbf{personification} in \textit{"He at length"}
given time human attribute.

\begin{verse}[\versewidth]
{\fontverse
Spend not then his gifts in vain.\\
Suns that set may rise again;
}
\end{verse}

Don't waste the gift of time. The day that is over 
may be followed by a new day.

\begin{verse}[\versewidth]
{\fontverse
But if once we lose this light,\\
'Tis with us perpetual night.
}
\end{verse}

But if we lose the light of the sun (life), then 
we will face the night (death). The poet tries to warn 
his beloved about the importance of time.

\begin{verse}[\versewidth]
{\fontverse
Why should we defer our joys?\\
Fame and rumor are but toys.
}
\end{verse}

Why should we delay joys? (this is a \textbf{rhetorical question}
\footnote{\, a question that doesn't need an answer and asked to create dramatic effect.}).
Reputation is trivial. 

\begin{verse}[\versewidth]
{\fontverse
Cannot we delude the eyes\\
Of a few poor household spies,\\
Or his easier ears beguile,\\
So removed by our wile?
}
\end{verse}

Cannot we deceive the people around us? (a \textbf{rhetorical question}).
The poet tries to encourage his beloved to meet in secret.

\begin{verse}[\versewidth]
{\fontverse
'Tis no sin love's fruit to steal;\\
But the sweet theft to reveal.\\
To be taken, to be seen,\\
These have crimes accounted been.
}
\end{verse}


Its no sin to fall in love and act on it if it was in 
private. It is only a crime if we are seen.


\subsubhead{\textit{Song to Celia} Summary}

In this poem Ben invite his beloved to enjoy love while 
they have the time, it is a \hl{carpe diem} poem (seize the day).
He tries to convince her of the shortness of time and youth and how
they should seize it in private while they can. This poem is from the play Volpone.


\subhead{Robert Herrick}
A Royalist and Cavalier poet, part of the school of Ben Jonson,
he is the \underLine{best representative of Ben's school}.

\poemhead{Delight in Disorder}
\settowidth{\versewidth}{A sweet disorder in the dress. }
%i have some space here just to fix the width of the verse
\begin{verse}[\versewidth]
{\fontverse
A sweet disorder in the dress\\
Kindles in clothes a wantonness;\\
A lawn about the shoulders thrown\\
Into a fine distraction;\\
An erring lace, which here and there\\
Enthrals the crimson stomacher;\\
A cuff neglectful, and thereby\\
Ribands to flow confusedly;\\
A winning wave, (deserving note),\\
In the tempestuous petticoat;\\
A careless shoe-string, in whose tie\\
I see a wild civility:\\
Do more bewitch me, than when art\\
Is too precise in every part.
}
\end{verse}

\subsubhead{\textit{Delight in Disorder} Summary}

In this poem Robert tells his beloved that there is
some joy in disorder. He tells her that she doesn't 
need to look perfect, that imperfection is charming and natural.
He expresses this in \underLine{subject/predicate} manner where
the first 12 lines is subject and the last two lines is the 
predicate that describe the subject.

\poemhead{To the Virgins, to Make Much of Time}
\settowidth{\versewidth}{Gather ye rose-buds while ye may}
\begin{verse}[\versewidth]
{\fontverse
Gather ye rose-buds while ye may,\\
Old Time is still a-flying;\\
And this same flower that smiles today\\
Tomorrow will be dying.
} 
\end{verse}
seize your youth days while you can, time passes quickly and the
beautiful flower you see today will die tomorrow. \textbf{Metaphor} in \textit{"rose-buds"}
to mean youth. \textbf{Personification} in \textit{"old time is still a-flying"}.

\begin{verse}[\versewidth]
{\fontverse
The glorious lamp of heaven, the sun,\\
The higher he’s a-getting,\\
The sooner will his race be run,\\
And nearer he’s to setting.
} 
\end{verse}

The higher the sun gets, the nearer our death would be. Here the poet 
uses metaphor to compare our life to one day; the sunrise 
is when we are born, the noon is our youth, the afternoon is middle age and
the sunset is death. 

\begin{verse}[\versewidth]
{\fontverse
That age is best which is the first,\\
When youth and blood are warmer;\\
But being spent, the worse, and worst\\
Times still succeed the former. 
} 
\end{verse}

The best age to live is youth time, you are passionate and energetic
but when it is spent it will become worse and worse (when you get older)
and eventually time will surpass your youth. \textbf{Personification} in \textit{"time still succeed"}.

\begin{verse}[\versewidth]
{\fontverse
Then be not coy, but use your time,\\
And while ye may, go marry;\\
For having lost but once your prime,\\
You may forever tarry.
} 
\end{verse}

Then do not be shy, marry while you can because if
your prime years past, you may end up waiting forever.


\subsubhead{\textit{To the Virgins, to Make Much of Time} Summary}

In this poem Robert Herrick tries to encourage women to marry while they can
because their prime (youth) will not last forever.
Time is quick and when our youth passes we become worse because of old age. 
Robert uses a metaphor to compare between our life to one day;
it start by sunrise (being born) and end by sunset (death).
\hl{The poem is about carpe diem philosophy}.

\newpage
\poemhead{To Daffodils}
\settowidth{\versewidth}{Fair Daffodils, we weep to see}
\begin{verse}[\versewidth]
{\fontverse
Fair Daffodils, we weep to see\\
You haste away so soon;\\
As yet the early-rising sun\\
Has not attain'd his noon.
} 
\end{verse}

Beautiful daffodil\footnote{\, a type of flower that blooms in the spring time.} we 
are sad to see you die so soon, even before the arrival of noon. \textbf{Apostrophe}
is used to addresses \textit{"daffodil"}.

\begin{verse}[\versewidth]
{\fontverse
Stay, stay,\\
Until the hasting day\\
Has run\\
But to the even-song;\\
And, having pray'd together, we\\
Will go with you along.
} 
\end{verse}

Stay daffodil until the day that fades so quickly reaches noon. Wait until the evening 
prayer so we can pray together and then we can go (die). 

\begin{verse}[\versewidth]
{\fontverse
We have short time to stay, as you,\\
We have as short a spring;\\
As quick a growth to meet decay,\\
As you, or anything.\\
We die\\
As your hours do, and dry\\
Away,\\
Like to the summer's rain;\\
Or as the pearls of morning's dew,\\
Ne'er to be found again.
} 
\end{verse}

Here the poet uses simile\footnote{\, a figure of speech that compares between 
two things directly using "as" and "like".} to compare human life to:

\begin{itemize}
  \item \textbf{Daffodil}, that our time is as short as daffodil's.
  \item \textbf{Spring}, that our time is similar to spring; fleeing.
  \item \textbf{Summer rain}, that our time is as quick as the rain in summer.
  \item \textbf{Pearls of morning dew}, that our time is valuable.
\end{itemize}

The poet says that we have short time in this life and that we will eventually die. 
He compares it to daffodil, spring, summer rain and pearls morning's dew. 

\subsubhead{\textit{To Daffodils} Summary}
In this poem Robert addresses daffodil using \hl{Apostrophe}. He compares
our life to the daffodil's; short and fleeting, precious and beautiful.
He says that we have short time in this life, and we are heading for our quick
death that feels like hours. He uses simile repeatedly to emphasize his point.
The poem is a \underLine{contemplation of life} using \hl{carpe diem} philosophy, its message is to
\underLine{cherish the time we got}.

