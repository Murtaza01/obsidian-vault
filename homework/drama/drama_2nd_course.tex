\documentclass[12pt, a4paper]{article}
\usepackage{polyglossia}
\usepackage{geometry}
\usepackage{lua-ul}
\usepackage{color,soul}
\usepackage{xcolor}

\geometry{top=0.6in}
\setlength\parindent{0pt}

\setmainfont{Erode}
\newfontfamily\fonthead{LobsterTwo}
%\newfontfamily\fontpar{Erode}


\begin{document}

\begin{center}
{\fonthead
\huge{Twelfth Night}\\[0.2cm]
\Large{Shakespeare}\\[0.5cm]
\large{Done By: Birdman}\\[1cm]
}
\end{center}

\subsection*{\centering{{Questions}}}\bigbreak

\subsubsection*{Q1/Discuss Twelfth Night as a Romanic Comedy.}
Romantic comedy is a temporary hold on reality. It is a world of 
celebration but at the end there is return to normality.There is 
deviation from
the norm and restoration to it. In comedies the writer usually 
concentrates on social mistakes like: hypocrisy, cheating, lying. He
exaggerate them in a funny way that pushes the audience to laugh at them so as to avoid them. 
Through laughing the moral lesson is delivered. The main complication
rises from choosing desire over duty, but in the end the norm is
restored and characters accept to return to duty. The romantic comedy
of the play contains a love relationship between the characters, 
especially the hero of the story. Obstacles that are formed due to 
different troubles, the hero's struggles, or various other characters
must overcome these obstacles. As for the conclusion, it is happy for 
all the characters in general.\medbreak

\textbf{Short Answer:}\medbreak

Romantic comedy is a temporary hold from reality. There is deviation from
the norm and restoration to it. In comedies the writer concentrate on
social mistakes like: lying, cheating and hypocrisy. He exaggerate them
in a funny way to deliver the moral lesson. Romantic comedy contains
love relationships and obstacles that characters must overcome. As
for the ending, it is happy.

\subsubsection*{Q2/Comment on the love sick of Duke Orsino.}

Duke Orsino was in love with idea of love. He is in love with Olivia
and keeps sending messengers of love to her, but she reject them, yet he 
is too proud to accept rejection. He suffers from the symptoms of 
rejected love and wants to satisfy his "appetite" of love but cannot.
Although in the end it turns out that his love for Olivia is not genuine
because he chooses Viola as his beloved.

\subsection*{\centering{Identify, comment and give the dramatic value of the following quotation.}}

\subsubsection*{Quote 1/
  If music be the food of love, play on;\\
	Give me excess of it, that, surfeiting,\\
	The appetite may sicken, and so die.\\
	That strain again! it had a dying fall:\\
	O, it came o'er my ear like the sweet south,\\
	That breathes upon a bank of violets,\\
	Stealing and giving odour! Enough; no more:\\
	'Tis not so sweet now as it was before.\\
	O spirit of love! how quick and fresh art thou,\\
	That, notwithstanding thy capacity	10\\
	Receiveth as the sea, nought enters there,\\
	Of what validity and pitch soe'er,\\
	But falls into abatement and low price,\\
	Even in a minute: so full of shapes is fancy\\
	That it alone is high fantastical.
}
These lines are said by Duke Orsino  in the opening speech. He suffers because
his love to Countess Olivia is one sided. He depicts love as an "appetite" that
he cannot fulfill. He thinks that music may help feed his appetite, comparing
it to wind that comes over flowers and steals the odor. However he does not
enjoy it like before because he is love sick. He uses apostrophe to address the spirit of love
describing it as fresh, beautiful and quick but degrading, that when you
love someone you must forget about your dignity.

\subsubsection*{Quote 2/ 
  So please my lord. I might not be addmited\\ 
  But from her handmaid do return this answer:\\
  The element itself, till seven years' heat, \\
  Shall not behold her face at ample view;\\ 
  But like a cloistress she will veiled walk,\\ 
  And water once a day her chamber round \\ 
  With eye-offending brine---all this to season\\
  A brother's dead love, which she would keep fresh\\
  And lasting in her sad remembrance.
}

These lines are said by Valentine (Orsino's messenger) after he went
to Olivia. He bring a message from Olivia handmaiden that she will 
continue to mourn her dead brother and live like a nun without the 
society of men for seven years. Walking in veil and watering her 
chamber once a day to keep her brother's memory.

\subsubsection*{Quote 3/
  O, she that hath a heart of that fine frame\\
  To pay this debt of love but to a brother,\\
	How will she love, when the rich golden shaft\\
	Hath kill'd the flock of all affections else\\
	That live in her; when liver, brain and heart,\\
	These sovereign thrones, are all supplied, and fill'd\\
	Her sweet perfections with one self king!
}

These lines are said by Orisno to Valentine when Valentine told him
about Olivia decision to leave the society of men and mourn over her
dead brother for seven years. Orisno consider this an act of nobility and
loyalty and gets more attracted to Olivia. He wonders if she loves her
dead brother this much, how much love she would give to a husband who 
will sit like king on her three thrones: liver, brain, heart.

\subsubsection*{Quote 4/
  True, Madam. And, to comfort you with chance,\\
  Assure yourself, after our ship did split,\\
	When you and those poor number saved with you	\\
	Hung on our driving boat, I saw your brother,\\
	Most provident in peril, bind himself,\\
	Courage and hope both teaching him the practise,\\
	To a strong mast that lived upon the sea;\\
	Where, like Arion on the dolphin's back,\\
	I saw him hold acquaintance with the waves\\
	So long as I could see.
}

These lines are said by the Captain who saved Viola after the shipwreck.
He is loyal and native of IIIyria, he gives Viola money and cloths and
advice to find work with Orsino. He also assures her that her brother 
(Sebastian) is still alive and he saw 
him fighting the waves courageously like Arion. Arion is a poet in Greek
mythology who was thrown in the sea and because of his lyre the dolphins
were charmed and saved him.

\subsection*{\centering{Give very short answers to the following question}}\bigbreak

\subsubsection*{Q/Of what did Duke Orsino compare his desires to and why?}
Orsino compared his desires to \underLine{cruel hounds} because they are so intense 
and painful.

\subsubsection*{Q/What are the three sovereign thrones?}

Liver, brain and heart.

\subsubsection*{Q/Who saved Viola from shipwreck?}

An unnamed Captain.

\end{document}
